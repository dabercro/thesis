\documentclass{beamer}

\author[D. Abercrombie]{
  Daniel Abercrombie
}

\makeatletter
\g@addto@macro\bfseries{\boldmath}
\makeatother

\title{\bf \sffamily Measurement of $H\rightarrow b\bar{b}$ in \\ Associated Production with the CMS Detector}
\date{March 30, 2021}

\usecolortheme{dove}

\usepackage[absolute,overlay]{textpos}
\usefonttheme{serif}
\usepackage{appendixnumberbeamer}
\usepackage{isotope}
\usepackage{hyperref}
\usepackage[english]{babel}
\usepackage{amsmath}
\setbeamerfont{frametitle}{size=\Large,series=\bf\sffamily}
\setbeamertemplate{frametitle}[default][center]
\usepackage{siunitx}
\usepackage{tabularx}
\usepackage{makecell}
\usepackage{comment}
\usepackage{feynmp-auto}
\usepackage{multirow}

\setbeamertemplate{navigation symbols}{}
\usepackage{graphicx}
\usepackage{color}
\setbeamertemplate{footline}[text line]{\parbox{1.083\linewidth}{\footnotesize \hfill \insertshortauthor \hfill \insertpagenumber /\inserttotalframenumber}}
\setbeamertemplate{headline}[text line]{\parbox{1.083\linewidth}{\footnotesize \hspace{-0.083\linewidth} \textcolor{blue}{\sffamily \insertsection \hfill \insertsubsection}}}

\IfFileExists{/Users/dabercro/GradSchool/Presentations/MIT-logo.pdf}
             {\logo{\includegraphics[height=0.5cm]{/Users/dabercro/GradSchool/Presentations/MIT-logo.pdf}}}
             {\logo{\includegraphics[height=0.5cm]{/home/dabercro/MIT-logo.pdf}}}

\usepackage[export]{adjustbox}
\usepackage{changepage}

\newcommand{\beginbackup}{
  \newcounter{framenumbervorappendix}
  \setcounter{framenumbervorappendix}{\value{framenumber}}
}
\newcommand{\backupend}{
  \addtocounter{framenumbervorappendix}{-\value{framenumber}}
  \addtocounter{framenumber}{\value{framenumbervorappendix}}
}

\graphicspath{{figs/}}

\newcommand{\link}[2]{\href{#2}{\textcolor{blue}{\underline{#1}}}}
\newcommand{\clink}[2]{\link{#1}{http://t3serv001.mit.edu/~dabercro/redir/?k=#2}}}

\newcommand{\twofigs}[4]{
  \begin{columns}
    \begin{column}{0.5\linewidth}
      \centering
      \textcolor{blue}{#1} \\
      \includegraphics[width=\linewidth]{#2}
    \end{column}
    \begin{column}{0.5\linewidth}
      \centering
      \textcolor{blue}{#3} \\
      \includegraphics[width=\linewidth]{#4}
    \end{column}
  \end{columns}
}

\newcommand{\fourfigs}[8]{
  \begin{columns}
    \begin{column}{0.5\linewidth}
      \centering
      \textcolor{blue}{#1} \\
      \includegraphics[width=0.6\linewidth]{#2} \\
      \textcolor{blue}{#3} \\
      \includegraphics[width=0.6\linewidth]{#4}
    \end{column}
    \begin{column}{0.5\linewidth}
      \centering
      \textcolor{blue}{#5} \\
      \includegraphics[width=0.6\linewidth]{#6} \\
      \textcolor{blue}{#7} \\
      \includegraphics[width=0.6\linewidth]{#8}
    \end{column}
  \end{columns}
}

\newcommand{\threefigs}[6]{
  \centering
  \textcolor{blue}{#1} \\
  \includegraphics[width=0.35\linewidth]{#2}
  \begin{columns}
    \begin{column}{0.5\linewidth}
      \centering
      \textcolor{blue}{#3} \\
      \includegraphics[width=0.7\linewidth]{#4}
    \end{column}
    \begin{column}{0.5\linewidth}
      \centering
      \textcolor{blue}{#5} \\
      \includegraphics[width=0.7\linewidth]{#6}
    \end{column}
  \end{columns}
}

\newcommand{\ttbar}{\ensuremath{t\bar{t}}}
\newcommand{\bbbar}{\ensuremath{b\bar{b}}}

\AtBeginSection[]
{
  \begin{frame}<beamer>
    \frametitle{Outline}
    \tableofcontents[currentsection]
  \end{frame}
}

\begin{document}

\begin{frame}
  \titlepage
\end{frame}

\begin{frame}
  \frametitle{Introduction}

  \begin{columns}
    \begin{column}{0.5\linewidth}
      \begin{itemize}
      \item This work is a differential cross section of the Higgs Boson decaying to bottom quarks
      \item Discovery of $H \rightarrow b\bar{b}$ is already published
      \item Improved measurements will be done for a while to see if discrepancies arise
      \end{itemize}
    \end{column}
    \begin{column}{0.5\linewidth}
      \centering
      \textcolor{blue}{Recent result from LHC$b$}
      \includegraphics[width=\linewidth]{figures/RK2021.png}
      \tiny{https://lhcb-public.web.cern.ch/Welcome.html}
    \end{column}
  \end{columns}

\end{frame}

\begin{frame}
  \frametitle{Outline}
  \tableofcontents
\end{frame}

\section{Theory}

\begin{frame}
  \frametitle{Particles of the Standard Model}

  \centering
  \adjincludegraphics[width=0.8\linewidth,trim={0 0 0 {0.09\height}},clip]{figures/particles.png}
  \tiny{https://en.wikipedia.org/wiki/Standard\_Model}

\end{frame}

\begin{frame}
  \frametitle{Production and Decay of the Higgs}

  \begin{columns}
    \begin{column}{0.5\linewidth}
  \begin{fmffile}{gluon_fusion}
    \fmfframe(0,0)(0,5){
    \begin{fmfgraph*}(150, 120)
      \fmfleft{i0,i1}
      \fmfright{o0}
      \fmf{gluon}{i0,v0}
      \fmf{gluon}{i1,v1}
      \fmf{fermion, label=$t$}{v0,v1,v2,v0}
      \fmf{dashes, label=$H$}{v2,o0}
    \end{fmfgraph*}
    }
  \end{fmffile}
  \begin{fmffile}{vbf}
    \fmfframe(0,0)(0,0){
    \begin{fmfgraph*}(150, 120)
      \fmfleft{i0,i1}
      \fmfright{o0}
      \fmf{boson, label=$W/Z$}{i0,v0,i1}
      \fmf{dashes, label=$H$}{v0,o0}
    \end{fmfgraph*}
    }
  \end{fmffile}
    \end{column}
    \begin{column}{0.5\linewidth}
      \includegraphics[width=\linewidth]{figures/YRHXS_BR_fig1.pdf}
      \tiny{Modified from \\ https://twiki.cern.ch/twiki/bin/view/LHCPhysics/\\CERNYellowReportPageBR2010}
    \end{column}
  \end{columns}

\end{frame}

\begin{frame}
  \frametitle{$V\!H(b\bar{b})$ Process}

  \begin{center}
  \begin{fmffile}{generic_diagram}
    \fmfframe(0,0)(0, 20){
    \begin{fmfgraph*}(250, 150)
      \fmfleft{i0,i1}
      \fmfright{o0,o1,o2,o3}
      \fmf{quark}{i1,v0,i0}
      \fmflabel{$\bar{q}$}{i0}
      \fmflabel{$q$}{i1}
      \fmf{boson, label=$W/Z$}{v0,v1,v2}
      \fmf{fermion}{o0,v2,o1}
      \fmflabel{$\bar{\ell}$}{o0}
      \fmflabel{$\ell$}{o1}
      \fmf{dashes, label=$H$}{v1,v3}
      \fmf{fermion}{o2,v3,o3}
      \fmflabel{$\bar{b}$}{o2}
      \fmflabel{$b$}{o3}
    \end{fmfgraph*}
    }
  \end{fmffile}
  \end{center}

\end{frame}

\section{Experimental Apparatus}

\subsection{The LHC and CMS Detector}

\begin{frame}
  \frametitle{The Large Hadron Collider at CERN}

  \centering
  \includegraphics[width=0.8\linewidth]{figures/LHC-PHO-1986-001.pdf} \\
  \tiny{Modified from https://cds.cern.ch/record/841506}

\end{frame}

\begin{frame}
  \frametitle{The Compact Muon Solenoid (CMS) Detector}

  \centering
  \includegraphics[width=0.9\linewidth]{figures/cms_160312_02.png}
  \tiny{https://cms.cern/detector}

\end{frame}

\begin{frame}
  \frametitle{Particle Propagation Through CMS}

  \centering
  \includegraphics[width=0.9\linewidth]{figures/CMSslice_whiteBackground.png}
  \tiny{https://cds.cern.ch/record/2120661}

\end{frame}

\begin{frame}
  \frametitle{Coordinate System for $\eta$}

  \centering
  \includegraphics[width=0.8\linewidth]{figures/pictures_MuonSys-mod3.png}
  \tiny{https://cds.cern.ch/record/1456510/plots}

\end{frame}

\subsection{Particle Definitions}

\begin{frame}
  \frametitle{Isolated Leptons}

  \begin{itemize}
  \item Leptons from boson resonances will tend to be separate from other energy deposits
  \item \textcolor{blue}{$\mu (e)$ satisfies $p_T > 25 (15) \si{GeV}$}
  \item The lepton passes through the tracker with $|\eta| < 2.4$
  \item The lepton originates from the primary vertex
  \item \textcolor{blue}{$\mu (e)$ must be well isolated with $I < 0.06(0.28_\mathrm{HCAL})$}
  \item Tracks must be reasonable:
    \begin{itemize}
    \item Muons must have a clean track fit
    \item Electrons must have ECAL deposits close to track
    \end{itemize}
  \end{itemize}

  \centering
  \includegraphics[width=0.4\linewidth,page=2]{figures/cms_interactive.pdf}
  \includegraphics[width=0.4\linewidth,page=3]{figures/cms_interactive.pdf}
  \\
  \tiny{https://cds.cern.ch/record/2205172/}

\end{frame}

\begin{frame}
  \frametitle{Jets}

  \begin{columns}
    \begin{column}{0.7\linewidth}
      \begin{itemize}
      \item Jets are made of particles from decays of massive particles into hadrons
      \item The anti-$k_T$ algorithm clusters particles into jets based on
        $R = \sqrt{\eta^2 + \phi^2}$
      \item Also passed through the tracker with $|\eta < 2.4|$
      \end{itemize}
    \end{column}
    \begin{column}{0.3\linewidth}
      \includegraphics[width=\linewidth]{figures/CCrec1_05_16.jpg}
      \tiny{https://cerncourier.com/a/\\particle-flow-in-cms/}
    \end{column}
  \end{columns}


\end{frame}

\begin{frame}
  \frametitle{Two Collections}

  \begin{itemize}
  \item Use resolved ($R = 0.4$) and boosted (R = 0.8) jets
  \item $p_T > \SI{20}{GeV}$ and $p_T > \SI{250}{GeV}$
  \end{itemize}

  \begin{columns}
    \begin{column}{0.5\linewidth}
      \centering
      \textcolor{blue}{Resolved} \\
      \includegraphics[width=\linewidth]{figures/resolved.pdf}
    \end{column}
    \begin{column}{0.5\linewidth}
      \centering
      \textcolor{blue}{Boosted, PUPPI} \\
      \includegraphics[width=0.6\linewidth]{figures/boosted.pdf}
    \end{column}
  \end{columns}

\end{frame}

\begin{frame}
  \frametitle{$b$ Jets}

  \begin{itemize}
  \item $b$ hadrons have a relatively long lifetime, which can be seen in the shower vertex location
  \item $b$ jets identified with DeepCSV or a double $b$-tagger
  \end{itemize}

  \centering
  \includegraphics[width=0.8\linewidth]{figures/Illustration-of-a-heavy-flavour-jet-with-a-secondary-vertex-SV-from-the-decay-of-a-b-or.png}
  \tiny{https://www.researchgate.net/figure/fig1\_321962688}

\end{frame}

\begin{frame}
  \frametitle{$b$ Jet Energy Regression Inputs}

  A DNN was trained to improve energy resolution of $b$ jets with the following inputs:

  \begin{itemize}
  \item the jet's $p_T$, $\eta$, mass and transverse mass
  \item the event's median energy density
  \item leading lepton information,
    including perpendicular momentum,
    $\Delta R$ from the center,
    and lepton flavor
  \item the $p_T$, mass, and number of tracks from any secondary vertex linked to the jet
  \item the fractions energy in the jet due to
    charged and neutral hadrons and electromagnetic constituents
  \item the highest $p_T$ of charged hadron constituents
  \item energy fraction contained in five concentric rings
    binned by $\Delta R \in [0, 0.05, 0.1, 0.2, 0.3, 0.4]$
  \item number of PF candidates in a jet
  \item energy sharing computed by
    $\frac{\sqrt{\sum_i p_{T,i}^2}}{\sum_i p_{T,i}}$
  \end{itemize}

\end{frame}

\begin{frame}
  \frametitle{Kinematic Fit for Two Leptons}

  \begin{center}
    \includegraphics[width=0.6\linewidth]{figures/Screenshot_2020-11-30_20-55-31.png}
  \end{center}

  \begin{itemize}
  \item Events with two isolated leptons and at least two jets are varied within uncertainties to balance each other
  \end{itemize}

\end{frame}

\begin{frame}
  \frametitle{MET}

  \begin{columns}
    \begin{column}{0.4\linewidth}
      \includegraphics[width=\linewidth]{figures/met_schematic_800.png}
      \tiny{https://twiki.cern.ch/twiki/bin/view/\\CMSPublic/WorkBookMetAnalysis}
    \end{column}
    \begin{column}{0.6\linewidth}
      \begin{itemize}
      \item Momentum is conserved in the transverse direction
      \item Missing transverse momentum (MET) is the momentum invisible particles must have
        to balance visible particles
      \item MET is caused by neutrinos
      \item Mismeasured jets and other instrumental effects also contribute
      \end{itemize}
    \end{column}
  \end{columns}

\end{frame}

\section{Event Selection}

\begin{frame}
  \frametitle{Selection: 0-leptons}

  \begin{columns}
    \begin{column}{0.5\linewidth}
  \begin{fmffile}{z_zero_lep}
    \fmfframe(0, 0)(0, 0){
    \begin{fmfgraph*}(150, 120)
      \fmfleft{i0}
      \fmfright{o0,o1}
      \fmf{boson, label=$Z$}{i0,v0}
      \fmf{fermion}{o0,v0,o1}
      \fmflabel{$\bar{\nu}_e/\bar{\nu}_\mu/\bar{\nu}_\tau$}{o0}
      \fmflabel{$\nu_e/\nu_\mu/\nu_\tau$}{o1}
    \end{fmfgraph*}
    }
  \end{fmffile}
    \end{column}
    \begin{column}{0.6\linewidth}
      \begin{itemize}
      \item 0 loose leptons
      \item $p_T(V) = \mathrm{MET} > \SI{170}{GeV}$
      \item $\Delta\phi(jj(fj), V) > 2.0$ (except $t\bar{t}$)
      \item \textcolor{blue}{$p_T(jj) > \SI{120}{GeV}$}
      \item \textcolor{blue}{$p_T(j) > 60, \SI{35}{GeV}$}
      \item \textcolor{green}{$p_T(fj) > \SI{250}{GeV}$}
      \end{itemize}
    \end{column}
  \end{columns}

\end{frame}

\begin{frame}
  \frametitle{Selection: 1-lepton}

  \begin{columns}
    \begin{column}{0.5\linewidth}
  \begin{fmffile}{w_one_lep}
    \fmfframe(0, 0)(0, 0){
    \begin{fmfgraph*}(150, 120)
      \fmfleft{i0}
      \fmfright{o0,o1}
      \fmf{boson, label=$W^-$}{i0,v0}
      \fmf{fermion}{o0,v0,o1}
      \fmflabel{$\bar{\nu}_e/\bar{\nu}_\mu$}{o0}
      \fmflabel{$e^-/\mu^-$}{o1}
    \end{fmfgraph*}
    }
  \end{fmffile}
    \end{column}
    \begin{column}{0.6\linewidth}
      \begin{itemize}
      \item 1 tight lepton, 0 additional loose
      \item $p_T(V) > \SI{150}{GeV}$
      \item $\Delta\phi(jj(fj), V) > 2.5$ (except $t\bar{t}$)
      \item \textcolor{blue}{$p_T(jj) > \SI{100}{GeV}$}
      \item \textcolor{blue}{$p_T(j) > \SI{25}{GeV}$}
      \item \textcolor{green}{$p_T(fj) > \SI{250}{GeV}$}
      \end{itemize}
    \end{column}
  \end{columns}

\end{frame}

\begin{frame}
  \frametitle{Selection: 2-leptons}

  \begin{columns}
    \begin{column}{0.5\linewidth}
  \begin{fmffile}{z_two_lep}
    \fmfframe(0, 0)(0, 0){
    \begin{fmfgraph*}(150, 120)
      \fmfleft{i0}
      \fmfright{o0,o1}
      \fmf{boson, label=$Z$}{i0,v0}
      \fmf{fermion}{o0,v0,o1}
      \fmflabel{$e^+/\mu^+$}{o0}
      \fmflabel{$e^-/\mu^-$}{o1}
    \end{fmfgraph*}
    }
  \end{fmffile}
    \end{column}
    \begin{column}{0.6\linewidth}
      \begin{itemize}
      \item 2 tight leptons, 0 additional loose
      \item $p_T(V) > \SI{75}{GeV}$
      \item $\Delta\phi(jj(fj), V) > 2.5$ (except $t\bar{t}$)
      \item \textcolor{blue}{$p_T(jj) > \SI{50}{GeV}$}
      \item \textcolor{blue}{$p_T(j) > \SI{20}{GeV}$}
      \item \textcolor{green}{$p_T(fj) > \SI{250}{GeV}$}
      \end{itemize}
    \end{column}
  \end{columns}

\end{frame}

\begin{frame}
  \frametitle{Backgrounds to Signal}

  \begin{columns}
    \begin{column}{0.5\linewidth}
      \centering
      \textcolor{blue}{V+jets} \\
    \begin{fmffile}{dy_2lep_smol}
    \fmfframe(0,0)(0,0){
    \begin{fmfgraph*}(120, 100)
      \fmfleft{i0,i1}
      \fmfright{o2,o3,o0,o1}
      \fmf{quark}{i1,v0,v1,i0}
      \fmflabel{$\bar{f}$}{i0}
      \fmflabel{$f$}{i1}
      \fmf{boson, label=$Z$}{v0,v2}
      \fmf{fermion}{o0,v2,o1}
      \fmflabel{$\ell^+$}{o0}
      \fmflabel{$\ell^-$}{o1}
      \fmf{gluon, label=$g$}{v1,v3}
      \fmf{fermion}{o2,v3,o3}
      \fmflabel{$\bar{b}$}{o2}
      \fmflabel{$b$}{o3}
    \end{fmfgraph*}
    }
  \end{fmffile}
    \end{column}
    \begin{column}{0.5\linewidth}
      \centering
      \textcolor{blue}{$t\bar{t}$} \\
  \begin{fmffile}{tt_2lep_smol}
    \fmfframe(0,0)(0,0){
    \begin{fmfgraph*}(120, 100)
      \fmfleft{i0,i1}
      \fmfright{o0,o1,o2,o3,o4,o5}
      \fmf{quark}{i1,v0,i0}
      \fmflabel{$\bar{f}$}{i0}
      \fmflabel{$f$}{i1}
      \fmf{gluon, label=$g$}{v0,v1}
      \fmf{fermion, label=$\bar{t}$}{v2,v1}
      \fmf{fermion, label=$t$}{v1,v3}
      \fmf{fermion}{v2,o0}
      \fmflabel{$\bar{b}$}{o0}
      \fmf{boson, label=$W^-$}{v2,v4}
      \fmf{fermion}{o1,v4,o2}
      \fmflabel{$\bar{\nu}$}{o1}
      \fmflabel{$\ell^-$}{o2}
      \fmf{fermion}{v3,o5}
      \fmflabel{$b$}{o5}
      \fmf{boson, label=$W^+$}{v3,v5}
      \fmf{fermion}{o3,v5,o4}
      \fmflabel{$\ell^+$}{o3}
      \fmflabel{$\nu$}{o4}
    \end{fmfgraph*}
    }
  \end{fmffile}
    \end{column}
  \end{columns}


\end{frame}

\begin{frame}
  \frametitle{CR vs SR Selections}

  {
  \begin{tabular}{|l|c|c|c|c|c|}
    \hline
    Region & max $b$ & min $b$ & double $b$ & $m_{jj,SD}$ & $N_\textrm{aj}$\footnote{Not used for two-lepton channels or boosted} \\
    \hline
    Signal & med. & loose & $> 0.8$ & $>90, <150$ & $\le 1$ \\
    Z + $b$ & med. & loose & $> 0.8$ & $<90 \,\,\mathrm{or} >150$ & $\le 1$ \\
    Z + $udsg$ & !med. & loose & $< 0.8$ & $>50, <500$ & $\le 1$\footnote{0-lepton only} \\
    $t\bar{t}$ & med. & loose & $> 0.8$ & $>50, <500$ & $\ge 2$ \\
    \hline
  \end{tabular}
  }

  \vfill
  Instead of $N_\textrm{aj}$, boosted selections counted $b$ jets outside of the fat jet,
  and place any events with those into $t\bar{t}$

\end{frame}

\section{Corrections to Simulation}

\subsection{$b$-jet Momentum Smearing}


\begin{frame}
  \frametitle{Measure Jet Resolution in Data and MC}

  \begin{itemize}
  \item Take advantage of this background (with a lower $m_{jj}$):
  \end{itemize}

  \hfill

  \begin{center}
    \begin{fmffile}{DY_bjet}
      \begin{fmfgraph*}(200, 120)
	\fmfleft{i1,i0}
	\fmfright{o3,o2,o1,o0}
	\fmf{fermion, label=$q$}{i0,v0,v1,i1}
	\fmf{photon, label=$Z$}{v0,v0_1}
	\fmf{fermion, label=$\ell$}{o0,v0_1,o1}
	\fmf{gluon}{v1,v1_1}
	\fmf{fermion, label=$b$}{o2,v1_1,o3}
      \end{fmfgraph*}
    \end{fmffile}
  \end{center}

  \hfill

  \begin{itemize}
  \item The jet resolution can be measured by assuming balance
  \end{itemize}

\end{frame}

\begin{frame}
  \frametitle{Smearing Fit Strategy}

  \begin{itemize}
  \item We extrapolate to $\alpha = \frac{p_T(j_2)}{p_T(\ell\ell)} = 0$
  \end{itemize}

  \begin{columns}
    \begin{column}{0.3\linewidth}
      \centering
      \includegraphics[width=\linewidth]{/home/dabercro/GradSchool/hbb/docs/200304/figs/200303_nbjets_noenv/smearplot_1_jet1_response.pdf} \\
      \includegraphics[width=\linewidth]{/home/dabercro/GradSchool/hbb/docs/200304/figs/200303_nbjets_noenv/smearplot_3_jet1_response.pdf}
    \end{column}
    \begin{column}{0.3\linewidth}
      \centering
      \includegraphics[width=\linewidth]{/home/dabercro/GradSchool/hbb/docs/200304/figs/200303_nbjets_noenv/smearplot_2_jet1_response.pdf} \\
      \includegraphics[width=\linewidth]{/home/dabercro/GradSchool/hbb/docs/200304/figs/200303_nbjets_noenv/smearplot_4_jet1_response.pdf}
    \end{column}
  \end{columns}

\end{frame}


\begin{frame}
  \frametitle{Smearing Fit}

  We want resolution to agree at $\alpha = 0$

  \vfill

  \begin{center}
    \includegraphics[width=0.6\linewidth]{/home/dabercro/GradSchool/hbb/docs/200304/figs/200303_smear_200303_nbjets/resolution_jet1_response_smear_0.pdf}
  \end{center}

  \vfill

  Scaling is done with a linear fit of the histogram means

\end{frame}


\begin{frame}
  \frametitle{Scaling and Smearing Fit Results}

  \begin{center}
    \begin{tabular}{c|c|c}
      \hline
      Year & Scaling & Smearing \\
      \hline
      2016 & $0.998 \pm 0.019$ & $0.017 \pm 0.060$ \\
      2017 & $1.020 \pm 0.023$ & $0.088 \pm 0.071$ \\
      2018 & $0.985 \pm 0.019$ & $0.080 \pm 0.073$ \\
      \hline
    \end{tabular}
  \end{center}

  \vfill

  2016 has better calibrations to cause MC to agree with Data. \\
  It also has a different \texttt{PYTHIA} tune.


\end{frame}

\subsection{Other Corrections and Uncertainties}

\begin{frame}
  \frametitle{LO to NLO Reweighting}

  \twofigs{Before Reweighting}
          {figures/Vjets_NLOreweighting_2017V5_Znn_withoutWeight.pdf}
          {After Reweighting}
          {figures/Vjets_NLOreweighting_2017V5_Znn_withWeight.pdf}

\end{frame}

\begin{frame}
  \frametitle{Other Corrections and Uncertainties}

  \begin{itemize}
  \item Theory uncertainties
    \begin{itemize}
    \item Branching ratios
    \item QCD scale uncertainties
    \item PDF uncertainties
    \item NLO electroweak and NNLO QCD uncertainties
    \end{itemize}
  \item Experimental uncertainties
    \begin{itemize}
    \item $b$-tagging uncertainties
    \item Measured cross section of single top and diboson processes
    \item Jet energy scale uncertainties
    \item Lepton identification
    \item Trigger efficiency uncertainties
    \item Luminosity measurements
    \end{itemize}
  \end{itemize}

\end{frame}


\begin{frame}
  \frametitle{Number of Distributions in Selection}

  243 total distributions to fit for 2016, 2017, and 2018
  \vfill

  \centering
  {\tiny
  \begin{tabular}{|c|c|c|c|c|c|c|c|c|c|c|c|c|}
    \hline
    \multirow{3}{*}{$p_T(V)$ [GeV]} & \multicolumn{4}{c|}{0-leptons} & \multicolumn{4}{c|}{1-lepton} & \multicolumn{4}{c|}{2-leptons} \\
    & \multicolumn{2}{c|}{Resolved} & \multicolumn{2}{c|}{Boosted} & \multicolumn{2}{c|}{Resolved} & \multicolumn{2}{c|}{Boosted} & \multicolumn{2}{c|}{Resolved} & \multicolumn{2}{c|}{Boosted} \\
    & CR & SR & CR & SR & CR & SR & CR & SR & CR & SR & CR & SR \\
    \hline
    75 -- 150     &   &   &   &   &   &   &   &   & X & X &   &   \\
    \hline
    150 -- 250    & X &   &   &   & X & X &   &   & X &   &   &   \\
    with jet      &   & X &   &   &   &   &   &   &   & X &   &   \\
    without       &   & X &   &   &   &   &   &   &   & X &   &   \\
    \hline
    250 -- \infty & X &   & X &   & X &   & X &   & X &   & X &   \\
    250 -- 400    &   & X &   & X &   & X &   & X &   & X &   & X \\
    400 -- \infty &   & X &   & X &   & X &   & X &   & X &   & X \\
    \hline
    number X      & 2 & 4 & 1 & 2 & 2 & 3 & 1 & 2 & 3 & 5 & 1 & 2 \\
    multiplier    & 3 & 1 & 3 & 1 & 6 & 2 & 6 & 2 & 6 & 2 & 6 & 2 \\
    regions/year  & 6 & 4 & 3 & 2 & 12 & 6 & 6 & 4 & 18 & 10 & 6 & 4 \\
   \hline
  \end{tabular}
  }

  \begin{itemize}
  \item There are 3 control regions for each channel
  \item There are 2 lepton flavors for 1- and 2-lepton channels
  \item There are 81 selection regions per year
  \end{itemize}

\end{frame}

\begin{frame}
  \frametitle{Background Migration Uncertainties}

  \begin{itemize}
  \item Background events can move between adjacent bins in our differential measurement
  \item No exhaustive studies, so large uncertainties applied
  \item Some distributions (esp. $t\bar{t}$) have uncertainties up to 50\%
  \end{itemize}

\end{frame}

\section{Combination Fit}

\begin{frame}
  \frametitle{Classification DNN}

  \centering
  {\tiny
  \begin{tabularx}{\textwidth}{|l|X|c|c|c|}
    \hline
    Variable & Explanation & 0-lepton & 1-lepton & 2-lepton \\
    \hline\hline
    $m_{jj}$ & Di-jet mass & $\checkmark$ & $\checkmark$ & $\checkmark$ \\
    $p_{T,jj}$ & Di-jet transverse momentum & $\checkmark$ & $\checkmark$ & $\checkmark$ \\
    MET & Missing transverse energy & $\checkmark$ & $\checkmark$ & $\checkmark$ \\
    \hline
    $m_{T,V}$ & Vector boson transverse mass & & $\checkmark$ & \\
    $p_T(V)$ & Vector boson $p_T$ & & $\checkmark$ & $\checkmark$ \\
    $p_{T,jj}/p_T(V)$ & Redundant ratio & & $\checkmark$ & $\checkmark$ \\
    \hline
    $\Delta\phi(V, jj)$ & Azimuthal angle between vector boson and di-jet & $\checkmark$ & $\checkmark$ & $\checkmark$ \\
    $b$-tag$_\mathrm{max}$ WP & 1, 2, or 3 if higher $b$-tag discriminate meets the tight, medium, or loose working point respectively & $\checkmark$ & $\checkmark$ & $\checkmark$ \\
    $b$-tag$_\mathrm{min}$ WP & 1, 2, or 3 if lower $b$-tag discriminate meets the tight, medium, or loose working point respectively & $\checkmark$ & $\checkmark$ & $\checkmark$ \\
    \hline
    $\Delta\eta(jj)$ & $\eta$ difference between jets & $\checkmark$ & $\checkmark$ & $\checkmark$ \\
    $\Delta\phi(jj)$ & Azimuthal angle between jets & $\checkmark$ & $\checkmark$ & \\
    $p_{T, \mathrm{lead}}$ & Leading jet $p_T$ & $\checkmark$ & $\checkmark$ & $\checkmark$ \\
    \hline
    $p_{T, \mathrm{trail}}$ & Trailing jet $p_T$ & $\checkmark$ & $\checkmark$ & $\checkmark$ \\
    SA5 & Number of soft jets, $p_T > \SI{5}{GeV}$ & $\checkmark$ & $\checkmark$ & $\checkmark$ \\
    N_{aj} & Number of additional jets & $\checkmark$ & $\checkmark$ & \\
    \hline
    $b$-tag$_\mathrm{add}$ & Maximum $b$-tag of additional jets & $\checkmark$ & & \\
    $p_{T,\mathrm{add}}$ & Maximum $p_T$ of additional jets & $\checkmark$ & & \\
    $\Delta\phi(\mathrm{add, MET})$ & Azimuthal angle between additional jet and MET & $\checkmark$ & & \\
    \hline
    $\Delta\phi(\ell, \mathrm{MET})$ & Azimuthal angle between lepton and MET & & $\checkmark$ & \\
    $m_t$ & Reconstruction top mass & & $\checkmark$ & \\
    $m_V$ & Vector boson mass & & & $\checkmark$ \\
    \hline
    $\Delta R(V, jj)$ & Separation between vector boson and di-jet & & & $\checkmark$ \\
    $\Delta R_{jj}$ & Separation between jets & & & $\checkmark$ \\
    \hline
  \end{tabularx}
  }

\end{frame}

\begin{frame}
  \frametitle{Boosted BDT}

  The Boosted regions use a BDT with the following input variables for the signal region shape

  \vfill

  \begin{itemize}
  \item Soft-drop mass of the reconstructed fat jet
  \item Transverse momentum of the fat jet
  \item Transverse momentum of the reconstructed vector boson
  \item Number of soft-track jets with $p_T > \SI{5}{GeV}$
  \item Double $b$-tagger output node for boosted jets
  \end{itemize}

\end{frame}

\subsection{$V\!Z$ Cross Check}

%\begin{frame}
%  \frametitle{$V\!Z$ Goodness of Fit}
%
%  A generalized chi-squared test is compared to toys
%
%  \centering
%  \includegraphics[width=0.8\linewidth]{figures/210308_STXS_VZ_XbbVZ_e4179c95_inclusive_gof/Gof_inclusive_.pdf}
%
%\end{frame}

\begin{frame}
  \frametitle{$V\!Z$ Pre/Post-fit Distributions}

  \twofigs{Pre-fit}
          {figures/210323_STXS_VZ_unblinded_XbbVZ_8fe9e9cd_postfitplots/plot_shapes_vhbb_Zmm_1_13TeV2017_prefit}
          {Post-fit}
          {figures/210323_STXS_VZ_unblinded_XbbVZ_8fe9e9cd_postfitplots/plot_shapes_vhbb_Zmm_1_13TeV2017_postfit}

\end{frame}

\begin{frame}
  \frametitle{Shrinking Systematics}

  \includegraphics[width=0.85\linewidth,page=1]{figures/impacts/impacts_r_whhi1.pdf}

\end{frame}

\begin{frame}
  \frametitle{$V\!Z$ Distributions}

  \fourfigs{V + light}
           {figures/210323_STXS_VZ_unblinded_XbbVZ_8fe9e9cd_postfitplots/plot_shapes_vhbb_Zmm_2_13TeV2017_postfit}
           {V + heavy}
           {figures/210323_STXS_VZ_unblinded_XbbVZ_8fe9e9cd_postfitplots/plot_shapes_vhbb_Zmm_3_13TeV2017_postfit}
           {$t\bar{t}$}
           {figures/210323_STXS_VZ_unblinded_XbbVZ_8fe9e9cd_postfitplots/plot_shapes_vhbb_Zmm_4_13TeV2017_postfit}
           {Signal}
           {figures/210323_STXS_VZ_unblinded_XbbVZ_8fe9e9cd_postfitplots/plot_shapes_vhbb_Zmm_1_13TeV2017_postfit}

\end{frame}

\begin{frame}
  \frametitle{$V\!Z$ Boosted Distributions}

  \fourfigs{Boosted V + light}
           {figures/210323_STXS_VZ_unblinded_XbbVZ_8fe9e9cd_postfitplots/plot_shapes_vhbb_Wen_18_13TeV2018_postfit}
           {Boosted $t\bar{t}$}
           {figures/210323_STXS_VZ_unblinded_XbbVZ_8fe9e9cd_postfitplots/plot_shapes_vhbb_Wen_20_13TeV2018_postfit}
           {$250 < p_T(V) < \SI{400}{GeV}$}
           {figures/210323_STXS_VZ_unblinded_XbbVZ_8fe9e9cd_postfitplots/plot_shapes_vhbb_Wen_22_13TeV2018_postfit}
           {$p_T(V) > \SI{400}{GeV}$}
           {figures/210323_STXS_VZ_unblinded_XbbVZ_8fe9e9cd_postfitplots/plot_shapes_vhbb_Wen_24_13TeV2018_postfit}

\end{frame}

\begin{frame}
  \frametitle{$V\!Z$ Cross-check Results}

  \twofigs{Inclusive Scan}
          {figures/210309_inclVZ_unblinded_XbbVZ_e4179c95_a866aef8/scan_nominal_r.pdf}
          {STXS Bins}
          {figures/210308_STXS_VZ_unblinded_XbbVZ_e4179c95_a866aef8/summary_stxs.pdf}

  \begin{itemize}
  \item Inclusive p-value: 22\% (1.2$\sigma$)
  \item STXS p-value: 23\%
  \end{itemize}

\end{frame}

\subsection{$V\!H$ Analysis Results}

%\begin{frame}
%  \frametitle{$V\!H$ Goodness of Fit}
%
%  \centering
%  \includegraphics[width=0.8\linewidth]{figures/210304_STXSfine_400split_Xbb_8f854f5a_inclusive_gof/Gof_inclusive_.pdf}
%
%\end{frame}

\begin{frame}
  \frametitle{$V\!H$ Pre/Post-fit is the Same}

  \twofigs{Pre-fit}
          {figures/210322_STXSfine_400split_unblinded_Xbb_025349b6_postfitplots/plot_shapes_vhbb_Znn_15_13TeV2016_prefit.pdf}
          {Post-fit}
          {figures/210322_STXSfine_400split_unblinded_Xbb_025349b6_postfitplots/plot_shapes_vhbb_Znn_15_13TeV2016_postfit.pdf}

\end{frame}

\begin{frame}
  \frametitle{$V\!H$ Distributions}

  \fourfigs{V + light}
           {figures/210322_STXSfine_400split_unblinded_Xbb_025349b6_postfitplots/plot_shapes_vhbb_Wmn_6_13TeV2017_postfit.pdf}
           {V + heavy}
           {figures/210322_STXSfine_400split_unblinded_Xbb_025349b6_postfitplots/plot_shapes_vhbb_Wmn_7_13TeV2017_postfit.pdf}
           {$t\bar{t}$}
           {figures/210322_STXSfine_400split_unblinded_Xbb_025349b6_postfitplots/plot_shapes_vhbb_Wmn_8_13TeV2017_postfit.pdf}
           {Signal}
           {figures/210322_STXSfine_400split_unblinded_Xbb_025349b6_postfitplots/plot_shapes_vhbb_Wmn_5_13TeV2017_postfit.pdf}

\end{frame}

\begin{frame}
  \frametitle{$V\!H$ Results}

  \twofigs{Inclusive Scan}
          {figures/210309_incl_unblinded_Xbb_8f854f5a_a866aef8/scan_nominal_r.pdf}
          {STXS Bins}
          {figures/210308_STXSfine_400split_unblinded_Xbb_8f854f5a_a866aef8/summary_stxs.pdf}

  \begin{itemize}
  \item Inclusive p-value: 3.9\% (2.1$\sigma$)
  \item STXS p-value using inclusive strength: 9.3\%
  \end{itemize}

\end{frame}

%\begin{frame}
%  \frametitle{Contribution from Systematic Groups}
%
%  {\tiny
%  \begin{tabular}{|l|c|c|c|c|c|}
%    \hline
%& ZH Low & ZH no J & ZH with J & ZH high & ZH highest \\
%\hline
%Signal & {+0.049} {-0.164} & {+0.040} {-0.103} & {+0.113} {-0.280} & {+0.142} {-0.031} & {+0.130} {-0.025}\\
%Background & {+0.038} {-0.050} & {+0.000} {-0.015} & {+0.069} {-0.065} & {+0.027} {-0.027} & {+0.038} {-0.025}\\
%\hline
%$b$-tagging & {+0.131} {-0.135} & {+0.038} {-0.036} & {+0.098} {-0.093} & {+0.052} {-0.046} & {+0.058} {-0.043}\\
%Jet energy & {+0.050} {-0.048} & {+0.019} {-0.030} & {+0.101} {-0.093} & {+0.045} {-0.040} & {+0.070} {-0.069}\\
%Lepton ID & {+0.032} {-0.032} & {+0.007} {-0.021} & {+0.000} {-0.067} & {+0.018} {-0.012} & {+0.028} {-0.022}\\
%LOtoNLO & {+0.154} {-0.162} & {+0.035} {-0.039} & {+0.113} {-0.083} & {+0.044} {-0.040} & {+0.045} {-0.025}\\
%Luminosity & {+0.016} {-0.020} & {+0.000} {-0.012} & {+0.017} {-0.014} & {+0.023} {-0.023} & {+0.039} {-0.012}\\
%Meas. XS & {+0.000} {-0.016} & {+0.000} {-0.000} & {+0.000} {-0.000} & {+0.026} {-0.016} & {+0.004} {-0.000}\\
%Triggers & {+0.024} {-0.010} & {+0.000} {-0.000} & {+0.029} {-0.000} & {+0.000} {-0.010} & {+0.023} {-0.013}\\
%$p_T(V)$ Mig. & {+0.362} {-0.387} & {+0.091} {-0.079} & {+0.232} {-0.210} & {+0.133} {-0.121} & {+0.252} {-0.235}\\
%\hline
%  \end{tabular}
%
%  \begin{tabular}{|l|c|c|c|}
%    \hline
%& WH med. & WH high & WH highest \\
%\hline
%Signal & {+0.060} {-0.000} & {+0.117} {-0.034} & {+0.223} {-0.117}\\
%Background & {+0.052} {-0.047} & {+0.053} {-0.025} & {+0.084} {-0.060}\\
%\hline
%$b$-tagging & {+0.128} {-0.118} & {+0.084} {-0.069} & {+0.092} {-0.077}\\
%Jet energy & {+0.225} {-0.203} & {+0.090} {-0.081} & {+0.116} {-0.076}\\
%Lepton ID & {+0.022} {-0.000} & {+0.054} {-0.042} & {+0.023} {-0.047}\\
%LOtoNLO & {+0.047} {-0.038} & {+0.035} {-0.005} & {+0.050} {-0.050}\\
%Luminosity & {+0.024} {-0.000} & {+0.045} {-0.000} & {+0.066} {-0.043}\\
%Meas. XS & {+0.034} {-0.000} & {+0.039} {-0.000} & {+0.047} {-0.057}\\
%Triggers & {+0.011} {-0.000} & {+0.022} {-0.035} & {+0.034} {-0.048}\\
%$p_T(V)$ Mig. & {+0.165} {-0.156} & {+0.133} {-0.123} & {+0.305} {-0.293}\\
%\hline
%  \end{tabular}
%  }
%
%\end{frame}

\begin{frame}
  \frametitle{Cross Section}

  In the kappa framework $\kappa_V = \kappa_b = 1.0$ in the Standard Model

  \begin{columns}
    \begin{column}{0.5\linewidth}
      \includegraphics[width=\linewidth]{figures/limit.pdf}
    \end{column}
    \begin{column}{0.5\linewidth}
  \begin{fmffile}{kappa_diagram}
    \fmfframe(0,0)(0,0){
    \begin{fmfgraph*}(150, 100)
      \fmfleft{i0,i1}
      \fmfright{o0,o1}
      \fmf{boson,label=$V$}{i0,v0,i1}
      \fmf{dashes,label=$H$}{v0,v1}
      \fmf{fermion,label=$b$}{o0,v1,o1}
      \fmflabel{$\kappa_V$}{v0}
      \fmflabel{$\kappa_b$}{v1}
    \end{fmfgraph*}
    }
  \end{fmffile}
    \end{column}
  \end{columns}

  To first order, cross section scales with $\kappa_V^2 \times \kappa_b^2$

\end{frame}

\section{}

\begin{frame}
  \frametitle{Conclusion}

  \begin{itemize}
  \item A method for cross section measurement was developed
  \item Gives agreement with the Standard Model in $V\!Z$
  \item There is some deviation from the Standard Model
    \begin{itemize}
    \item The inclusive signal strength is low
    \item There are trends at higher $p_T(V)$
    \end{itemize}
  \item The deviation is not large enough to conflict with the Standard Model
  \end{itemize}

\end{frame}

\begin{frame}
  \frametitle{Thank You to My Collaborators}

  {\bf DESY}: A. de Wit, R. Mankel, S. Kaveh, A. Nigamova \\
  \vfill
  {\bf ETH}: C. Grab, A. Calandri, P. Berger, G. Perrin, K. Gedia, \\
  \qquad \quad C. Reissel, B. Kaech, L. Perrozzi \\
  \vfill
  {\bf INFN Pisa}: A. Rizz \\
  \vfill
  {\bf MIT}: \emph{D. Abercrombie}, G. G\'omez-Ceballos, D. Kovalskyi, \\
  \qquad \quad B. Maier, C. Paus \\
  \vfill
  {\bf Princeton}: J. Olsen, N. Haubrich, C. Palmer \\
  \vfill
  {\bf U. Florida}: J. Konigsberg \\
  \vfill
  {\bf RBI}: V. Brigljevic, B. Chitroda, D. Ferencek, \\
  \qquad \quad S. Mishra, T. Susa \\
  \vfill
  {\bf RWTH}: A. Schmidt \\

\end{frame}

\begin{comment}
\beginbackup

\begin{frame}
  \centering
    {\Huge \bf\sffamily Backup Slides}
\end{frame}


\backupend
\end{comment}

\end{document}
