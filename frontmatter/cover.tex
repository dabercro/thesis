% -*-latex-*-
% 
% For questions, comments, concerns or complaints:
% thesis@mit.edu
% 
%
% $Log: cover.tex,v $
% Revision 1.9  2019/08/06 14:18:15  cmalin
% Replaced sample content with non-specific text.
%
% Revision 1.8  2008/05/13 15:02:15  jdreed
% Degree month is June, not May.  Added note about prevdegrees.
% Arthur Smith's title updated
%
% Revision 1.7  2001/02/08 18:53:16  boojum
% changed some \newpages to \cleardoublepages
%
% Revision 1.6  1999/10/21 14:49:31  boojum
% changed comment referring to documentstyle
%
% Revision 1.5  1999/10/21 14:39:04  boojum
% *** empty log message ***
%
% Revision 1.4  1997/04/18  17:54:10  othomas
% added page numbers on abstract and cover, and made 1 abstract
% page the default rather than 2.  (anne hunter tells me this
% is the new institute standard.)
%
% Revision 1.4  1997/04/18  17:54:10  othomas
% added page numbers on abstract and cover, and made 1 abstract
% page the default rather than 2.  (anne hunter tells me this
% is the new institute standard.)
%
% Revision 1.3  93/05/17  17:06:29  starflt
% Added acknowledgements section (suggested by tompalka)
% 
% Revision 1.2  92/04/22  13:13:13  epeisach
% Fixes for 1991 course 6 requirements
% Phrase "and to grant others the right to do so" has been added to 
% permission clause
% Second copy of abstract is not counted as separate pages so numbering works
% out
% 
% Revision 1.1  92/04/22  13:08:20  epeisach

% NOTE:
% These templates make an effort to conform to the MIT Thesis specifications,
% however the specifications can change. We recommend that you verify the
% layout of your title page with your thesis advisor and/or the MIT 
% Libraries before printing your final copy.
\title{Measurement of $H\rightarrow b\bar{b}$ in Associated Production with the CMS Detector}

\author{Daniel Robert Abercrombie}
% If you wish to list your previous degrees on the cover page, use the 
% previous degrees command:
%       \prevdegrees{A.A., Harvard University (1985)}
% You can use the \\ command to list multiple previous degrees
       \prevdegrees{B.S., Pennsylvania State University (2014)}
%                    S.M., Massachusetts Institute of Technology (1981)}
\department{Department of Physics}

% If the thesis is for two degrees simultaneously, list them both
% separated by \and like this:
\degree{Doctor of Philosophy}
% \degree{Bachelor of Science in Computer Science and Engineering}

% As of the 2007-08 academic year, valid degree months are September, 
% February, or June.  The default is June.
\degreemonth{June}
\degreeyear{2021}
\thesisdate{\today}

%% By default, the thesis will be copyrighted to MIT.  If you need to copyright
%% the thesis to yourself, just specify the `vi' documentclass option.  If for
%% some reason you want to exactly specify the copyright notice text, you can
%% use the \copyrightnoticetext command.  
%\copyrightnoticetext{\copyright IBM, 1990.  Do not open till Xmas.}

% If there is more than one supervisor, use the \supervisor command
% once for each.
\supervisor{Christoph M. E. Paus}{Professor of Physics}

% This is the department committee chairman, not the thesis committee
% chairman.  You should replace this with your Department's Committee
% Chairman.
\chairman{Deepto Chakrabarty}{Associate Department Head, Physics}

% Make the titlepage based on the above information.  If you need
% something special and can't use the standard form, you can specify
% the exact text of the titlepage yourself.  Put it in a titlepage
% environment and leave blank lines where you want vertical space.
% The spaces will be adjusted to fill the entire page.  The dotted
% lines for the signatures are made with the \signature command.
\maketitle

% The abstractpage environment sets up everything on the page except
% the text itself.  The title and other header material are put at the
% top of the page, and the supervisors are listed at the bottom.  A
% new page is begun both before and after.  Of course, an abstract may
% be more than one page itself.  If you need more control over the
% format of the page, you can use the abstract environment, which puts
% the word "Abstract" at the beginning and single spaces its text.

%% You can either \input (*not* \include) your abstract file, or you can put
%% the text of the abstract directly between the \begin{abstractpage} and
%% \end{abstractpage} commands.

% First copy: start a new page, and save the page number.
\cleardoublepage
% Uncomment the next line if you do NOT want a page number on your
% abstract and acknowledgments pages.
% \pagestyle{empty}
\setcounter{savepage}{\thepage}
\begin{abstractpage}
% $Log: abstract.tex,v $
% Revision 1.1  93/05/14  14:56:25  starflt
% Initial revision
% 
% Revision 1.1  90/05/04  10:41:01  lwvanels
% Initial revision
% 
%
%% The text of your abstract and nothing else (other than comments) goes here.
%% It will be single-spaced and the rest of the text that is supposed to go on
%% the abstract page will be generated by the abstractpage environment.  This
%% file should be \input (not \include 'd) from cover.tex.
The differential cross section of $V\!H \rightarrow b\bar{b}$ is measured with the CMS Detector.
The Simplified Template Cross Section framework is used.
The inclusive strength of the measured signal relative to the Standard Model is
$0.568^{+0.154}_{-0.147} \mathrm{(stat)}^{+0.134}_{-0.133} \mathrm{(sys)}$,
which agrees with the Standard Model within 2.1 standard deviations.
The measured spectrum of the recoiling vector boson transverse momentum
has a $p$-value of 9.3\%, assuming Standard Model predictions at the measured signal strength.

\end{abstractpage}

% Additional copy: start a new page, and reset the page number.  This way,
% the second copy of the abstract is not counted as separate pages.
% Uncomment the next 6 lines if you need two copies of the abstract
% page.
% \setcounter{page}{\thesavepage}
% \begin{abstractpage}
% % $Log: abstract.tex,v $
% Revision 1.1  93/05/14  14:56:25  starflt
% Initial revision
% 
% Revision 1.1  90/05/04  10:41:01  lwvanels
% Initial revision
% 
%
%% The text of your abstract and nothing else (other than comments) goes here.
%% It will be single-spaced and the rest of the text that is supposed to go on
%% the abstract page will be generated by the abstractpage environment.  This
%% file should be \input (not \include 'd) from cover.tex.
The differential cross section of $V\!H \rightarrow b\bar{b}$ is measured with the CMS Detector.
The Simplified Template Cross Section framework is used.
The inclusive strength of the measured signal relative to the Standard Model is
$0.568^{+0.154}_{-0.147} \mathrm{(stat)}^{+0.134}_{-0.133} \mathrm{(sys)}$,
which agrees with the Standard Model within 2.1 standard deviations.
The measured spectrum of the recoiling vector boson transverse momentum
has a $p$-value of 9.3\%, assuming Standard Model predictions at the measured signal strength.

% \end{abstractpage}

\cleardoublepage

\section*{Acknowledgments}

My work was partially funded by the National Science Foundation's
Graduate Research Fellowship Program and
the Otis Fellowship from the MIT Department of Physics.
I appreciate the recognition that both of these programs bestowed upon me.

I also would like to thank Christoph Paus, my thesis advisor,
for giving me immense freedom to explore and educate myself during my time at MIT and
for his patience while I eventually got around to finishing up this thesis.

I have the usual fear of leaving out individuals if I were to continue this section with names,
so instead I will try to paint my positive experiences of the past seven years in broad strokes.
Those who knew me will know where they fit in.
Before arriving at MIT, I possessed a solid foundation in physics education.
This was thanks to instructors and classmates that I had in high school,
while taking classes at Lycoming College, and while obtaining my Bachelor degrees at Penn State.
When I started at MIT,
I met many exceptional students whom I explored Boston with at bars and concerts.
Friends that I shared classes with continued to aid my education,
and those I shared offices or floors with also aided in my research.

I also had a positive experience outside of work.
I enjoyed my time in the Isshinryu Karate-do Club and
in the IM sports of Air Pistol, Tennis, and Softball.
I then found one of my favorite hobbies
thanks to a welcoming group of powerlifters at the MIT gym.
Despite being separated from this first group of ``gym bros'' due to the events of 2020,
I was able to meet other motivational and kind people in weight rooms in Indiana and New Mexico.
The climbers I met in Albuquerque have also encouraged me to experiment in one of the popular local hobbies.
Outside of physical endeavors,
a group of friends rescued me from creative torpidity by participating in games of Pathfinder.
I also want to thank those friends from high school and Penn State that kept in touch
through regular video calls and gave me a perspective of life outside of MIT at that time.

To end, I will return to individual names for those that I do and did consider immediate family.
Though it is painful to remember how we grew apart as we learned who we were and what we wanted,
Liang Yu had a large impact on my experience at MIT,
and I should not neglect to acknowledge our brief marriage here.
My oldest brother Michael offered support and encouragement when I needed it, and
my other brother Luke always reminded me to have fun while I can.
Their significant others, Lacey and Paula, are wonderful people that I enjoy seeing during
holidays and other family events.
I am forever indebted to my sister and her husband, Sarah and Peter Vaughn,
for providing me with a place to stay during the 2020 COVID lock-down.
They, their dogs, Remi and Willow, and their friends gave me a glimpse of how I could experience
the American dream after finishing my graduate program.
I also want to thank the rest of the Vaughns in Indiana,
Erik, Christine, Sara, Ben, and Nancy, for making me feel like family.
I look forward to seeing them all again when I return to meet Henry.
Finally, I want to thank my parents, David and Diane Abercrombie.
They contributed as much as anyone else for my personal foundation that
prepared me for my time at MIT.
I could also tell that they were very curious, but I appreciate that
they did not fall into the typical parental nagging as I finished up this thesis.

%%%%%%%%%%%%%%%%%%%%%%%%%%%%%%%%%%%%%%%%%%%%%%%%%%%%%%%%%%%%%%%%%%%%%%
% -*-latex-*-
