\chapter{Event Selection}

\section{Object Definitions}

Section~\ref{sec:event-reco} describes how
detector responses are linked to possible physical particles.
We want to remove false positives,
so here are some tighter requirements for counting for event selection.

Once we have our objects defined,
we can count them.

\subsection{Muons}

Muons can show up in weakly decaying jets,
so we're only interested in isolated ones here.

\subsection{Electrons}

Electrons do the same thing as muons,
but messier because the ECAL isn't as clean as the muon chambers.

\subsection{Jets}

Jets are messier still.

Pileup removal is a big deal here.

\subsection{MET}

MET is corrected.

\subsection{Undesirable Particles}

There are certain particles that we do not want present.
We make very loose selections for those and veto on them.

\subsubsection{Photons}

\subsubsection{Tau Leptons}

\section{Removal of QCD}

We have some cuts across the board on our objects
in order to remove events that are just QCD.

\section{Categories of Vector Boson Decay}

Now that we are ready to count,
we can count leptons in order to characterise
potential vector bosons.

\subsection{0 Leptons}

\subsection{1 Lepton}

\subsection{2 Leptons}

\section{Topology of Higgs Decay}

\subsection{Resolved Jets}

We reconstruct two $b$ jets.

\subsection{Boosted Jet}

When the Higgs has very high $p_T$,
the jet clustering algorithms can find both daughter particles
as being part of a single jet.
