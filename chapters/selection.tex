\chapter{Event Selection} \label{ch:selection}

This chapter gives the specific selection requirements on each physics object
that allows us to count particle candidates and to reject or otherwise classify events,
based on the decription of physics processes described thus far.
First, objects are defined in terms of variables and particle candidates provided by the
detector reconstruction algorithms.
Then selection requirements based on these objects used to reject events entirely
from the analysis are given.
After that, selection requirements used to classify events
into different decay channels of the vector boson are specified.
There are also selection requirements that allow events to be treated separately
when the Higgs decay products can be resolved as separate jets
and where they are merged into a single massive jet.

\section{Object Definitions}

Detector responses are linked to possible physical particles.
Most of the particle ID techniques described so far can give
false positives for individual particle candidates
or provide composite physics objects that are in reality composed of background particles.
What follows are tighter selections used in order to reduce these backgrounds.
Once objects are more strictly defined,
they can be used for more reliable event classification.

Each type of object generally has a method of loose pre-selection
and additional tighter requirements for a selection.
The distinction is particularly useful for categorizing events.
Each category is designed to be enriched with a particular physics process.
Each physics process would result in certain final states
with specific multiplicities for some particles.
For objects, passing the loose selection often means that the object is defined
well enough to veto events for inclusion in categories that
would not include the corresponding particle.
Additional selection requirements are added for objects to classify
as a particular particle candidate in order to reduce false positives
of events that are included in a given category.

\subsection{Variable Definitions}

Many of the object definitions use variables that are derived from reconstructed quantities.
They can be understood in terms of the reconstruction
described in Section~\ref{sec:event-reco}.
Lepton isolation is quantified using the following formula.
\begin{gather}
  I = \frac{1}{p_T^\ell} \left(\sum p_T^\mathrm{charged} +
      \max\left[0, \sum p_T^\mathrm{neutral} +
               \sum p_T^\mathrm{\gamma} - p_T^\mathrm{PU}
               \right]\right) \label{eq:isolation}
\end{gather}
The sums are over charged hadrons originating from the primary vertex
and all neutral hadrons and photons within a distance of $\Delta R < 0.4$
from the lepton if it is a muon, or $\Delta R < 0.3$ from an electron,
where $\Delta R$ is a distance on the $(\eta, \phi)$ plane.
\begin{gather}
  \Delta R = \sqrt{\Delta\eta^2 + \Delta\phi^2}
\end{gather}
The term $p_T^\mathrm{PU}$ is defined as the following for muons.
\begin{gather}
  p_T^\mathrm{PU} = 0.5 \times \sum p_T^{\mathrm{PU}, \mathrm{charged}}
\end{gather}
Electrons use a different definition.
\begin{gather}
  p_T^\mathrm{PU} = \rho \times A_\mathrm{eff}
\end{gather}
$A_\mathrm{eff}$ is the area of the isolation cone,
and $\rho$ is the median of the $p_T$ density of neutral particles in that area.

Particles can also be defined as coming from the primary vertex of an event or from pileup.
Vertices are defined through deterministic annealing \cite{726788},
using the closest approach of tracks to the beamline \cite{Collaboration_2014}.
The primary vertex is the vertex with
the greatest sum of $E_T$ of the charged particles originating from it.
After identification of the primary vertex,
charged particles are classified as originating from the primary vertex or as pileup
using their extrapolated track's distance in the transverse plane, $d_{xy}$
and distance along the beamline, $d_z$.

\subsection{Muons} \label{sec:muon-def}

An isolated muon gives one of the cleanest signatures in CMS,
with only perhaps the exception of an isolated photon that does not
undergo pair production in the pixel tracker.
Muons can also show up in jets from weakly decaying hadrons,
in which case they are not isolated.
Since weakly decaying $b$ jets are central to this analysis,
events with non-isolated leptons are not rejected,
but the distinction is important.
Loosely selected muons must meet the following requirements.
\begin{itemize}
\item The muon must have a relatively high energy of $p_T > \SI{5}{GeV}$.
\item The muon should pass through the inner tracker within $|\eta| < 2.4$.
\item The muon originates from the primary vertex, satisfying both
  $d_{xy} < \SI{0.5}{cm}$ and $d_z < \SI{1.0}{cm}$.
\item The muon must pass a loose isolation cut of $I < 0.4$.
\item The muon must be a PF muon.
\item The muon is either a global muon or a tracker muon.
\end{itemize}
Tightly identified muons have some additional cuts they must pass.
\begin{itemize}
\item They must have a higher transverse momentum at $p_T > \SI{25}{GeV}$.
  In events with two muons, such as caused by $Z \rightarrow \mu\mu$,
  one muon only needs to satisfy the slightly looser selection of $p_T > \SI{15}{GeV}$,
  as long as the other has $p_T > \SI{25}{GeV}$.
\item The muon must be a global muon, leaving tracks in both the central tracker
  and the muon chambers.
\item There must be more than five hits in the inner tracker with one hit on a pixel.
\item The fit for the global muon track must be good with $\chi^2/ndof < 10$.
\item The muon must be well isolated with $I < 0.06$
\end{itemize}

These definitions are accepted by all members of the CMS collaboration
as loose and tight working points, respectively.
This allows the analysis to use efficiency measurements created for wider use.

\subsection{Electrons} \label{sec:ele-def}

The kinematic variables associated with an electron are extracted from the GSF fit.
Loosely selected electrons must meet the following requirements.
\begin{itemize}
\item They must have a transverse momentum satisfying $p_T > \SI{7}{GeV}$.
\item They should be centered in the detector with $|\eta| < 2.4$.
\item The distance from the primary vertex is limited, requiring
  $d_{xy} < \SI{0.05}{cm}$ and $d_Z < \SI{0.2}{cm}$.
\item They pass a loose isolation cut of $I < 0.4$.
\end{itemize}
To optimize the electron selection,
electrons are identified with the aid of an MVA \cite{Rembser_2019}.
Fully selected electrons pass the tight working point used by the CMS collaboration.
In order to also match the samples of simulated electrons used in the training sample,
the selected electrons also must pass the following cuts.
\begin{itemize}
\item The electron must have higher energy with $p_T > \SI{15}{GeV}$.
\item The deposit of HCAL energy must be less than 9\% of the ECAL energy deposit
  along the electron track.
\item The track sum $p_T$ component of the isolation must be
  less than 18\% of the electron $p_T$.
\item There is a gap in the ECAL geometry,
  so the electron must either have $|\eta| < 1.4442$ or $|\eta| > 1.5660$.
\item For electrons with $|\eta| < 1.4442$:
  \begin{itemize}
  \item The shower shape must satisfy $\sigma_{i\eta i\eta} < 0.012$
  \item Isolation in the ECAL cluster must be less than 0.4,
    and isolation in the HCAL must be less than 0.25.
  \item The difference between super cluster and track location of the electron
    must be small with $\Delta \eta < 0.0095$ and $\Delta \phi < 0.065$.
  \end{itemize}
\item For electrons with $|\eta| > 1.5660$:
  \begin{itemize}
  \item The shower shape must satisfy $\sigma_{i\eta i\eta} < 0.033$
  \item Isolation in the ECAL cluster must be less than 0.45,
    and isolation in the HCAL must be less than 0.28.
  \end{itemize}
\end{itemize}

\subsection{Jets} \label{sec:jets-def}

Strong interactions cause jets of particles when quarks or gluons hadronize.
Conservation of energy and momentum means that the sum of jet constituents
give the kinematics of the initial parton that produced them.
Jets are constructed by clustering all particle-flow candidates
with the anti-$k_T$ algorithm \cite{Cacciari_2008}
using the jet clustering parameter $R = 0.4$.
Due to factors like pileup and imperfect detector response,
the energy of the reconstructed jets are corrected \cite{Khachatryan_2017}.

Loose jet cuts, based on the constituents,
are applied to remove jets constructed from detector noise.
Jets that get a significant fraction of their energy from pileup are also removed.
Pre-selected electrons and muons are also often reconstructed as jets.
Any jet within $\Delta R < 0.4$ from a pre-selected lepton is removed.

To be considered for the study of decay products of the Higgs boson,
jets must be within the inner tracker of the detector with $|\eta| < 2.5$.
This allows pileup to be removed and for accurate vertexing of the consituents.
The jets must satisfy $p_T > \SI{25}{GeV}$ for the zero and one lepton signatures.
The two lepton signature from $Z(\ell\ell)H$ is cleaner,
and looser jet selection criteria of $p_T > \SI{20}{GeV}$ are applied to the jets.

\subsection{Identification of $b$ Jets and Energy Regression}

Jets containing $b$ hadrons have a distinct signature.
This includes secondary vertices displaced from the beamline,
as well as non-isolated leptons from weak decays.
When looking at jets in the inner tracker,
all of these features can be considered in a deep neural network (DNN)
called Deep Combined Secondary Vertex (DeepCSV)
designed to identify $b$ jets \cite{Sirunyan_2018}.
The output of DeepCSV has three working points
that are defined based on the amount of false positives that can be expected
in a collection of jets passing the cut.
The specific values are different for each year of operation.
The detector and collision conditions change,
and a separate model is trained each year to account for that.
The working points for each year are given in Table~\ref{tab:deepcsv}.
\begin{table}
  \centering
  \caption[DeepCSV working points]{
    The minimal cut value on the neural network output for each DeepCSV working point
    are defined for each year of Run~2 of the LHC.
    The working points are defined by their mis-tag rates.
  }
  {\renewcommand{\arraystretch}{1.5}
  \begin{tabular}{|c | c | c c c|}
    \hline
    Working Point & Mistag Rate & 2016 & 2017 & 2018 \\
    \hline
    Loose  &  10\% & 0.2219 & 0.1522 & 0.1241 \\
    Medium &   1\% & 0.6324 & 0.4941 & 0.4184 \\
    Tight  & 0.1\% & 0.8958 & 0.8001 & 0.7527 \\
    \hline
  \end{tabular}
  }
  \label{tab:deepcsv}
\end{table}

The non-isolated leptons within $b$ jets are caused
by flavor-changing weak decay of $b$ hadrons.
This decay mode also results in neutrinos which carry away a portion of the jet energy.
In order to more accurately reconstruct the di-jet mass of a candidate Higgs decay,
a prediction on the amount of energy carried away by undetected neutrinos is made.
A Deep Neural Network (DNN) is trained in Tensorflow
\cite{DBLP:journals/corr/AbadiABBCCCDDDG16},
which is designed to improve the energy measurement and resolution of all $b$ jets for CMS
\cite{collaboration2019deep}.
The following variables are used as inputs to the regression:
\begin{itemize}
\item the jet's $p_T$, $\eta$, mass and transverse mass
\item the event's median energy density, commonly denoted as $\rho$
\item information about the hardest lepton clustered into the jet,
  including momentum perpendicular to the jet,
  distance $\Delta R$ from the center of the jet,
  and the lepton's flavor
\item the $p_T$, mass, and number of tracks from any secondary vertex linked to the jet,
  as well as the secondary vertex's distance from the collision point and
  associated uncertainties
\item the fractions energy in the jet due to
  charged and neutral hadrons and electromagnetic consistuents
\item the highest $p_T$ of charged hadron consituents
\item the energy fraction contained in five concentric rings around the jet center
  binned by $\Delta R \in [0, 0.05, 0.1, 0.2, 0.3, 0.4]$
\item number of PF candidates in a jet
\item energy sharing computed by
  \[
  \frac{\sqrt{\sum_i p_{T,i}^2}}{\sum_i p_{T,i}}}
  \]
  where $i$ runs over the jet constituents
\end{itemize}
This list results in 41 input variables for the DNN.
All selections depending on the $p_T$ values of $b$-tagged jets are evaluated after
applying the $b$-jet energy regression.

\subsection{Fat Jets}

When the Higgs boson is highly boosted, its decay products are closer together
in the frame of the detector.
At some point, the jets from the $b$ quarks would not be clustered separately.
These highly boosted events are interesting for the differential cross section measurements,
and they are more enriched with the signal associated production process than
events with lower $p_T$ intermediate particles.
To not loose the events, single jets are analyzed for evidence of containing two $b$ hadrons.
In order to handle the transistion from resolved jets to boosted single jets,
larger jets, labelled fat jets, are used.
A second collection of jets is made from the same set of PF candidates that are
clustered to create the jets described previously.
This collection also uses the anti-$k_T$ algorithm,
but with the jet clustering parameter $R = 0.8$.
The requirement on the fat jet which ensures that Higgs decay products are contained
in the jet when they are at a maximum opening angle is $p_T > \SI{250}{GeV}$.

Being significantly larger in area,
the fat jets also contain much additional radiation from the underlying event.
As a result, the mass of the constituents collected into the jet
is also significantly larger than the original mass of the primary parent particle
whose daughters make up most of the jet constituents.
A number of grooming algorithms,
which are designed to remove particles from pileup from the jets,
were considered within CMS \cite{dabercro2014}.
The soft drop algorithm \cite{Larkoski_2014} was chosen as the standard in the experiment
and is used in this analysis.
The resulting groomed mass of the jet is close to its original parent particle.
The soft drop algorithm has the additional benefit of forcing pileup jets to low mass,
which is why a second requirement on fat jets considered for the analysis is
$m_\mathrm{SD} > \SI{50}{GeV}$.

\subsection{Missing Transverse Energy}

Missing transverse energy, which is actually the missing transverse momentum,
also labelled $E_T^\mathrm{miss}$ or MET,
is a vector that takes advantage of the fact that momentum
transverse to the beamline is conserved.
Particle flow MET is calculated by taking the negative vector sum of the
transverse momentum of all particle flow candidates in the event.
The resulting vector is then adjusted by taking into account the difference
between the uncorrected and corrected jet energies \cite{collaboration_2015}.
The resulting magnitude and direction is a proxy for the transverse momentum of
any neutrinos in the event.
Since particle flow MET is the type of MET that is used most often in this analysis,
it is referred to throughout as just MET.
Large MET values can be generated by instrumental and beam effects as well.
Therefore, there are additional event filters applied to events with large MET
that removes events where these known instrumental and beam effects have been identified.

One technique outside of using the event filter lists from dedicated CMS groups
is to compare the particle flow MET to other calculations of MET.
This analysis also uses track MET, or trkMET.
Track MET is calculated using just the reverse vector sum of charged particles
that are detected by the tracker.
In collisions without leptons,
charged and neutral particles energy distributions are similar in the detector,
so differences in track and particle flow MET directions can indicate
instrumental failures in purely hadronic environments.

\subsection{Soft Hadronic Activity}

In signal $VH$ events, hadronic activity outside of the $b\bar{b}$ decay of the Higgs
is expected to be low.
This hadronic activity is defined by considering the
additional charged PF tracks coming from the primary vertex.
An exclusion region is defined in an ellipse in $(\eta, \phi)$ space
containing the two selected $b$ jets
with a major axis length of $\Delta R (\bb) + 1$ and a minor axis length of 1.
All charged tracks outside of this ellipse that also do not correspond with
the selected leptons and that satisfy $p_T > \SI{300}{MeV}$ and $d_Z < \SI{0.2}{cm}$
are clustered using the anti-$k_T$ algorithm \cite{Cacciari_2008} with $R = 0.4$.
The resulting collection of soft jets is used to define four variables:
\begin{itemize}
\item $H_T^\mathrm{soft}$ -- The scalar sum of soft jets' $p_T$ for
  jets with $p_T > \SI{1}{GeV}$
\item $N_2^\mathrm{soft}$ -- The number of soft jets with $p_T > \SI{2}{GeV}$
\item $N_5^\mathrm{soft}$ -- The number of soft jets with $p_T > \SI{5}{GeV}$
\item $N_{10}^\mathrm{soft}$ -- The number of soft jets with $p_T > \SI{10}{GeV}$
\end{itemize}
These variables are used in the training of the BDT that discriminates
signal and background events.

\subsection{Kinematic Fit}

In the two-lepton region, the $Z$ boson's decay products are observed directly, and
the $Z$ boson momentum can be reconstructed precisely.
A kinematic fit is performed to constrain the momenta of the two leptons by requiring their
combined mass to match the $Z$ boson mass.
After this constraint is applied, the tranvserve momentum of the dijet system,
along with ISR jets, is balanced with the transverse momentum of the dilepton system.
The fit is performed by minimizing the chi-squared of the kinematic system that
meets these constraints.

For electrons and muons,
the corrections defined in centralized efforts by the CMS collaboration include uncertainties.
The energy uncertainty used for the $b$-jets and recoil jets is the same as that used
by the $HH \rightarrow \bb\bb$ analysis from CMS \cite{Sirunyan_2019}.
The mass of the $Z$ boson is given a Gaussian uncertainty of \SI{5}{GeV},
and it is assumed that the MET in the event is 0.
The kinematic fit minimizes the chi-squared value of these constraints,
resulting in new energies for all particles in the fit.
The full implementation is in the \texttt{PhysicsTools/KinFitter} package of
\texttt{CMSSW\_10\_2\_0\_pre3} \cite{cmssw_doxygen}.
Only the resulting energies of the $b$-jets, which are otherwise relatively loosely constrained,
are used for analyzing the two-lepton regions.
The improvement of the di-jet mass in the signal sample as a result of the kinematic fit
are shown in Figure~\ref{fig:kinfit} and Table~\ref{tab:kinfit}.

\begin{figure}
  \includegraphics[width=\linewidth]{figures/fits_SR_medhigh_Hmass__.pdf}
  \caption[Higgs di-jet mass fit with kinematic fit]{
    The Higgs di-jet mass in the 2-lepton signal samples is shown above.
    Peaks from the raw jet, the regressed jet energy, and the kinematic fit are compared.
  }
  \label{fig:kinfit}
\end{figure}

\begin{table}
  \centering
  \caption[Mass resolutions after kinematic fit]{
    The value of $\sigma/\mu$ for each fitted di-jet mass peak is shown below.
    Figure~\ref{fig:kinfit} shows the mass peaks that were fit to fill the inclusive column.
  }
  \begin{tabular}{|r|c|c|c|}
    \hline
    & Low $p_T$ & High $p_T$ & incl. \\
    \hline
    No R. & 0.157 & 0.133 & 0.150 \\
    Reg.  & 0.134 & 0.121 & 0.130 \\
    Fit   & 0.129 & 0.112 & 0.124 \\
    \hline
  \end{tabular}
  \label{tab:kinfit}
\end{table}

\section{Simplified Template Cross Section Bins}

The measurement performed in this analysis is a differential
cross section of Higgs production.
This is done with a Simplified Template Cross Section (STXS) measurement \cite{Kato:2687920},
where the Higgs boson production cross sections are measured in multiple kinematic bins.
This allowed for sensitivity to new physics with reduced model dependence.
For this measurement, the vector boson produced as well as the $p_T$ of the vector boson
separates data points into different bins.
The clean signal of the $Z\rightarrow\ell\ell$ decay channel allows for more
bins to be measured for $ZH$ processes.
In addition, the middle $p_T$ bin for $Z$ boson production is split by multiplicity
of additional jets.
There are eight bins overall:
\begin{itemize}
\item $WH, \SI{150}{GeV} < p_{T,V} \le \SI{250}{GeV}$
\item $WH, \SI{250}{GeV} < p_{T,V} \le \SI{400}{GeV}$
\item $WH, \SI{400}{GeV} < p_{T,V}$
\item $ZH, \SI{75}{GeV} < p_{T,V} \le \SI{150}{GeV}$
\item $ZH, \SI{150}{GeV} < p_{T,V} \le \SI{250}{GeV}, n_\mathrm{jet} = 0$
\item $ZH, \SI{150}{GeV} < p_{T,V} \le \SI{250}{GeV}, n_\mathrm{jet} \ge 1$
\item $ZH, \SI{250}{GeV} < p_{T,V} \le \SI{400}{GeV}$
\item $ZH, \SI{400}{GeV} < p_{T,V}$
\end{itemize}
All of the selections in the following sections are also divided into the appropriate
set of STXS bins for the generation of datacards and fits.
Since the fat jets are most helpful in events where
intermediate particles are highly boosted,
they are only considered in selections for the bins where $p_{T,V} > \SI{250}{GeV}$.

\section{Treatment of Background Yields}

In order to effectively measure Higgs production,
we need to be able to accurately estimate other events
that end up in our selection.
To start, the most significant contaminant physics processes with a final state
identical to the $V\!Hb\bar{b}$ process are identified.
Then, selections that are enriched with these processes, called control regions, are created,
which allows more direct comparison between simulation of these processes and collected data.
When fitting for the signal strength, the background process yields are simultaneously fit
from these control regions.

\subsection{Indentification of Background Processes}

The final state for the two lepton decay in Figure~\ref{fig:two-lep-diagram}
contains two oppositely-charged, same flavor leptons and two $b$-tagged jets.
It can also be achieved by a Drell-Yan process radiating $b$ jets
or a $t\bar{t}$ event where both $W$ bosons from the top decays decay leptonically.
Feynman diagrams in Figure~\ref{fig:dy-2lep} and Figure~\ref{fig:tt-2lep}
show how the two respective processes can result in
the same final state as the signal process.
The Drell-Yan process can also radiate jets initiated by lighter flavor quarks
that are mistakenly identified as $b$-jets,
and those make up a significant portion of the backgrounds as well.
Less significant, but still important backgrounds include processes like
di-boson production, QCD jets, and signal top processes.
\begin{figure}
  \centering
  \begin{fmffile}{dy_2lep}
    \fmfframe(0,0)(0, 20){
    \begin{fmfgraph*}(250, 150)
      \fmfleft{i0,i1}
      \fmfright{o2,o3,o0,o1}
      \fmf{quark}{i1,v0,v1,i0}
      \fmflabel{$\bar{f}$}{i0}
      \fmflabel{$f$}{i1}
      \fmf{boson, label=$Z$}{v0,v2}
      \fmf{fermion}{o0,v2,o1}
      \fmflabel{$\ell^+$}{o0}
      \fmflabel{$\ell^-$}{o1}
      \fmf{gluon, label=$g$}{v1,v3}
      \fmf{fermion}{o2,v3,o3}
      \fmflabel{$\bar{b}$}{o2}
      \fmflabel{$b$}{o3}
    \end{fmfgraph*}
    }
  \end{fmffile}
  \caption[Feynman diagram for DY + jets background]{
    Above is the Feynman diagram matching the two lepton final state coming from
    Drell-Yan and jets.
  }
  \label{fig:dy-2lep}
\end{figure}
\begin{figure}
  \centering
  \begin{fmffile}{tt_2lep}
    \fmfframe(0,0)(0, 20){
    \begin{fmfgraph*}(250, 150)
      \fmfleft{i0,i1}
      \fmfright{o0,o1,o2,o3,o4,o5}
      \fmf{quark}{i1,v0,i0}
      \fmflabel{$\bar{f}$}{i0}
      \fmflabel{$f$}{i1}
      \fmf{gluon, label=$g$}{v0,v1}
      \fmf{fermion, label=$\bar{t}$}{v2,v1}
      \fmf{fermion, label=$t$}{v1,v3}
      \fmf{fermion}{v2,o0}
      \fmflabel{$\bar{b}$}{o0}
      \fmf{boson, label=$W^-$}{v2,v4}
      \fmf{fermion}{o1,v4,o2}
      \fmflabel{$\bar{\nu}$}{o1}
      \fmflabel{$\ell^-$}{o2}
      \fmf{fermion}{v3,o5}
      \fmflabel{$b$}{o5}
      \fmf{boson, label=$W^+$}{v3,v5}
      \fmf{fermion}{o3,v5,o4}
      \fmflabel{$\ell^+$}{o3}
      \fmflabel{$\nu$}{o4}
    \end{fmfgraph*}
    }
  \end{fmffile}
  \caption[Feynman diagram for $t\bar{t}$ background]{
    Above is the Feynman diagram matching the two lepton final state coming from
    fully leptonic $t\bar{t}$ decay.
    In events with little energy carried away by the neutrinos,
    this can appear to be the same as the two-lepton signal process.
  }
  \label{fig:tt-2lep}
\end{figure}

The backgrounds for the one- and zero-lepton signal decay channels are
caused by similar processes.
For the one-lepton decays of $WH$, the Drell-Yan background in Figure~\ref{fig:dy-2lep}
is replaced with a flavor changing current of $W$ + jets.
The $t\bar{t}$ background would instead be caused by either a hadronic decay of
one of the $W$ bosons in Figure~\ref{fig:tt-2lep}
or by one of the pictured leptons travelling out of the detector without being observed.
For the zero-lepton channel, the Drell Yan process is instead replaced with
$Z \rightarrow \nu\bar{\nu}$.
The $t\bar{t}$ process still needs high MET in order to appear to
contain a hard neutrino presence,
so it is most often caused when a single $W$ decays leptonically with the lepton
falling outside of the detector acceptance.
For both of these channels, di-boson, QCD, and single top backgrounds can also contribute,
but not signficantly enough to require a tight constraint on their yields.

\subsection{Goals for Control Regions}

To accurately estimate the contribution of each of these backgrounds,
control regions are used.
These are selections that are in similar phase spaces as the signal selection,
but are instead enriched with background events.
By comparing the prediction from Monte Carlo to data,
scaling corrections to the simulation can be made for the phase space.

For each channel, three control regions are defined.
There are two control regions for the appropriate vector boson radiating jets.
The region is enriched with light flavor jets, labelled $udsg$ for
up, down, and strange quarks, and gluons, by requiring a di-jet system that fails
the $b$-tagging working points.
The heavy flavor $V$ + jets control regions are differentiated from the signal selection
primarily by requiring the di-jet mass to be different than Higgs mass.
The $t\bar{t}$ process tends to radiate more jets than the signal process,
but the lack of a di-lepton resonance is taken advantage of in the two-lepton regions.
Specific cuts for each signal region and the control regions are given in the next section.

\section{Resolved Analysis Selection}

First, the selection for events with two $b$-tagged jets,
also known as resolved events, will be given.
The next section will explain the selections relying on a single fat jet.
With objects defined, the selections differ mostly in counting the number
of charged leptons present in the event.
However, other adjustments are made per channel to optimize the presence of signal events.
Therefore, the first channels described will have the most thorough selection description,
with later channel sections noting many similarities and differences.
For each channel,
multiple control region selections are also used in order to more accurately
estimate the contribution of each physics process to the events in each phase space,
and these will be described after each channel's signal region.
The most significant contributions to background contamination of the signal region are from
$Z/W$ + jets (depending on the number of leptons in the selection) and $t\bar{t}$ processes.
The vector boson plus jets regions are separated into heavy-flavor and light-flavor jets
to control the relative contributions independently since they are not well-known.
For each channel, there are therefore three control regions in addition to the signal region.

A summary of cuts for each region in each channel
is given in Table~\ref{tab:resolved}.
A few channel- or region-specific cuts are left out,
but are described in the appropriate sections.
\begin{table}
  \centering
  \caption[Summary of resolved selection cuts]{
    Below is a summary of common cuts for all regions in the resolved channels.
    See the text for each channel for an explanation of variables.
    All energy equivalent values are in GeV.
  }
  \begin{tabular}{|l|c|c|c|c|c|c|c|c|}
    \hline
    \multicolumn{9}{|l|}{0-lepton channel} \\
    \hline
    Region & $p_{T,V}$ & $p_{T,j}$ & max $b$ & min $b$ & $p_{T,jj}$ & $m_{jj}$ & $N_\textrm{aj}$ & $\Delta\phi(jj, V)$ \\
    \hline
    Signal & 170 & 60, 35 & med. & loose & 120 & >90, <150 & $\le 1$ & $> 2.0$ \\
    Z + $b$ & 170 & 60, 35 & med. & loose & 120 & <90 or >150 & $\le 1$ & $> 2.0$ \\
    Z + $udsg$ & 170 & 60, 35 & !med. & loose & 120 & >50, <500 & $\le 1$ & $> 2.0$ \\
    $t\bar{t}$ & 170 & 60, 35 & med. & loose & 120 & >50, <500 & $\ge 2$ & -- \\
    \hline
    \hline
    \multicolumn{9}{|l|}{1-lepton channel} \\
    \hline
    Region & $p_{T,V}$ & $p_{T,j}$ & max $b$ & min $b$ & $p_{T,jj}$ & $m_{jj}$ & $N_\textrm{aj}$ & $\Delta\phi(jj, V)$ \\
    \hline
    Signal & 150 & 25 & med. & loose & 100 & >90, <150 & $\le 1$ & $> 2.5$ \\
    W + $b$ & 150 & 25 & med. & loose & 100 & <90 or >150 & $\le 1$ & $> 2.5$ \\
    W + $udsg$ & 150 & 25 & !med. & loose & 100 & >50, <250 & -- & $> 2.5$ \\
    $t\bar{t}$ & 150 & 25 & med. & loose & 100 & >50, <250 & $\ge 2$ & -- \\
    \hline
    \hline
    \multicolumn{9}{|l|}{2-lepton channel} \\
    \hline
    Region & $p_{T,V}$ & $p_{T,j}$ & max $b$ & min $b$ & $p_{T,jj}$ & $m_{jj}$ & $N_\textrm{aj}$ & $\Delta\phi(jj, V)$ \\
    \hline
    Signal & 50 & 20 & med. & loose & 50 & >90, <150 & -- & $> 2.5$ \\
    Z + $b$ & 50 & 20 & med. & loose & 50 & <90 or >150 & -- & $> 2.5$ \\
    Z + $udsg$ & 50 & 20 & !loose & !loose & 50 & >50, <250 & -- & $> 2.5$ \\
    $t\bar{t}$ & 50 & 20 & tight & loose & 50 & >50, <250 & -- & -- \\
    \hline
  \end{tabular}
  \label{tab:resolved}
\end{table}
The common variables that are cut on are the following:
\begin{itemize}
\item $p_{T,V}$ -- The transverse momentum of the reconstructed vector boson
\item $p_{T,j}$ -- The minimum transverse momentum of both jets,
  which may be an asymmetric cut
\item max $b$ -- The working point that the jet with the higher $b$-tag value must satisfy
\item min $b$ -- The working point that the jet with the lower $b$-tag value must satisfy
\item $p_{T,jj}$ -- The summed transverse momentum of the di-jet system
\item $m_{jj}$ -- The mass of the di-jet system
\item $N_\textrm{aj}$ -- The number of additional jets outside of the selected di-jet system
\item $\Delta\phi(jj, V)$ -- The distance in angle $\phi$ between the di-jet system
  and the recoiling reconstructed vector boson
\end{itemize}

\subsection{0 Leptons} \label{sec:resolved-0}

In the 0-lepton channel, the transverse momentum carried away by the neutrinos
in the $Z$ boson decay results in a large amount of MET.
CMS has triggers that identify events with large values of MET.
Slightly different triggers are used for each year of Run 2 of the LHC,
and they are listed in Table~\ref{tab:0-triggers}.
\begin{table}
  \centering
  \caption[Triggers for the 0 lepton selection]{
    Below are the trigger paths used for the 0 lepton selections
    for all three years of Run 2 of the LHC.
  }
  \begin{tabular}{|l|l|}
    \hline
    Year & Trigger path(s) \\
    \hline
    2016 & HLT\_PFMET110\_PFMHT110\_IDTight \\
    & HLT\_PFMET120\_PFMHT120\_IDTight \\
    & HLT\_PFMET170\_NoiseCleaned \\
    & HLT\_PFMET170\_BeamHaloCleaned \\
    & HLT\_PFMET170\_HBHECleaned \\
    \hline
    2017 & HLT\_PFMET120\_PFMHT120\_IDTight \\
    & HLT\_PFMET120\_PFMHT120\_IDTight\_PFHT60 \\
    \hline
    2018 & HLT\_PFMET120\_PFMHT120\_IDTight \\
    \hline
  \end{tabular}
  \label{tab:0-triggers}
\end{table}
The MET for the event must be larger than \SI{170}{GeV}.
The trigger efficiency does not quite plateau at that point,
as can be seen in Figure~\ref{fig:0-lep-trigger-eff}.
\begin{figure}
  \centering
  \includegraphics[width=1.0\linewidth]{figures/CompData3Years.pdf}
  \caption[Efficiency for MET trigger in data]{
    The efficiency for the MET triggers are shown for all three years in data.
  }
  \label{fig:0-lep-trigger-eff}
\end{figure}
However, the extracted efficiency scale factor for MC matches
for both $t\bar{t}$ and $W$ + jets simulation at and above this value.
During the 2018 run, a number of HCAL endcap modules were taken offline
due to power supply problems.
These modules were all located in the region $-1.57 < \phi < -0.87$,
so an excess number of high MET events with $\phi_\mathrm{MET}$ in that region were recorded
in later 2018 runs when jets would have been registered in the deactivated detector elements.
The resulting peak can be seen in Figure~\ref{fig:met-peak}.
\begin{figure}
  \centering
  \includegraphics[width=0.8\linewidth]{figures/METPhi319077.pdf}
  \caption[MET $\phi$ distribution before and after shutting off HCAL modules]{
    Above compares the MET $\phi$ distributions before and after shutting off
    problematic HCAL modules during for run 319077.
    The excess of events in the region $-1.86 < \phi_\mathrm{MET} < -0.7$
    is caused by mis-measuring the momenta of forward jets in that region.
  }
  \label{fig:met-peak}
\end{figure}
To handle this, all events with $-1.86 < \phi_\mathrm{MET} < -0.7$ that occurred
during and after run 319077, when the faulty detector elements were shut off.

In addition to the neutrino decay of the $Z$ boson, the $b\bar{b}$ decay
of the Higgs also needs to be selected and backgrounds need to be removed.
Many of the following cuts are similar for all of the channels.
For the Higgs decay, two jets are selected.
One jet is the one with the highest $b$-tag score out of jets with $p_T > \SI{60}{GeV}$.
The other jet is the highest $b$-tag score out of jets remaining with $p_T > \SI{35}{GeV}$.
This distinction is important, because these two jets may not have the highest
$b$-tag score of the event if the two highest $b$-tag scored jets both have
$p_T < \SI{60}{GeV}$, for example.
However, we are not interested in such relatively low $p_T$ systems since
the di-jet system should be reasonably balanced against the MET requirement.
To additionally enforce this balance, the di-jet system must satisfy $p_T > \SI{120}{GeV}$.
Also the di-jet system is selected to be back-to-back with the MET
by requiring $\Delta\phi(\mathrm{MET}, jj) > 2.0$.
The di-jet mass is required to be less than \SI{500}{GeV} to remove high energy
combinatoric backgrounds.
To increase purity of $Z\nu\nu$ processes,
events are not considered if there are any isolated leptons
with $|\eta| < 2.5$ and $p_T > \SI{15}{GeV}$.
Finally, to reduce QCD background contributions,
all jets in the event with $p_T > \SI{30}{GeV}$
must be a minimum distance from the event MET satisfying
$\Delta \phi(\mathrm{MET}, j) > 0.5$.

The signal region it is further required that
the selected $b$-jets to be of high quality,
the di-jet system has a mass close to the Higgs,
and low additional hadronic activity.
At least one of the $b$-jets must pass the medium working point,
and the other $b$-jet must pass the loose working point.
The value of the di-jet mass must satisfy $\SI{90}{GeV} < m_{jj} < \SI{150}{GeV}$.
At most one jet with $p_T > \SI{30}{GeV}$ is allowed
in addition to the selected $b$-jets.
Since the event has no high energy leptons,
the particle flow MET and track MET must have good agreement with
$\Delta \phi(\mathrm{MET}, \mathrm{trkMET}) < 0.5$.

As mentioned earlier, three processes make up the majority of the backgrounds
in the signal selection, and control regions must be defined for them.
Two of them when a $Z \rightarrow \nu\bar{\nu}$ occurs recoiling off of jets.
In one processes, the recoiling jets are $b$-jets, and in the other, the jets are light jets.
Since the fraction of actual $Z \rightarrow \nu\bar{\nu}$ events
containing $b$-jets is not well known, they are scaled separately.
The third background process that needs to be separately measured in this phase space
is a $t\bar{t}$ semi-leptonic decay
where the lepton falls outside of the detector acceptance.
This type of background also results in a final state with high MET and $b$-jets.

Of the three processes,
the $Z$ + heavy flavor jets selection is the most similar to the signal selection.
The only difference is the di-jet mass.
Events with a mass between $\SI{50}{GeV} < m_{jj} < \SI{500}{GeV}$ but not between
$\SI{90}{GeV} < m_{jj} < \SI{150}{GeV}$, which is the signal region window,
are selected to quantify the $Z$ + heavy flavor.
The $Z$ + light flavor selection includes the entire mass range,
without a veto for the Higgs mass.
It is instead enriched with light jets by requiring that the selected $b$-jet with
a higher $b$-tag score fails the medium working point.
The selection for $t\bar{t}$ events is different from the signal selection
mostly because more hadronic activity is expected.
It uses the same full di-jet mass window as the $Z$ + light flavor region,
but requires at least two additional jets with $p_T > \SI{30}{GeV}$ instead of zero or one.
Also, the $\Delta \phi(\mathrm{MET}, \mathrm{trkMET})$ requirement is dropped.

\subsection{1 Lepton} \label{sec:resolved-1}

In the 1-lepton channel, a single fully-selected isolated lepton as defined in
Section~\ref{sec:muon-def} or Section~\ref{sec:ele-def} is required.
CMS has triggers for isolated leptons,
and the ones used in the analysis are listed in Table~\ref{tab:1-triggers}.
\begin{table}
  \centering
  \caption[Triggers for the 1 lepton selections]{
    Below are the trigger paths used for the 1 lepton selections
    for all three years of Run 2 of the LHC.
    Different triggers are used for the electron and muon channel,
    and are labelled separately.
  }
  \begin{tabular}{|l|c|l|}
    \hline
    Year & Lepton & Trigger path(s) \\
    \hline
    2016 & $e$ & HLT\_Ele27\_WPTight\_Gsf \\
    \hline
    2016 & $\mu$ & HLT\_IsoMu24 \\
    & & HLT\_IsoTkMu24 \\
    \hline
    2017 & $e$ & HLT\_Ele32\_WPTight\_Gsf\_L1DoubleEG \\
    \hline
    2017 & $\mu$ & HLT\_IsoMu27 \\
    \hline
    2018 & $e$ & HLT\_Ele32\_WPTight\_Gsf \\
    \hline
    2018 & $\mu$ & HLT\_IsoMu24 \\
    \hline
  \end{tabular}
  \label{tab:1-triggers}
\end{table}
That lepton must point in a similar direction of the MET,
satisfying $\Delta \phi(\ell, \mathrm{MET}) < 2.0$.
The $p_T$ of the reconstructed $W$ boson, consisting of the vector sum of MET and the lepton,
must satisfy $p_T > \SI{150}{GeV}$.
If there are any additional leptons, the event is not used.
The presence of an isolated lepton provides
a much cleaner signal than in the zero lepton channel.
Therefore, the kinematic cuts on the selected $b$-jets can be looser.
The $b$-jets only need to satisfy $p_T > \SI{25}{GeV}$,
and the di-jet system only needs $p_T > \SI{100}{GeV}$.
Any events with $m_{jj} \ge \SI{250}{GeV}$ are not considered for any regions.
The $b$-tagging requirement is the same as the 0-lepton channel.
There is a slightly tighter cut on the di-jet direction of
$\Delta\phi(\mathrm{MET}, jj) > 2.5$.

In the signal region, the additional cuts are again similar to the 0-lepton channel.
The di-jet mass window is the same, as is the requirement of at most one additional jet.
The only other difference from the 0-lepton selection
aside from the adjusted common cuts listed above
is that there is no dependence on the track MET direction.
The change to create the $W$ + heavy flavor jets control region is the exact same as the
$Z$ + heavy flavor region in the 0-lepton channel.
The mass window for the Higgs is vetoed.
The cut for the $W$ + light flavor control region is also the same in terms of $b$-tagging,
but the additional jet requirement is also removed.
The $t\bar{t}$ control region is also the same in that
the only changes from the signal region
are a relaxed mass window, the requirement of at least two additional jets,
and no requirement on the di-jet direction relative to MET.

\subsection{2 Leptons} \label{sec:resolved-2}

For the 2-lepton channel, two oppositely charged leptons with the same flavor are required.
The triggers used that are designed to match this requirement are given
in Table~\ref{tab:2-triggers}.
\begin{table}
  \centering
  \caption[Triggers for the 2 lepton selections]{
    Below are the trigger paths used for the 2 lepton selections
    for all three years of Run 2 of the LHC.
    Different triggers are used for the electron and muon channel,
    and are labelled separately.
  }
  \begin{tabular}{|l|c|l|}
    \hline
    Year & Leptons & Trigger path(s) \\
    \hline
    2016 & $e^+e^-$ & HLT\_Ele23\_Ele12\_CaloIdL\_TrackIdL\_IsoVL\_DZ \\
    \hline
    2016 & $\mu^+\mu^-$ & HLT\_Mu17\_TrkIsoVVL\_Mu8\_TrkIsoVVL \\
    & & HLT\_Mu17\_TrkIsoVVL\_TkMu8\_TrkIsoVVL \\
    & & HLT\_Mu17\_TrkIsoVVL\_Mu8\_TrkIsoVVL\_DZ \\
    & & HLT\_Mu17\_TrkIsoVVL\_TkMu8\_TrkIsoVVL\_DZ \\
    \hline
    2017 & $e^+e^-$ & HLT\_Ele23\_Ele12\_CaloIdL\_TrackIdL\_IsoVL\_DZ \\
    \hline
    2017 & $\mu^+\mu^-$ & HLT\_Mu17\_TrkIsoVVL\_Mu8\_TrkIsoVVL\_DZ\_Mass3p8 \\
    & & HLT\_Mu17\_TrkIsoVVL\_TkMu8\_TrkIsoVVL\_DZ\_Mass8 \\
    \hline
    2018 & $e^+e^-$ & HLT\_Ele23\_Ele12\_CaloIdL\_TrackIdL\_IsoVL\_DZ \\
    \hline
    2018 & $\mu^+\mu^-$ & HLT\_Mu17\_TrkIsoVVL\_Mu8\_TrkIsoVVL\_DZ\_Mass3p8 \\
    \hline
  \end{tabular}
  \label{tab:2-triggers}
\end{table}
They must have an mass satisfying $75 < m_{\ell\ell} < \SI{105}{GeV}$.
This process is clean enough to relax the kinematic cuts on the selected $b$-jets
even further than the relaxed cuts of the 1-lepton channel.
The selected $b$-jets only need to have a $p_T > \SI{20}{GeV}$,
and the di-jet system only needs $p_T > \SI{50}{GeV}$.
There are no cuts on the number of additional, outside of categorization for STXS bins.
The di-jet system still needs to satisfy the tighter cut of $\Delta\phi(jj,V) > 2.5$.

The signal region uses the same $b$-tag and mass window cuts as the other two channels.
For the $Z$ + heavy flavor control region,
the di-lepton mass cut is narrowed to $85 < m_{\ell\ell} < \SI{97}{GeV}$
in order to cut out $t\bar{t}$, and the usual Higgs mass veto is applied.
The MET is also required to be low with $\mathrm{MET} < \SI{60}{GeV}$
for the $Z$ + heavy region, but no other.
For the $Z$ + light flavor region, purity is achieved by requiring
that both selected jets fail the loose working point for $b$-tagging.
The $t\bar{t}$ region is then selected by requiring
one selected jet to pass the tight working point.
The di-lepton mass value also must be either $10 < m_{\ell\ell} < \SI{75}{GeV}$
or $m_{\ell\ell} > \SI{120}{GeV}$.

\section{Boosted Analysis Selection}

When the Higgs has an adequately high $p_T$,
the jet clustering algorithms will often find both daughter particles
as being part of a single jet.
The boosted analysis only targets the STXS bins with $p_{T,V} > \SI{250}{GeV}$.
The selection differs primarily in the fact that a single fat jet
which passes a double $b$-tag cut \cite{Sirunyan_2018}
is used to reconstruct the potential Higgs instead of two $b$-tagged jets.
The double $b$-tagger used in this analysis is DeepAK8,
a DNN as opposed to a BDT tagger.
The output is decorrelated with mass, allowing for the jet mass to be used
to generate control regions, as is done in the resolved analysis.

Additional $b$ jets outside of the fat jet are also counted to define selections.
These come from the regular jet collections of Section~\ref{sec:jets-def}.
In order to count as an additional $b$-jet,
the jet must pass the DeepCSV medium working point,
have a $p_T > \SI{25}{GeV}$, and be outside of the selected fat jet
with $\Delta R(j, f\!j) > 0.8$.

\subsection{0 Leptons}

The boosted selection regions are designed to study the same processes
as the resolved selection regions.
The 0-lepton channel is enriched with signal events where
the $Z$ boson decays to neutrinos, so the 0-lepton channel has high MET.
A requirement of $\mathrm{MET} > \SI{250}{GeV}$ is applied,
balancing out the $p_T$ requirement of the fat jet.
As for the resolved analysis, the presence of any leptons leads to
the event not being considered for the 0-lepton channel.
To remove QCD background for all regions, the same cut from the resolved analysis of
$\Delta \phi(\mathrm{MET}, j) > 0.5$ for all jets with $p_T > \SI{30}{GeV}$ is used.

In the signal region, jets must have a score of 0.8 or higher in the
bbVsLight output node of the DeepAK8 tagger.
They must also have a soft drop mass in the range of $90 < m_\mathrm{SD} < \SI{150}{GeV}$.
No additional jets outside of the fatjet are allowed in the event.
The control regions are the same as for the resovled analysis:
$Z$ + heavy flavor, $Z$ + light flavor, and $t\bar{t}$.
For the $Z$ + heavy flavor control region,
the mass cut is changed to instead veto the Higgs mass window.
For the $Z$ + light flavor, there is no mass requirement outside of the
$m_\textrm{SD} > \SI{50}{GeV}$ required for all fat jets.
Orthogonality is enforced by requiring the bbVsLight score to be less than 0.8.
For $t\bar{t}$, the lack of mass requirement is also present,
but there must be at least one $b$ jet outside of the fat jet.

\subsection{1 Lepton}

For the single lepton channel, exactly one selected lepton must be present.
It must also point in the same direction as the MET with $\Delta \phi(MET, \ell) < 2.0$.
Otherwise, the selection criteria for the different 1-lepton regions
are the exact same as for the boosted 0-lepton regions.

\subsection{2 Leptons}

For the two lepton channel, two oppositely charged, same flavor leptons must be present,
as described in Section~\ref{sec:resolved-2}.
These leptons must also have a mass near the $Z$ boson mass for the signal region
and the two $Z$ + jets regions.
The selection for the signal and control regions are otherwise similar to the selections
for the 0- and 1-lepton channels in the boosted analysis.
The only difference is that instead of requiring a $b$ jet outside of the fat jet
for the $t\bar{t}$ control region,
a mass veto of the di-lepton mass is applied, just as was done for the resolved analysis.

\section{Overlap in Resolved and Boosted Selections}

An important note is that each signal and control region described in this chapter
must be orthogonal to all other regions.
To prevent any statistical bias in the analysis,
both simulated and measured events must not be double counted.
It is easy to make orthagonal selections within resolved or boosted categories
by counting the number of leptons, comparing their flavor, and reversing other selections
to go from signal to control regions.
However, when making selections for the resolved and boosted channels,
enforcing orthogonality is not as straightforward.
The same PF candidates are reused to define two different kinds of jet collections,
making it harder to reverse a single selection cut to define a different category.
The selections described so far result in some events simultaneously passing
both boosted and resolved selections.
To maximize the expected sensitivity of this analysis,
most events that are in both resolved and boosted categories are assigned to the
resolved category and omitted from the boosted.
The only exception to this rule is when the event is assigned to
one of the resolved control region and a boosted signal region.
In that case, the event is assigned to its boosted category.
