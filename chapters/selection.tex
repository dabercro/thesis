\chapter{Event Selection} \label{ch:selection}

Since Chapter~\ref{ch:theory},
a picture of the overall event topology of interest has been depicted.
Until now, the description of the events have been vague.
This chapter gives the specific cuts used to identify each particle
and to reject or otherwise classify events.
First, objects will be defined in terms of reconstructed variables.
Then common cuts used to reject events are given.
After that, cuts used to classify events into different decay channels are specified.
This chapter ends with the differentiation between events where
the Higgs decay products can be resolved as separate jets
and where they are merged into a single massive jet.

\section{Object Definitions}

Section~\ref{sec:event-reco} describes how
detector responses are linked to possible physical particles.
Most of the particle ID techniques described so far also give false positives.
What follows are tighter selections used in order to reduce these backgrounds.
Once objects are defined, they can be counted for event classification.

Each type of particle generally has a method of pre-selection
and additional cuts for a selection.
For these objects, pre-selection often means that the object is defined
well enough to veto events in categories that don't want the object to exist.
The additional cuts are required for an event to classify as a category
that wants the object to exist with reduced false positives.

\subsection{Variable Definitions}

Many of the object definitions use variables that have not been defined yet.
They can be understood in terms of the reconstruction
described in Section~\ref{sec:event-reco}.

Lepton isolation is quantified using the following formula.
\begin{gather}
  I = \frac{1}{p_T^\ell} \left(\sum p_T^\mathrm{charged} +
      \max\left[0, \sum p_T^\mathrm{neutral} +
               \sum p_T^\mathrm{\gamma} - p_T^\mathrm{PU}
               \right]\right) \label{eq:isolation}
\end{gather}
The sums are over charged hadrons originating from the primary vertex
and all neutral hadrons and photons within a distance of $\Delta R < 0.4$
from the lepton if it is a muon, or $\Delta R < 0.3$ from an electron.
$\Delta R$ is a distance used in the $(\eta, \phi)$ plane.
\begin{gather}
  \Delta R = \sqrt{\Delta\eta^2 + \Delta\phi^2}
\end{gather}
The term $p_T^\mathrm{PU}$ is defined as the following for muons.
\begin{gather}
  p_T^\mathrm{PU} = 0.5 \times \sum_i p_T^{\mathrm{PU},i}
\end{gather}
$i$ refers to charged hadrons that do not originate from the primary vertex.
Electrons use the following definition.
\begin{gather}
  p_T^\mathrm{PU} = \rho \times A_\mathrm{eff}
\end{gather}
$A_\mathrm{eff}$ is the area of the isolation cone,
and $\rho$ is the median of the $p_T$ density of neutral particles in that area.

Particles can also be defined as coming from the primary vertex of an event or from pileup.
Vertices are defined through deterministic annealing \cite{726788},
using the closest approach of tracks to the beamline \cite{Collaboration_2014}.
The primary vertex is the vertex with
the greatest sum of $E_T$ of the charged particles originating from it.
After idenfication of the primary vertex,
charged particles are classified as originating from the primary vertex or as pileup
using their extrapolated track's distance in the transverse plane, $d_{xy}$
and distance along the beamline, $d_z$.

\subsection{Muons}

An isolated muon gives one of the cleanest signatures in CMS,
with only perhaps the exception of an isolated photon that does not
undergo pair production in the pixel tracker.
Muons can also show up in jets from weakly decaying hadrons,
in which case they are not isolated.
Since weakly decaying $b$ jets are central to this analysis,
events with non-isolated leptons are not rejected,
but the distinction is important.

Pre-selected muons must meet the following requirements.
\begin{itemize}
\item The muon must have a relatively high energy of $p_T > \SI{5}{GeV}$.
\item The muon should be located in the barrel within $|\eta| < 2.4$.
\item The muon originates from the primary vertex, satisfying both
  $d_{xy} < \SI{0.5}{cm}$ and $d_z < \SI{1.0}{cm}$.
\item The muon must pass a loose isolation cut of $I < 0.4$.
\item The muon must be a PF muon.
\item The muon is either a global muon or a tracker muon.
\end{itemize}

Fully selected muons have some additional cuts they must pass.
\begin{itemize}
\item They must have a higher transverse momentum at $p_T > \SI{25}{GeV}$.
  In events with two muons, such as caused by $Z \rightarrow \mu\mu$,
  the second muon can pass the slightly looser cut of $p_T > \SI{15}{GeV}$.
\item The muon must be a global muon, leaving tracks in both the central tracker
  and the muon chambers.
\item There must be more than five hits in the inner tracker with one hit on a pixel.
\item The fit for the global muon track must be good with $\chi^2/ndof < 10$.
\item The muon must be well isolated with $I < 0.06$
\end{itemize}

These definitions are accepted by all members of the CMS collaberation
as loose and tight working points, respectively.
This allows the analysis to use efficiency measurements created for wider use.

\subsection{Electrons}

The kinematic variables associated with an electron are extracted from the GSF fit.
Pre-selected electrons must meet the following requirements.
\begin{itemize}
\item They must have a high transverse momentum with $p_T > \SI{7}{GeV}$
\item They should be centered in the detector with $|\eta| < 2.4$.
\item They originate from the primary vertex with
  $d_{xy} < \SI{0.05}{cm}$ and $d_Z < \SI{0.2}{cm}$.
\item They pass a loose isolation cut of $I < 0.4$.
\end{itemize}

To reduce backgrounds, electrons are identified with the aid of an MVA \cite{Rembser_2019}.
Fully selected electrons pass the tight working point used by the CMS collaboration.
In order to also match the samples of simulated electrons used in the training sample,
the selected electrons also must pass the following cuts.
\begin{itemize}
\item The electron must have higher energy with $p_T > \SI{15}{GeV}$.
\item The deposit of HCAL energy must be less than 9\% of the ECAL energy deposit
  along the electron track.
\item The track sum $p_T$ component of the isolation must be
  less than 18\% of the electron $p_T$.
\item The electron must either have $|\eta| < 1.4442$ or $|\eta| > 1.5660$
\item For electrons with $|\eta| < 1.4442$:
  \begin{itemize}
  \item $\sigma_{i\eta i\eta} < 0.012$
  \item Isolation in the ECAL cluster must be less than 0.4,
    and isolation in the HCAL must be less than 0.25.
  \item The difference between super cluster and track location of the electron
    must be small with $\Delta \eta < 0.0095$ and $\Delta \phi < 0.065$.
  \end{itemize}
\item For electrons with $|\eta| > 1.5660$:
  \begin{itemize}
  \item $\sigma_{i\eta i\eta} < 0.033$
  \item Isolation in the ECAL cluster must be less than 0.45,
    and isolation in the HCAL must be less than 0.28.
  \end{itemize}
\end{itemize}
There is a gap in the $\eta$ direction that accounts for a gap in the CMS detector.
This gap in the active detecting volume is needed to
accomadate various electronics and structural components.

\subsection{Jets}

As described in Section~\ref{sec:lhc},
hadrons produced at the LHC are often accompanied by sprays of particles called jets.
Conservation of energy and momentum means that the sum of jet constituents
give the kinematics of the initial parton that produced them.
Jets are constructed by clustering all particle-flow candidates
with the anti-$k_T$ algorithm \cite{Cacciari_2008} using $R = 0.4$.
Due to factors like pileup and detector response,
the energy of the reconstructed jets are corrected \cite{Khachatryan_2017}.

Loose jet cuts, based on the constituents,
are applied to remove jets constructed from detector noise.
Jets that get a significant fraction of their energy from pileup are also removed.
Preselected electrons and muons are also often reconstructed as jets.
Any jet within $\Delta R < 0.4$ from a preselected lepton is removed.

To be considered for the study of decay products of the Higgs boson,
jets must be within the barrel of the detector with $|\eta| < 2.5$.
The jets must satisfy $p_T > \SI{25}{GeV}$ for the zero and one lepton signatures.
The two lepton signature from $Z(\ell\ell)H$ is cleaner and can apply the looser
criteria of $p_T > \SI{20}{GeV}$ to the jets.

\subsubsection{Identification of $b$ Jets}

As mentioned in Section~\ref{sec:requirements} while motivating the design of CMS,
jets containing $b$ hadrons have a distinct signature.
This includes secondary vertices displaced from the beamline,
as well as non-isolated leptons from weak decays.
All of these features are considered in a deep neural network (DNN)
called Deep Combined Secondary Vertex (DeepCSV) \cite{Sirunyan_2018}.
The output of DeepCSV has three working points
that are defined based on the amount of false positives that can be expected
in a collection of jets passing the cut.
The specific values are different for each year of operation.
They are given in Table~\ref{tab:deepcsv}.
\begin{table}
  \centering
  \caption[DeepCSV working points]{
    The cuts for each DeepCSV working point are defined for each year of Run~2 of the LHC.
    The working points are defined by their mistag rates.
  }
  {\renewcommand{\arraystretch}{1.5}
  \begin{tabular}{|c | c | c c c|}
    \hline
    Working Point & Mistag Rate & 2016 & 2017 & 2018 \\
    \hline
    Loose  &  10\% & 0.2219 & 0.1522 & 0.1241 \\
    Medium &   1\% & 0.6324 & 0.4941 & 0.4184 \\
    Tight  & 0.1\% & 0.8958 & 0.8001 & 0.7527 \\
    \hline
  \end{tabular}
  }
  \label{tab:deepcsv}
\end{table}

\subsection{Missing Transverse Energy}

Missing transverse energy, also labelled $E_T^\mathrm{miss}$ or MET,
is a vector that takes advantage of the fact that momentum
transverse to the beamline is conserved.
MET is calculated by first taking the negative vector sum of the transverse momentum of
all particle flow candidates in the event.
The resulting vector is then adjusted by taking into account the difference
between the uncorrected and corrected jet energies \cite{collaboration_2015}.
The resulting magnitude and direction is a proxy for the transverse momentum of
any neutrinos in the event.
However, large MET values can be generated by instrumental and beam effects as well.
Therefore, there are additional event filters applied to events with large MET
that removes events where the rest of the detector exhibits suspicious behavior.

\subsection{Soft Hadronic Activity}

In signal $VH$ events, hadronic activity outside of the \bb decay of the Higgs is expected to be low.
This hadronic activity is defined by considering the additional charged PF tracks coming from the primary vertex.
An exclusion region is defined in an ellipse in $(\eta, \phi)$ space containing the two selected $b$ jets
with a major axis length of $\Delta R (\bb) + 1$ and a minor axis length of 1.
All charged tracks outside of this ellipse that also do not correspond with the selected leptons and
that satisfy $p_T > \SI{300}{MeV}$ and $d_Z < \SI{0.2}{cm}$ are clustered
using the anti-$k_T$ algorithm \cite{Cacciari_2008} with $R = 0.4$.
The resulting collection of soft jets is used to define four variables:

\begin{itemize}
\item $H_T^\mathrm{soft}$ -- The scalar sum of soft jets' $p_T$ for jets with $p_T > \SI{1}{GeV}$
\item $N_2^\mathrm{soft}$ -- The number of soft jets with $p_T > \SI{2}{GeV}$
\item $N_5^\mathrm{soft}$ -- The number of soft jets with $p_T > \SI{5}{GeV}$
\item $N_{10}^\mathrm{soft}$ -- The number of soft jets with $p_T > \SI{10}{GeV}$
\end{itemize}

These variables are used in the training of the BDT that discriminates signal and background events.

\subsection{Kinematic Fit}

In the two-lepton region, the all of the $Z$ boson's decay products can be observed and
its momentum can be reconstructed precisely.
A kinematic fit is performed to constrain the leptons' momenta additionally by requiring their
invariant mass to match the $Z$ boson mass.
After this constraint is applied, the tranvserve momentum of the dijet system,
along with ISR jets, is balanced with the transverse momentum of the dilepton system.
The fit is performed by minimizing the chi-squared of the kinematic system that meets these constraints.

For electrons and muons, the corrections defined by their respective Physics Objects Groups
in CMS include uncertainties.
The energy uncertainty used for the $b$-jets and recoils jets is the same as that used
by the $HH \rightarrow \bb\bb$ analysis from CMS \cite{Sirunyan_2019}.
The mass of the $Z$ boson also has a width.
The kinematic fit minimizes the chi-squared value of these constraints,
resulting in new energies for all particles in the fit.
The full implementation is in the \texttt{PhysicsTools/KinFitter} package of
\texttt{CMSSW\_10\_2\_0\_pre3} \cite{cmssw_doxygen}.
Only the resulting energies of the $b$-jets, which are otherwise relatively loosely constrained,
are used for analyzing the two-lepton regions.
The improvement of the di-jet mass in the signal sample as a result of the kinematic fit
can be seen in Figure~\ref{fig:kinfit} and Table~\ref{tab:kinfit}.

\begin{figure}
  \includegraphics[width=\linewidth]{figures/fits_SR_medhigh_Hmass__.pdf}
  \caption[Higgs di-jet mass fit with kinematic fit]{}
  \label{fig:kinfit}
\end{figure}

\begin{table}
  \centering
  \caption[Mass resolutions after kinematic fit]{}
  \begin{tabular}{|r|c|c|c|}
    \hline
    & Low $p_T$ & High $p_T$ & incl. \\
    \hline
    No R. & 0.157 & 0.133 & 0.150 \\
    Reg.  & 0.134 & 0.121 & 0.130 \\
    Fit   & 0.129 & 0.112 & 0.124 \\
    \hline
  \end{tabular}
  \label{tab:kinfit}
\end{table}

\section{Common Cuts}

We have some cuts across the board on our objects
in order to remove events that are just QCD.

\section{Categories of Vector Boson Decay}

Now that we are ready to count,
we can count leptons in order to characterise
potential vector bosons.

\subsection{0 Leptons}

\subsection{1 Lepton}

\subsection{2 Leptons}

\section{Topology of Higgs Decay}

\subsection{Resolved Jets}

We reconstruct two $b$ jets.

\subsection{Boosted Jet}

When the Higgs has very high $p_T$,
the jet clustering algorithms can find both daughter particles
as being part of a single jet.

\section{Control Regions}

\subsection{Light Flavor Jets}

\subsection{Heavy Flavor Jets}

\subsection{$t\bar{t}$}
