\chapter{Event Selection} \label{ch:selection}

Since Chapter~\ref{ch:theory},
a picture of the overall event topology of interest has been depicted.
Until now, the description of the events have been vague.
This chapter gives the specific cuts used to identify each particle
and to reject or otherwise classify events.
First, objects will be defined in terms of reconstructed variables.
Then common cuts used to reject events are given.
After that, cuts used to classify events into different decay channels are specified.
This chapter ends with the differentiation between events where
the Higgs decay products can be resolved as separate jets
and where they are merged into a single massive jet.

\section{Object Definitions}

Section~\ref{sec:event-reco} describes how
detector responses are linked to possible physical particles.
Most of the particle ID techniques described so far also give false positives.
What follows are tighter selections used in order to reduce these backgrounds.
Once objects are defined, they can be counted for event classification.

Each type of particle generally has a method of pre-selection
and additional cuts for a selection.
For these objects, pre-selection often means that the object is defined
well enough to veto events in categories that don't want the object to exist.
The additional cuts are required for an event to classify as a category
that wants the object to exist with reduced false positives.

\subsection{Variable Definitions}

Many of the object definitions use variables that have not been defined yet.
They can be understood in terms of the reconstruction
described in Section~\ref{sec:event-reco}.

Lepton isolation is quantified using the following formula.
\begin{gather}
  I = \frac{1}{p_T^\ell} \left(\sum p_T^\mathrm{charged} +
      \max\left[0, \sum p_T^\mathrm{neutral} +
               \sum p_T^\mathrm{\gamma} - p_T^\mathrm{PU}
               \right]\right) \label{eq:isolation}
\end{gather}
The sums are over charged hadrons originating from the primary vertex
and all neutral hadrons and photons within a distance of $\Delta R < 0.4$
from the lepton if it is a muon, or $\Delta R < 0.3$ from an electron.
$\Delta R$ is a distance used in the $(\eta, \phi)$ plane.
\begin{gather}
  \Delta R = \sqrt{\Delta\eta^2 + \Delta\phi^2}
\end{gather}
The term $p_T^\mathrm{PU}$ is defined as the following for muons.
\begin{gather}
  p_T^\mathrm{PU} = 0.5 \times \sum_i p_T^{\mathrm{PU},i}
\end{gather}
$i$ refers to charged hadrons that do not originate from the primary vertex.
Electrons use the following definition.
\begin{gather}
  p_T^\mathrm{PU} = \rho \times A_\mathrm{eff}
\end{gather}
$A_\mathrm{eff}$ is the area of the isolation cone,
and $\rho$ is the median of the $p_T$ density of neutral particles in that area.

Particles can also be defined as coming from the primary vertex of an event or from pileup.
Vertices are defined through deterministic annealing \cite{726788},
using the closest approach of tracks to the beamline \cite{Collaboration_2014}.
The primary vertex is the vertex with
the greatest sum of $E_T$ of the charged particles originating from it.
After idenfication of the primary vertex,
charged particles are classified as originating from the primary vertex or as pileup
using their extrapolated track's distance in the transverse plane, $d_{xy}$
and distance along the beamline, $d_z$.

\subsection{Muons}

An isolated muon gives one of the cleanest signatures in CMS,
with only perhaps the exception of an isolated photon that does not
undergo pair production in the pixel tracker.
Muons can also show up in jets from weakly decaying hadrons,
in which case they are not isolated.
Since weakly decaying $b$ jets are central to this analysis,
events with non-isolated leptons are not rejected,
but the distinction is important.

Pre-selected muons must meet the following requirements.
\begin{itemize}
\item The muon must have a relatively high energy of $p_T > \SI{5}{GeV}$.
\item The muon should be located in the barrel within $|\eta| < 2.4$.
\item The muon originates from the primary vertex, satisfying both
  $d_{xy} < \SI{0.5}{cm}$ and $d_z < \SI{1.0}{cm}$.
\item The muon must pass a loose isolation cut of $I < 0.4$.
\item The muon must be a PF muon.
\item The muon is either a global muon or a tracker muon.
\end{itemize}

Fully selected muons have some additional cuts they must pass.
\begin{itemize}
\item They must have a higher transverse momentum at $p_T > \SI{25}{GeV}$.
  In events with two muons, such as caused by $Z \rightarrow \mu\mu$,
  the second muon can pass the slightly looser cut of $p_T > \SI{15}{GeV}$.
\item The muon must be a global muon, leaving tracks in both the central tracker
  and the muon chambers.
\item There must be more than five hits in the inner tracker with one hit on a pixel.
\item The fit for the global muon track must be good with $\chi^2/ndof < 10$.
\item The muon must be well isolated with $I < 0.06$
\end{itemize}

These definitions are accepted by all members of the CMS collaberation
as loose and tight working points, respectively.
This allows the analysis to use efficiency measurements created for wider use.

\subsection{Electrons}

The kinematic variables associated with an electron are extracted from the GSF fit.
Pre-selected electrons must meet the following requirements.
\begin{itemize}
\item They must have a high transverse momentum with $p_T > \SI{7}{GeV}$
\item They should be centered in the detector with $|\eta| < 2.4$.
\item They originate from the primary vertex with
  $d_{xy} < \SI{0.05}{cm}$ and $d_Z < \SI{0.2}{cm}$.
\item They pass a loose isolation cut of $I < 0.4$.
\end{itemize}

To reduce backgrounds, electrons are identified with the aid of an MVA \cite{Rembser_2019}.
Fully selected electrons pass the tight working point used by the CMS collaboration.
In order to also match the samples of simulated electrons used in the training sample,
the selected electrons also must pass the following cuts.
\begin{itemize}
\item The electron must have higher energy with $p_T > \SI{15}{GeV}$.
\item The deposit of HCAL energy must be less than 9\% of the ECAL energy deposit
  along the electron track.
\item The track sum $p_T$ component of the isolation must be
  less than 18\% of the electron $p_T$.
\item The electron must either have $|\eta| < 1.4442$ or $|\eta| > 1.5660$
\item For electrons with $|\eta| < 1.4442$:
  \begin{itemize}
  \item $\sigma_{i\eta i\eta} < 0.012$
  \item Isolation in the ECAL cluster must be less than 0.4,
    and isolation in the HCAL must be less than 0.25.
  \item The difference between super cluster and track location of the electron
    must be small with $\Delta \eta < 0.0095$ and $\Delta \phi < 0.065$.
  \end{itemize}
\item For electrons with $|\eta| > 1.5660$:
  \begin{itemize}
  \item $\sigma_{i\eta i\eta} < 0.033$
  \item Isolation in the ECAL cluster must be less than 0.45,
    and isolation in the HCAL must be less than 0.28.
  \end{itemize}
\end{itemize}
There is a gap in the $\eta$ direction that accounts for a gap in the CMS detector.
This gap in the active detecting volume is needed to
accomadate various electronics and structural components.

\subsection{Jets}

As described in Section~\ref{sec:lhc},
hadrons produced at the LHC are often accompanied by sprays of particles called jets.
Conservation of energy and momentum means that the sum of jet constituents
give the kinematics of the initial parton that produced them.
Jets are constructed by clustering all particle-flow candidates
with the anti-$k_T$ algorithm \cite{Cacciari_2008} using $R = 0.4$.
Due to factors like pileup and detector response,
the energy of the reconstructed jets are corrected \cite{Khachatryan_2017}.

\subsubsection{Identification of $b$ Jets}

\subsection{MET}

MET is corrected.

\subsection{Undesirable Particles}

There are certain particles that we do not want present.
We make very loose selections for those and veto on them.

\subsubsection{Photons}

\subsubsection{Tau Leptons}

\section{Removal of QCD}

We have some cuts across the board on our objects
in order to remove events that are just QCD.

\section{Categories of Vector Boson Decay}

Now that we are ready to count,
we can count leptons in order to characterise
potential vector bosons.

\subsection{0 Leptons}

\subsection{1 Lepton}

\subsection{2 Leptons}

\section{Topology of Higgs Decay}

\subsection{Resolved Jets}

We reconstruct two $b$ jets.

\subsection{Boosted Jet}

When the Higgs has very high $p_T$,
the jet clustering algorithms can find both daughter particles
as being part of a single jet.
