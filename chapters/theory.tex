\chapter{Theory} \label{ch:theory}

Before diving into the description of the experimental apparatus,
an explanation of why it is expected to work is needed.
There are many textbooks that cover the Standard Model, as there are many students who study it.
Much of what follows is taken from the book by Mark Thompson \cite{Thomson:1529540}.

The Standard Model Lagrangian can be defined as a sum of Lagrangians that each describe
the interactions between different fermions and bosons.
Equations of motion can be extracted from a Lagrangian $\mathcal{L}$
for a particle field $\phi_i$ using the Euler-Lagrange equations.

\begin{gather}
  \delta_\mu\left(\frac{\delta \mathcal{L}}{\delta(\delta_\mu \phi_i)}\right) -
  \frac{\delta \mathcal{L}}{\delta \phi_i} = 0
\end{gather}

In the measurement of $H\rightarrow \bb$ in associated production,
many components of the Standard Model are of interest.
These will be introduced as needed.
First, I will give a brief explanation of Higgs field's non-zero vacuum energy,
a trait that makes the Higgs one of the central keystones to Standard Model.
After that, the electroweak Lagrangian will be described since
the cross section of associated production depends on
the coupling of the Higgs Boson to the $W$ and $Z$ vector bosons.
The coupling of the electroweak force to fermions is also important to
understand both the generation of these intermediate states and
the resulting final state that the CMS detector records.
Another important factor for this work is a the decay of the Higgs boson itself into bottom quarks.
This depends on the Higgs directly coupling to fermions.
Finally, we will briefly consider the part of the Lagrangian describing the strong force.
Since the LHC is a hadron collider,
understanding of the strong force is required to extract data from LHC collisions.

\section{The Higgs Mechanism}

In both of these components of the Standard Model Lagrangian,
the Higgs coupling actually gives the vector bosons and massive fermions their mass
\cite{PhysRevLett.13.321, PhysRevLett.13.508, PhysRevLett.13.585}.
(In this work, neutrinos can be treated as massless.)
The granting of mass happens for two reasons:
the Higgs field has a non-zero vacuum expectation value,
and the Higgs field couples to vector boson and massive fermion fields.

The Higgs can be described as two complex scalar fields in a weak isospin doublet
with a quartic potential.
The Lagrangian for a free Higgs is then
\begin{gather}
  \mathcal{L} = (\delta_\mu \phi)^\dagger (\delta^\mu \phi) - (\mu^2(\phi^\dagger\phi) + \lambda(\phi^\dagger\phi)^2) \label{eq:free}
\end{gather}
Through the virial theorem, the potential has a minimum value when
\begin{gather}
  \phi^\dagger\phi = \frac{-\mu^2}{2\lambda} = \frac{v^2}{2} \label{eq:vacuum}
\end{gather}

This potential of the Higgs field breaks the SU(2) $\times$ U(1)
symmetry of the Standard Model Lagrangian.
Through this non-zero vacuum expectation value, the Higgs then has a constant influence
in other parts of the Standard Model Lagrangian.
For this measurement, three interactions that the Higgs makes
with this influence need to be considered:
the Higgs interacting with itself,
the Higgs interacting with the electroweak vector bosons,
and the Higgs interacting with quarks.

The first two interactions manifest in the Lagrangian when
we force the SU(2) $\times$ U(1) symmetry on the Lagrangian in Equation~(\ref{eq:free}).
The derivatives must be replaced.
\begin{gather}
  \delta_\mu \rightarrow D_\mu = \delta_\mu + i\frac{g_W}{2} {\boldsymbol \sigma} \cdot {\bf W_\mu} + ig'\frac{Y}{2}B_\mu
\end{gather}
$\phi$ can be rewritten to satisfy the vacuum expectation value in the gauge that will give us the massless neutral boson known as a photon.
\begin{gather}
  \phi(x) = \frac{1}{\sqrt{2}}
  \left(
  \begin{matrix}
    0 \\
    v + h(x)
  \end{matrix}
  \right)
\end{gather}
This leads to the following expansion for the kinetic term of the Lagrangian.
\begin{align}
  (D_\mu \phi)^\dagger(D^\mu \phi) = & \frac12 (\delta_\mu h)(\delta^\mu h)
  + \frac18 g_W^2 (W^{(1)}_\mu + iW^{(2)}_\mu)(W^{(1)\mu} - iW^{(2)\mu})(v + h)^2 \nonumber \\
  & + \frac18 (g_W W^{(3)}_\mu - g'B_\mu)(g_W W^{((3)\mu} - g' B^\mu)(v + h)^2 \label{eq:expanded}
\end{align}

Terms that are quadratic in terms of the gauge boson fields reveal the mass of the fields.
Taking $h(x) \rightarrow 0$, the terms for $W^{(1)}$ and $W^{(2)}$ are the just
\[
\frac14 g_W^2 v^2 W^{(1)}_\mu W^{(1)\mu} \quad \mathrm{and} \quad
\frac14 g_W^2 v^2 W^{(2)}_\mu W^{(2)\mu},
\]
giving the mass.
\begin{gather}
  m_W = \frac12 g_W v
\end{gather}
The quadratic terms for $W^{(3)}$ and $B$ mix to give a non-diagonal mass matrix ${\bf M}$.
\begin{gather}
  \frac{v^2}{8}
  \left(
  \begin{matrix}
  W^{(3)}_\mu & B_\mu
  \end{matrix}
  \right) {\bf M} \left(
  \begin{matrix}
  W^{(3)\mu} \\ B^\mu
  \end{matrix}
  \right) = 
  \frac{v^2}{8}
  \left(
  \begin{matrix}
  W^{(3)}_\mu & B_\mu
  \end{matrix}
  \right)
  \left(
  \begin{matrix}
    g_W^2 & -g_W g' \\
    -g_W g' & g'^2
  \end{matrix}
  \right)
  \left(
  \begin{matrix}
  W^{(3)\mu} \\ B^\mu
  \end{matrix}
  \right)
\end{gather}
The non-diagonal matrix allow $W^{(3)}$ and $B$ to mix.
Physical states must be represented by a diagonal Hamiltonian.
Diagonalizing the term above gives masses of the physical states.
\begin{gather}
  \frac18 v^2
  \left(
  \begin{matrix}
  A_\mu & Z_\mu
  \end{matrix}
  \right)
  \left(
  \begin{matrix}
    0 & 0 \\
    0 & g^2_W + g'^2
  \end{matrix}
  \right)
  \left(
  \begin{matrix}
  A^\mu \\ Z^\mu
  \end{matrix}
  \right) = 
  \frac12
  \left(
  \begin{matrix}
  A_\mu & Z_\mu
  \end{matrix}
  \right)
  \left(
  \begin{matrix}
    m_A^2 & 0 \\
    0 & m_Z^2
  \end{matrix}
  \right)
  \left(
  \begin{matrix}
  A^\mu \\ Z^\mu
  \end{matrix}
  \right) \label{eq:zmass}
\end{gather}
This gives us the masses of the neutral gauge bosons.
\begin{gather}
  m_A = 0 \quad \mathrm{and} \quad m_Z = \frac12 v \sqrt{g^2_W + g'^2}
\end{gather}

From the simple act of requiring SU(2) $\times$ U(1) symmetry on the Lagrangian of a scalar
doublet with non-zero vacuum expectation value,
the masses of all the electroweak gauge bosons have been produced.

\section{Associated Production}

The next thing to consider is the couplings also produced by this process.
The couplings will allow us to determine more precisely the parameters above
by measuring cross sections.

The physical states of $W^+$ and $W^-$ bosons can be determined by
the raising and lowering operators for isospin.
\begin{gather}
  W^\pm = \frac{1}{\sqrt2}\left(W^{(1)} \mp i W^{(2)}\right)
\end{gather}
The second term of Equation~(\ref{eq:expanded}) can be further expanded.
\begin{gather}
  \frac14 g^2_W W^-_\mu W^{+\mu} (v + h)^2 = \frac14 g^2_W v^2 W^-_\mu W^{+\mu} +
  \frac12 g^2_W v W^-_\mu W^{+\mu} h + \frac14 g^2_W W^-_\mu W^{+\mu} h^2 \label{eq:wwh}
\end{gather}
The second term on the right hand side of Equation~(\ref{eq:wwh}) gives us the coupling
strength of a vertex with a Higgs and two $W$ bosons.
\begin{gather}
  g_{HWW} = \frac12 g^2_W v = g_W m_W
\end{gather}
The coupling to the $Z$ boson can also be found from
Equation~(\ref{eq:zmass}) by substituting $(v + h)^2$ back in for $v^2$.
\begin{gather}
  g_{HZZ} = \frac12 \left(g^2_W + g'^2\right) v = \sqrt{g^2_W + g'^2} m_Z
\end{gather}
When arranged in a way that the $W$ or $Z$ boson radiates the Higgs,
as opposed to a Higgs decaying into a pair of $W$ or $Z$ bosons,
the process is called associated production or \emph{Higgstrahlung}.
The vertex showing associated production is pictured in Figure~\ref{fig:associated-production}.
\begin{figure}
  \centering
  \begin{fmffile}{associated_production}
    \begin{fmfgraph*}(200, 120)
      \fmfleft{i0}
      \fmfright{o0,o1}
      \fmf{dashes, label=$H$}{v0,o1}
      \fmf{boson, label=$W/Z$}{i0,v0,o0}
    \end{fmfgraph*}
  \end{fmffile}
  \caption[Feynman diagram of associated production]
          {
            Above is the Feynman diagram for associated production.
            The $W$ or $Z$ boson radiates a Higgs boson.
            Both bosons later decay into particles detected by CMS.
          }
  \label{fig:associated-production}
\end{figure}


\subsection{Production Mechanisms of Vector Bosons} \label{sec:produce-vector}

The $W$ and $Z$ bosons are themselves intermediate states,
never existing in a directly observable manner.
They must be produced through interacts with stable fermions.
Since the LHC is a hadron collider,
considering the vector bosons' couplings with quarks would be most relevant.

Quarks are fermions that couple to each other through the strong force,
resulting from a SU(3) symmetry.
There are three generations of quarks each consisting of a pair of quark types.
Their mass eigenstates are denoted as down-type or up-type.
Table~\ref{tab:quarks} displays some of the characteristics of these quarks.
\begin{table}
  \centering
  \caption[Quark flavors]{The quarks and some stuff about them.}
  {\renewcommand{\arraystretch}{1.5}
  \begin{tabular}{|l|c c c c|}
    \hline
    & 1st gen. & 2nd gen. & 3rd gen. & charge \\
    \hline
    down-type & $d$ & $s$ & $b$ & $-\frac13$ \\
    up-type & $u$ & $c$ & $t$ & $\frac23$ \\
    \hline
  \end{tabular}}
  \label{tab:quarks}
\end{table}
A feature of quarks is that their mass eigenstates do not match their weak eigenstates.
There is a mixing among the down-type quarks that is parametrized by the
Cabibbo-Kobayashi-Maskawa (CKM) matrix.
\begin{gather}
  \left(
  \begin{matrix}
    d' \\
    s' \\
    b'
  \end{matrix}
  \right)
  =
  \left(
  \begin{matrix}
    V_{ud} & V_{us} & V_{ub} \\
    V_{cd} & V_{cs} & V_{cb} \\
    V_{td} & V_{ts} & V_{tb}
  \end{matrix}
  \right)
  \left(
  \begin{matrix}
    d \\
    s \\
    b
  \end{matrix}
  \right)
\end{gather}
The mass eigenstates are denoted as $d, s$, and $b$,
while $d', s'$, and $b'$ are the weak eigenstates.
This mixing allows quarks to change generations through interaction with the weak force.
The following is the charge current vertex interaction.
\[
-i \frac{g_W}{\sqrt{2}}
\left(
\begin{matrix}
\bar{u} & \bar{c} & \bar{t}
\end{matrix}
\right)
\gamma^\mu \frac12 (1 - \gamma^5)
\left(
\begin{matrix}
  V_{ud} & V_{us} & V_{ub} \\
  V_{cd} & V_{cs} & V_{cb} \\
  V_{td} & V_{ts} & V_{tb}
\end{matrix}
\right)
\left(
\begin{matrix}
d \\ s \\ b
\end{matrix}
\right)
\]
The vertices for this interaction is shown in Figure~\ref{fig:w-production}
arranged in a way to show the processes of generating a $W^+$ or $W^-$ boson
from annihilating quarks.
\begin{figure}
  \centering
  \begin{fmffile}{w_plus_production}
    \begin{fmfgraph*}(180, 120)
      \fmfleft{i0,i1}
      \fmfright{o0}
      \fmf{quark}{i0,v0,i1}
      \fmf{boson, label=$W^+$}{v0,o0}
      \fmflabel{$u$}{i0}
      \fmflabel{$\bar{d}$}{i1}
      \fmflabel{V_{ud}\frac{g_W}{2}}{v0}
    \end{fmfgraph*}
  \end{fmffile}
  \begin{fmffile}{w_minus_production}
    \begin{fmfgraph*}(180, 120)
      \fmfleft{i0,i1}
      \fmfright{o0}
      \fmf{quark}{i1,v0,i0}
      \fmf{boson, label=$W^-$}{v0,o0}
      \fmflabel{$\bar{u}$}{i0}
      \fmflabel{$d$}{i1}
      \fmflabel{V_{ud}\frac{g_W}{2}}{v0}
    \end{fmfgraph*}
  \end{fmffile}
  \caption[Feynman diagram of generating $W^\pm$]
          {
            Above are diagrams for generating $W^+$ and $W^-$ bosons.
            the $u$ and $d$ quarks in the diagram can be replaced with
            any up-type or down-type quark, respectively.
            The CKM matrix element would in the vertex element would be changed accordingly.
          }
  \label{fig:w-production}
\end{figure}

Unifying electroweak forces doesn't quite get the Z to be an obvious thing.

Hadrons have anti-quarks in them too.
I'll show a diagram of the P.D.F.

% \cite{PhysRevD.2.1285}

\subsection{Decay Channels of Vector Bosons}

Due to the couplings described in Section~\ref{sec:produce-vector},
the vector bosons decay into quarks.
However, in the hadronic environment produced at the LHC
these are not the best indicators of a vector boson intermediate state.
This measurement uses leptonic decays in the final state
since the contributions of the vector boson intermediate states to leptonic final states
of appropriate kinematics are significantly larger.


\section{Characteristics of the Higgs}

What we are ultimately interested in measuring is the contribution of the
Higgs intermediate state to the final state of \bb.

\subsection{Energy Spectrum}

\subsection{Decay to \bb}

\section{Other Relevant Standard Model Processes}

There are other processes that must be acknowledged in order to explain how the CMS detector works.
