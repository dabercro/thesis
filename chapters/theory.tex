\chapter{Theory} \label{ch:theory}

The Standard Model describes the interactions of all observable matter.
There are many textbooks that cover the Standard Model, as there are many scientists and students who study it.
For an in depth presentation of the Standard Model beyond what is presented in this chapter,
please refer to Reference~\cite{Thomson:1529540}.

Matter is made up of 12 kinds of fermions.
The forces between the fermions are mediated by the gauge fields
created by the Standard Model's SU(3) $\times$ SU(2$)_L$ $\times$ U(1$)_Y$ symmetry.
Fermions with the appropriate charge are affected by the gauge fields.
Fermions are classified as quarks or leptons.
There are six types of quarks, separated into three generations of two quarks each.
Each quark has a color charge associated with the SU(3) symmetry.
The interactions arising from this is called QCD.
Each pair of quarks in a family also have an approximate SU(2) symmetry,
which allows interactions via the weak force.
Quarks also have a hypercharge, which is a relation of electromagnetic charge and weak isospin,
meaning the gauge field from the U(1) symmetry affects them as well.
Three charged leptons and three neutral leptons, called neutrinos, comprise the other six fermions.
The leptons do not carry a color charge, so they are not affected by the SU(3) symmetry,
but they do carry weak isospin and hypercharge.
Table~\ref{tab:fermions} displays the values of these charges for all fermions.
Fermions also have anti-particles, which carry the opposite charges of their counterparts.
\begin{table}
  \centering
  \caption[Fermion charges]{
    All of the fermions are listed below,
    along with their charges and weak isospin values.
    The three generations are listed from least to most massive,
    meaning only the first generation of quarks and charged leptons is stable.
    The masses and decays of neutrinos is beyond the scope of the Standard Model and this analysis.
  }
  {\renewcommand{\arraystretch}{1.5}
  \begin{tabular}{|l|c c c c c c c|}
    \hline
    & 1st gen. & 2nd gen. & 3rd gen. & Color & $Q$ & $I_W^{(3)}$ & Y \\
    \hline
    down-type quarks & $d$ & $s$ & $b$ & yes & $-\frac13$ & $-\frac12$ & $-\frac13$ \\
    up-type quarks & $u$ & $c$ & $t$ & yes & $+\frac23$ & $+\frac12$ & $+\frac53$ \\
    charged leptons & $e$ & $\mu$ & $\tau$ & no & -1 & $-\frac12$ & -1 \\
    neutral leptons & $\nu_e$ & $\nu_\mu$ & $\nu_\tau$ & no & 0 & $+\frac12$ & -1 \\
    \hline
  \end{tabular}}
  \label{tab:fermions}
\end{table}

In the Standard Model, the gauge files from the SU(2$)_L$ $\times$ U(1$)_Y$ symmetries are mixed
into what is known as the electroweak force.
The electroweak force is mediated by the neutral photon and $Z$ boson, and the charged $W$ boson.
This mixing happens due to the Higgs boson, a scalar which grants all of the charged fermions and
the $Z$ and $W$ bosons mass through interactions.

In associated production, a Higgs boson is created through radiation from the $Z$ and $W$ bosons,
which are also known as vector bosons.
The measurement of $H\rightarrow \bb$ in associated production
is therefore a measurement of interactions between the Higgs boson and vector bosons
and the coupling between the Higgs boson and bottom, or $b$, quarks.
The generation and observable decay of the vector bosons are also important to make this measurement.
The parameters for those interactions are measured more accurately by other analyses
not involving an observation of the Higgs \cite{sabrandt}.
Since the coupling of the Higgs boson with the vector bosons and with fermions also gives rise to the masses
of each in the Standard Model through what is called electroweak symmetry breaking or the Higgs mechanism,
most discussions of Higgs couplings include an explanation of the Higgs mechanism.

The treatment of these topics in this chapter are arranged as follows.
First, I will give a brief explanation of Higgs field's non-zero vacuum energy,
a trait that makes the electroweak symmetry breaking possible.
After that, the coupling of the Higgs boson to the $W$ and $Z$ vector bosons will be described.
These interactions arise from a mixing of the weak force with the electromagnetic force.
What results is collectively known as the electroweak force.
The coupling of the electroweak force to fermions is then discussed to
understand both the generation of the vector boson intermediate states and
the resulting final state that can be observed.
Finally, the decay of the Higgs boson itself into bottom quarks is explained.
This is allowed because the Higgs boson couples directly to massive fermions
and gives them mass through the Higgs mechanism.

The Standard model is described by its Lagrangian,
which is the difference between a system's kinetic and potential energies.
Equations of motion are extracted from a Lagrangian $\mathcal{L}$
for a particle field $\phi_i$ using the Euler-Lagrange equations.

\begin{gather}
  \delta_\mu\left(\frac{\delta \mathcal{L}}{\delta(\delta_\mu \phi_i)}\right) -
  \frac{\delta \mathcal{L}}{\delta \phi_i} = 0
\end{gather}

Gauge bosons are produced by symmetries of the Lagrangian.
The Standard Model has SU(3) $\times$ SU(2$)_L$ $\times$ U(1$)_Y$ symmetry.
The SU(3) symmetry produces gluons which mediate the strong force between quarks \cite{PhysRevLett.30.1343}.
The SU(2$)_L$ $\times$ U(1$)_Y$ symmetry produces two gauge fields.
The first interacts with left handed fermions,
and the second interacts with all fermions through their hypercharge, $Y$.
These forces are ultimately mixed into what is known as the electroweak force
due to electroweak symmetry breaking \cite{PhysRevLett.19.1264}.
Electroweak symmetry breaking is the required solution of the problem that
vector gauge bosons of the electroweak force had mass.
It was not possible to grant these bosons mass while maintaining the symmetries of the Standard Model
without the Higgs boson \cite{PhysRevLett.13.321, PhysRevLett.13.508, PhysRevLett.13.585}.
The granting of mass happens for two reasons:
the Higgs field has a non-zero vacuum expectation value,
and the Higgs field couples directly to vector boson and massive fermion fields.

\section{Electroweak Symmetry Breaking}

First, consider the SU(2) $\times$ U(1) symmetry where the Higgs interacts with the electroweak bosons.
The $L$ and $Y$ of the SU(2$)_L$ $\times$ U(1$)_Y$ symmetry can be forgotten for the moment,
since they describe fermion interactions with the electroweak force.
To preserve SU(2) symmetry, the Higgs boson is described as two complex scalar fields in
a weak isospin doublet with a quartic potential.
The SU(2) symmetry means rotations between the doublet states must be equivalent in the Lagrangian.
The Lagrangian for a free Higgs is then
\begin{gather}
  \mathcal{L} = (\delta_\mu \phi)^\dagger (\delta^\mu \phi) - (\mu^2(\phi^\dagger\phi) + \lambda(\phi^\dagger\phi)^2) \label{eq:free}
\end{gather}
Through the virial theorem \cite{1930ZPhy...63..855F}, the potential has a minimum value when
\begin{gather}
  \phi^\dagger\phi = \frac{-\mu^2}{2\lambda} = \frac{v^2}{2} \label{eq:vacuum}
\end{gather}

This potential of the Higgs field breaks the SU(2) $\times$ U(1)
symmetry of the Standard Model Lagrangian.
Through this non-zero vacuum expectation value, the Higgs then has a constant influence
in other parts of the Standard Model Lagrangian.
In this way, it gives mass to electroweak vector bosons,
to itself, and to massive fermions.

The first two sets of masses manifest when
we force the SU(2) $\times$ U(1) symmetry back onto the Lagrangian in Equation~(\ref{eq:free}).
The derivatives must be replaced.
\begin{gather}
  \delta_\mu \rightarrow D_\mu = \delta_\mu + i\frac{g_W}{2} {\boldsymbol \sigma} \cdot {\bf W_\mu} + ig'\frac{Y}{2}B_\mu
\end{gather}
To simplify the expansion of Equation~(\ref{eq:free}),
a particular gauge, or particular doublet state, is chosen.
$\phi$ is written to satisfy the vacuum expectation value of Equation~(\ref{eq:vacuum})
in the gauge that will give us the massless neutral boson known as a photon.
\begin{gather}
  \phi(x) = \frac{1}{\sqrt{2}}
  \left(
  \begin{matrix}
    0 \\
    v + h(x)
  \end{matrix}
  \right) \label{eq:higgs-doublet}
\end{gather}
This leads to the following expansion for the kinetic term of the Lagrangian.
\begin{align}
  (D_\mu \phi)^\dagger(D^\mu \phi) = & \frac12 (\delta_\mu h)(\delta^\mu h)
  + \frac18 g_W^2 (W^{(1)}_\mu + iW^{(2)}_\mu)(W^{(1)\mu} - iW^{(2)\mu})(v + h)^2 \nonumber \\
  & + \frac18 (g_W W^{(3)}_\mu - g'B_\mu)(g_W W^{((3)\mu} - g' B^\mu)(v + h)^2 \label{eq:expanded}
\end{align}

Terms that are quadratic in terms of the gauge boson fields reveal the mass of the fields.
Taking $h(x) \rightarrow 0$, the terms for $W^{(1)}$ and $W^{(2)}$ are the just
\[
\frac14 g_W^2 v^2 W^{(1)}_\mu W^{(1)\mu} \quad \mathrm{and} \quad
\frac14 g_W^2 v^2 W^{(2)}_\mu W^{(2)\mu},
\]
giving the mass.
\begin{gather}
  m_W = \frac12 g_W v
\end{gather}
The quadratic terms for $W^{(3)}$ and $B$ mix to give a non-diagonal mass matrix ${\bf M}$.
\begin{gather}
  \frac{v^2}{8}
  \left(
  \begin{matrix}
  W^{(3)}_\mu & B_\mu
  \end{matrix}
  \right) {\bf M} \left(
  \begin{matrix}
  W^{(3)\mu} \\ B^\mu
  \end{matrix}
  \right) = 
  \frac{v^2}{8}
  \left(
  \begin{matrix}
  W^{(3)}_\mu & B_\mu
  \end{matrix}
  \right)
  \left(
  \begin{matrix}
    g_W^2 & -g_W g' \\
    -g_W g' & g'^2
  \end{matrix}
  \right)
  \left(
  \begin{matrix}
  W^{(3)\mu} \\ B^\mu
  \end{matrix}
  \right) \label{eq:mixing}
\end{gather}
The non-diagonal matrix allow $W^{(3)}$ and $B$ to mix.
Physical states must be represented by a diagonal Hamiltonian.
Diagonalizing the term above gives masses of the physical states.
\begin{gather}
  \frac18 v^2
  \left(
  \begin{matrix}
  A_\mu & Z_\mu
  \end{matrix}
  \right)
  \left(
  \begin{matrix}
    0 & 0 \\
    0 & g^2_W + g'^2
  \end{matrix}
  \right)
  \left(
  \begin{matrix}
  A^\mu \\ Z^\mu
  \end{matrix}
  \right) = 
  \frac12
  \left(
  \begin{matrix}
  A_\mu & Z_\mu
  \end{matrix}
  \right)
  \left(
  \begin{matrix}
    m_A^2 & 0 \\
    0 & m_Z^2
  \end{matrix}
  \right)
  \left(
  \begin{matrix}
  A^\mu \\ Z^\mu
  \end{matrix}
  \right) \label{eq:zmass}
\end{gather}
This gives us the masses of the neutral gauge bosons.
\begin{gather}
  m_A = 0 \quad \mathrm{and} \quad m_Z = \frac12 v \sqrt{g^2_W + g'^2}
\end{gather}

From the simple act of requiring SU(2) $\times$ U(1) symmetry on the Lagrangian of a scalar
doublet with non-zero vacuum expectation value,
the masses of all the electroweak gauge bosons have been produced.
A similar procedure will produce the masses of fermions due to their coupling to the electroweak force.

\section{Associated Production}

The next thing to consider is the couplings also produced by this process.
The couplings will allow us to determine more precisely the parameters above
by measuring cross sections.

The physical states of $W^+$ and $W^-$ bosons can be written as
the raising and lowering operators for isospin.
\begin{gather}
  W^\pm = \frac{1}{\sqrt2}\left(W^{(1)} \mp i W^{(2)}\right) \label{eq:w-form}
\end{gather}
The second term of Equation~(\ref{eq:expanded}) can be further expanded.
\begin{gather}
  \frac14 g^2_W W^-_\mu W^{+\mu} (v + h)^2 = \frac14 g^2_W v^2 W^-_\mu W^{+\mu} +
  \frac12 g^2_W v W^-_\mu W^{+\mu} h + \frac14 g^2_W W^-_\mu W^{+\mu} h^2 \label{eq:wwh}
\end{gather}
The second term on the right hand side of Equation~(\ref{eq:wwh}) gives us the coupling
strength of a vertex with a Higgs and two $W$ bosons.
\begin{gather}
  g_{HWW} = \frac12 g^2_W v = g_W m_W
\end{gather}
The coupling to the $Z$ boson can also be found from
Equation~(\ref{eq:zmass}) by substituting $(v + h)^2$ back in for $v^2$ and
extracting the terms proportional to $h Z_\mu Z^\mu$.
\begin{gather}
  g_{HZZ} = \frac12 \left(g^2_W + g'^2\right) v = \sqrt{g^2_W + g'^2} m_Z
\end{gather}
When arranged in a way that the $W$ or $Z$ boson radiates the Higgs,
as opposed to a Higgs decaying into a pair of $W$ or $Z$ bosons,
the process is called associated production or \emph{Higgstrahlung}.
The vertex showing associated production is pictured in Figure~\ref{fig:associated-production}.
\begin{figure}
  \centering
  \begin{fmffile}{associated_production}
    \fmfframe(0,0)(0, 20){
    \begin{fmfgraph*}(200, 120)
      \fmfleft{i0}
      \fmfright{o0,o1}
      \fmf{dashes, label=$H$}{v0,o1}
      \fmf{boson, label=$W/Z$}{i0,v0,o0}
    \end{fmfgraph*}
    }
  \end{fmffile}
  \caption[Feynman diagram of associated production]
          {
            Above is the Feynman diagram for associated production.
            The $W$ or $Z$ boson radiates a Higgs boson.
            Both bosons later decay into particles detected by CMS.
          }
  \label{fig:associated-production}
\end{figure}


\subsection{Coupling Between Vector Bosons and Fermions} \label{sec:produce-vector}

The $W$ and $Z$ bosons are themselves intermediate states,
never existing in a directly observable manner.
They must be produced through interacts with stable fermions.
Since the LHC is a hadron collider,
considering the vector bosons' couplings with quarks would be most relevant.

Quarks couple to each other through the strong force,
resulting from a SU(3) symmetry.
There are three generations of quarks each consisting of a pair of quark types.
Their mass eigenstates are denoted as down-type or up-type.
A feature of quarks is that their mass eigenstates do not match their weak eigenstates.
There is a mixing among the down-type quarks that is parametrized by the
Cabibbo-Kobayashi-Maskawa (CKM) matrix.
\begin{gather}
  \left(
  \begin{matrix}
    d' \\
    s' \\
    b'
  \end{matrix}
  \right)
  =
  \left(
  \begin{matrix}
    V_{ud} & V_{us} & V_{ub} \\
    V_{cd} & V_{cs} & V_{cb} \\
    V_{td} & V_{ts} & V_{tb}
  \end{matrix}
  \right)
  \left(
  \begin{matrix}
    d \\
    s \\
    b
  \end{matrix}
  \right) \label{eq:ckm}
\end{gather}
The mass eigenstates are denoted as $d, s$, and $b$,
while $d', s'$, and $b'$ are the weak eigenstates.
This mixing allows quarks to change generations through interaction with $W^\pm$ bosons,
which raise or lower the weak isospin.
The following is the charge current vertex interaction.
\[
-i \frac{g_W}{\sqrt{2}}
\left(
\begin{matrix}
\bar{u} & \bar{c} & \bar{t}
\end{matrix}
\right)
\gamma^\mu \frac12 (1 - \gamma^5)
\left(
\begin{matrix}
  V_{ud} & V_{us} & V_{ub} \\
  V_{cd} & V_{cs} & V_{cb} \\
  V_{td} & V_{ts} & V_{tb}
\end{matrix}
\right)
\left(
\begin{matrix}
d \\ s \\ b
\end{matrix}
\right)
\]
The vertices for this interaction is shown in Figure~\ref{fig:w-production}
arranged in a way to show the processes of generating a $W^+$ or $W^-$ boson
from annihilating quarks.
\begin{figure}
  \centering
  \begin{fmffile}{w_plus_production}
    \fmfframe(0,0)(0, 20){
    \begin{fmfgraph*}(180, 120)
      \fmfleft{i0,i1}
      \fmfright{o0}
      \fmf{quark}{i0,v0,i1}
      \fmf{boson, label=$W^+$}{v0,o0}
      \fmflabel{$u$}{i0}
      \fmflabel{$\bar{d}$}{i1}
      \fmflabel{$V_{ud}\frac{g_W}{2}$}{v0}
    \end{fmfgraph*}
    }
  \end{fmffile}
  \begin{fmffile}{w_minus_production}
    \fmfframe(0,0)(0, 20){
    \begin{fmfgraph*}(180, 120)
      \fmfleft{i0,i1}
      \fmfright{o0}
      \fmf{quark}{i1,v0,i0}
      \fmf{boson, label=$W^-$}{v0,o0}
      \fmflabel{$\bar{u}$}{i0}
      \fmflabel{$d$}{i1}
      \fmflabel{$V_{ud}\frac{g_W}{2}$}{v0}
    \end{fmfgraph*}
    }
  \end{fmffile}
  \caption[Feynman diagram of generating $W^\pm$]
          {
            Above are diagrams for generating $W^+$ and $W^-$ bosons.
            The $u$ and $d$ quarks in the diagram can be replaced with
            any up-type or down-type quark, respectively.
            The CKM matrix element would in the vertex element would be changed accordingly.
          }
  \label{fig:w-production}
\end{figure}
The $\gamma$ matrices in the interaction are present because the SU(2) component of
the Standard Model only interacts with left-handed fermions and right-handed anti-fermions.
For this reason, the SU(2) component is more accurately labelled SU$(2)_L$.
From Equation~(\ref{eq:w-form}), the $W^\pm$ bosons are completely made up of the
$W^{(1)}$ and $W^{(2)}$ components of the SU$(2)_L$, so they also only interact with
left-handed fermions and right-handed anti-fermions.

Both the photon and the $Z$ boson mix the SU$(2)_L$ and U(1$)_Y$ components of the Standard Model.
Production of the $Z$ boson needs to be directly understood for this measurement,
but it is more straightforward to determine the strength of the $Z$ boson couplings to left-
and right-handed fermions by exploiting the symmetry of photon interactions.
That is, the photon interacts the same with left and right handed charged fermions,
and not at all with neutral fermions.
This is shown directly with experiments with leptons.
The charged leptons, electrons, muons, and taus, interact with photons,
while the respective neutrinos do not.
From the mixing in Equation~(\ref{eq:mixing}),
the photon and $Z$ fields can be expressed as the following.
\begin{gather}
  A_\mu = B_\mu \cos \theta_W + W_\mu^{(3)} \sin \theta_W \\
  Z_\mu = - B_\mu \sin \theta_W + W_\mu^{(3)} \cos \theta_W \label{eq:z-force}
\end{gather}
$\theta_W$ is known as the weak mixing angle.
The relative strengths of the $B$ and $W^{(3)}$ couplings
are determined directly through lepton electro-magnetic characteristics,
keeping in mind that $W^{(3)}$ only interacts with left handed particles.
The following are the electro-magnetic interaction strengths of left- and right-handed electrons and neutrinos.
\begin{align}
  e_L:& \qquad Qe = \frac12 g' Y_{e_L} \cos \theta_W - \frac12 g_W \sin \theta_W \\
  \nu_L:& \qquad \phantom{Q}0 = \frac12 g' Y_{\nu_L} \cos \theta_W - \frac12 g_W \sin \theta_W \\
  e_R:& \qquad Qe = \frac12 g' Y_{e_R} \cos \theta_W \\
  \nu_R:& \qquad \phantom{Q}0 = \frac12 g' Y_{\nu_R} \cos \theta_W
\end{align}
$Y_{e_L}$ and $Y_{\nu_L}$ must be equal to maintain SU$(2)_L$ symmetry.
To satisfy these contraints, the follow definition of $Y$ is needed.
\begin{gather}
  Y = 2\left(Q - I_W^{(3)}\right)
\end{gather}
The following relationship also arises from these experimental constraints.
\begin{gather}
  e = g_W \sin \theta_W = g' \cos \theta_W
\end{gather}

Returning to the $Z$ boson, from Equation~(\ref{eq:z-force}),
and defining
\begin{gather}
  g_Z = \frac{e}{\sin \theta_W \cos \theta_W},
\end{gather}
we have the following couplings to left- and right-handed fermions.
\begin{align}
  -\frac12 g' \sin \theta_W (Y_{f_L} \bar{u}_L \gamma^\mu u_L &+ Y_{f_R} \bar{u}_R \gamma^\mu u_R) + I_W^{(3)} g_W \cos \theta_W \left( \bar{u}_L \gamma^\mu u_L \right) = \nonumber \\
  & g_Z\left(\left( I^{(3)} - Q \sin^2 \theta_W \right)\bar{u}_L \gamma^\mu u_L
  - Q \sin^2 \theta_W \bar{u}_R \gamma^\mu u_R\right)
\end{align}
Now the coupling of the $Z$ to left- and right-handed quarks can be calculated from
Table~\ref{tab:fermions}, remembering that $I_W^{(3)}$ for right-handed fermions is 0.
Diagrams showing the interaction strengths of fermion-$Z$ vertices are shown
in Figure~\ref{fig:z-production}.
\begin{figure}
  \centering
  \begin{fmffile}{z_left_production}
    \fmfframe(0,0)(0, 20){
    \begin{fmfgraph*}(150, 120)
      \fmfleft{i0,i1}
      \fmfright{o0}
      \fmf{quark}{i0,v0,i1}
      \fmf{boson, label=$Z$}{v0,o0}
      \fmflabel{$f_L$}{i0}
      \fmflabel{$\bar{f}_R$}{i1}
      \fmflabel{$g_Z\left( I^{(3)} - Q \sin^2 \theta_W \right)$}{v0}
    \end{fmfgraph*}
    }
  \end{fmffile}
  \hspace{24pt}
  \begin{fmffile}{z_right_production}
    \fmfframe(0,0)(0, 20){
    \begin{fmfgraph*}(150, 120)
      \fmfleft{i0,i1}
      \fmfright{o0}
      \fmf{quark}{i1,v0,i0}
      \fmf{boson, label=$Z$}{v0,o0}
      \fmflabel{$\bar{f}_L$}{i0}
      \fmflabel{$f_R$}{i1}
      \fmflabel{$- g_Z Q \sin^2 \theta_W$}{v0}
    \end{fmfgraph*}
    }
  \end{fmffile}
  \vspace{6pt}
  \caption[Feynman diagram of generating $Z$]
          {
            Above are diagrams for generating $Z$ bosons.
            Left- and right-handed fermions are both coupled to,
            but with different coupling strengths.
          }
  \label{fig:z-production}
\end{figure}

Thus vector bosons couple to quarks, the constituents of hadrons,
which means they can be produced at the LHC.
As mentioned earlier in this section,
quarks interact through an SU(3) symmetry that results in the strong force.
The three states that this symmetry supports are known as color states,
and they are labelled red, green, and blue, or $r$, $g$, and $b$.
There are also anti-states for each color state,
labelled $\bar{r}$, $\bar{g}$, and $\bar{b}$.
The resulting gauge bosons are known as gluons, and they carry the following color states.
\[
r\bar{g}, g\bar{r}, r\bar{b}, b\bar{r}, g\bar{b}, b\bar{g}, \frac{1}{\sqrt{2}}(r\bar{r} - g\bar{g}) \mathrm{and} \frac{1}{\sqrt{6}}(r\bar{r} + g\bar{g} - 2b\bar{b})
\]
Since gluons carry color charge, they interact with other gluons.
As the distance between two color-charged particles grows,
the energy density of the self-interacting gluon field remains constant.
It soon becomes energetically favorable for new a particle/anti-particle pair
to pop into existence if it simultaneously reduces the distance that the strong force is interacting.
As a result, all observable hadronic states are color singlets.
The most common hadronic states are mesons, made of a quark/anti-quark pair
with the color singlet state
\begin{gather}
  \psi(q\bar{q}) = \frac1{\sqrt{3}} (r\bar{r} + g\bar{g} + b\bar{b}),
\end{gather}
and baryons, made of three quarks with the following color singlet state.
\begin{gather}
  \psi(qqq) = \frac{1}{\sqrt{6}}(rgb - rbg + gbr  grb + brg - bgr) \label{eq:baryon-singlet}
\end{gather}
Baryons can also be composed of three anti-quarks,
which has a state corresponding to Equation~(\ref{eq:baryon-singlet}),
but with anti-color.
The resulting spray of hadronic particles generated from the vacuum to restore
color singlets are called jets.

For this measurement, protons are collided at the LHC.
The proton consists of two $u$ quarks, and one $d$ quark.
Since the three quarks inside the proton interact strongly,
there are also many virtual gluons and quark/anti-quark pairs present at all times.
The quantity and energies of all these partons are not able to be calculated
since QCD is non-perturbative.
They can be measured in deep inelastic scattering experiments though.
In these, electrons are scattered off of protons,
and parton distribution functions (PDFs) can be measured.
The PDFs for protons are shown in Figure~\ref{fig:pdf}.
\begin{figure}
  \centering
  \includegraphics[width=0.7\linewidth]{figures/CTEQ6_parton_distribution_functions.png}
  \caption[Parton Distribution Function for protons]{
    The Parton Distribution Function for protons is shown above.
    Most of the proton's momentum is carried by $u$ and $d$ quarks,
    but virtual $s$ quarks as well as gluons can also interact with particles passing through the proton.
  }
  \label{fig:pdf}
\end{figure}
% https://en.wikipedia.org/wiki/Parton_(particle_physics)#/media/File:CTEQ6_parton_distribution_functions.png

Combining the known proton energy, PDFs, the CKM matrix, and the theory of the electroweak force,
we can predict the cross section of generating $W$ and $Z$ bosons at the LHC.
These initial vector bosons will be off-shell,
which means they will have a mass much different than the resonance peak.
Then they will radiate a Higgs in order to most commonly produce an on-shell vector boson and on-shell Higgs boson.
The cross section of generating off-shell particles are suppressed according to
the required center-of-mass energy, $E$, and the resonance mass $M$.
The suppression is in the form of the relativistic Breit-Wigner formula.
\begin{gather}
  f(E) = \frac{k}{(E^2 - M^2)^2 + M^2\Gamma^2} \label{eq:breitwigner}
\end{gather}
This associated production is one of three production mechanisms of the Higgs Boson.
The other two are gluon fusion, where gluons form a top loop, and vector boson fusion,
both shown in Figure~\ref{fig:other-prod}.
\begin{figure}
  \centering
  \begin{fmffile}{gluon_fusion}
    \fmfframe(0,0)(0, 20){
    \begin{fmfgraph*}(150, 120)
      \fmfleft{i0,i1}
      \fmfright{o0}
      \fmf{gluon}{i0,v0}
      \fmf{gluon}{i1,v1}
      \fmf{fermion, label=$t$}{v0,v1,v2,v0}
      \fmf{dashes, label=$H$}{v2,o0}
    \end{fmfgraph*}
    }
  \end{fmffile}
  \hspace{24pt}
  \begin{fmffile}{vbf}
    \fmfframe(0,0)(0, 20){
    \begin{fmfgraph*}(150, 120)
      \fmfleft{i0,i1}
      \fmfright{o0}
      \fmf{boson, label=$W/Z$}{i0,v0,i1}
      \fmf{dashes, label=$H$}{v0,o0}
    \end{fmfgraph*}
    }
  \end{fmffile}
  \vspace{6pt}
  \caption[Feynman diagrams of other production mechanisms]
          {
            Above are the Feynman diagrams for other production mechanisms of the Higgs boson.
            Gluon fusion is shown on the left, and vector boson fusion is shown on the right.
          }
  \label{fig:other-prod}
\end{figure}
In these other two production mechanisms, only the Higgs is in the final state.
These events can only offer additional identification through initial state radiation.
In contrast, associated production also results in leptons from the vector boson decay,
which allow for tighter selection criteria for event identification.

% \cite{PhysRevD.2.1285}

\subsection{Decay Channels of Vector Bosons} \label{sec:v-decay}

Due to the couplings described in Section~\ref{sec:produce-vector},
the vector bosons decay into quarks.
However, in the hadronic environment produced at the LHC
these are not the best indicators of a vector boson intermediate state.
This measurement uses leptonic decays in the final state
since they are easier to identify and separate from background processes.

There are three generations of leptons.
Each generation consists of a charged lepton, and a neutral lepton, also referred to as a neutrino.
The left-handed charged lepton and neutrino of each generation form an electroweak SU(2$)_L$ doublet.
In order of increasing mass, the three generations are called electron, muon, and tau.
Heavier charged leptons decay into lighter leptons via the weak force.
Two neutrinos result from this decay, as shown in Figure~\ref{fig:tau-decay},
making the characteristics of the parent lepton's parent difficult to reconstruct.
The tau lepton has a short enough lifetime to consistently decay before reaching the CMS detector.
The tau lepton is also massive enough to also decay into quarks,
making its measurement even more complicated.
Muons have an average lifetime long enough to penetrate the entire detector,
and electrons are stable particles.
As a result, only final states with muons and electrons are considered in this analysis.
\begin{figure}
  \centering
  \begin{fmffile}{tau_decay}
    \fmfframe(0,0)(0, 20){
    \begin{fmfgraph*}(150, 120)
      \fmfleft{i0}
      \fmfright{o0,o1,o2}
      \fmf{fermion}{i0,v0,o0}
      \fmf{boson, label=$W^-$}{v0,v1}
      \fmf{fermion}{o1,v1,o2}
      \fmflabel{$\tau^-$}{i0}
      \fmflabel{$\nu_\tau$}{o0}
      \fmflabel{$\bar{\nu}_\mu$}{o1}
      \fmflabel{$\mu^-$}{o2}
    \end{fmfgraph*}
    }
  \end{fmffile}
  \caption[Tau decay]{
    Heavier leptons can decay to lighter leptons while emitting two neutrinos.
    Above is an example of a decay of $\tau \rightarrow \nu_\tau\mu\bar{\nu}_\mu$.
  }
  \label{fig:tau-decay}
\end{figure}
The Feynman diagrams for the decay channels of interest are shown in Figure~\ref{fig:v-decay}.
\begin{figure}
  \centering
  \begin{fmffile}{z_zero_lep}
    \fmfframe(0, 0)(0, 30){
    \begin{fmfgraph*}(150, 120)
      \fmfleft{i0}
      \fmfright{o0,o1}
      \fmf{boson, label=$Z$}{i0,v0}
      \fmf{fermion}{o0,v0,o1}
      \fmflabel{$\bar{\nu}_e/\bar{\nu}_\mu/\bar{\nu}_\tau$}{o0}
      \fmflabel{$\nu_e/\nu_\mu/\nu_\tau$}{o1}
    \end{fmfgraph*}
    }
  \end{fmffile}
  \begin{fmffile}{z_two_lep}
    \fmfframe(0, 0)(0, 30){
    \begin{fmfgraph*}(150, 120)
      \fmfleft{i0}
      \fmfright{o0,o1}
      \fmf{boson, label=$Z$}{i0,v0}
      \fmf{fermion}{o0,v0,o1}
      \fmflabel{$e^+/\mu^+$}{o0}
      \fmflabel{$e^-/\mu^-$}{o1}
    \end{fmfgraph*}
    }
  \end{fmffile}
  \begin{fmffile}{w_one_lep}
    \fmfframe(0,30)(0, 20){
    \begin{fmfgraph*}(150, 120)
      \fmfleft{i0}
      \fmfright{o0,o1}
      \fmf{boson, label=$W^-$}{i0,v0}
      \fmf{fermion}{o0,v0,o1}
      \fmflabel{$\bar{\nu}_e/\bar{\nu}_\mu$}{o0}
      \fmflabel{$e^-/\mu^-$}{o1}
    \end{fmfgraph*}
    }
  \end{fmffile}
  \begin{fmffile}{wp_one_lep}
    \fmfframe(0,30)(0, 20){
    \begin{fmfgraph*}(150, 120)
      \fmfleft{i0}
      \fmfright{o0,o1}
      \fmf{boson, label=$W^+$}{i0,v0}
      \fmf{fermion}{o0,v0,o1}
      \fmflabel{$\nu_e/\nu_\mu$}{o0}
      \fmflabel{$e^+/\mu^+$}{o1}
    \end{fmfgraph*}
    }
  \end{fmffile}
  \caption[Vector Boson decays in the analysis]{
    Above are the three different vector boson decays we are interested in.
    $\tau$ decays do also contribute to the charged lepton final states as seen by a detector,
    but the energy carried away by neutrinos significantly reduces those decay modes' contribution
    to the accepted states.
  }
  \label{fig:v-decay}
\end{figure}


\section{Decay Channels of the Higgs}

What we are ultimately interested in measuring is the contribution of the
Higgs intermediate state to the final state of \bb.
Since the Higgs is a SU$(2)_L$ doublet of scalar fields,
the term $-g_f(\bar{L}\phi R + \bar{R} \phi^\dagger L)$ in the
Standard Model Lagrangian is invariant under SU$(2)_L \times$ SU$(1)_Y$,
where $L$ is a left-handed fermion doublet, and $R$ is a right-handed singlet.
If the Higgs doublet is expanded around the vacuum expectation value,
as Equation~(\ref{eq:higgs-doublet}), the Lagrangian term becomes the following.
\begin{gather}
  \mathcal{L}_f =
  -\frac{g_f}{\sqrt2} v \left(\bar{f}_L f_R + \bar{f}_R f_L \right)
  -\frac{g_f}{\sqrt2} h \left(\bar{f}_L f_R + \bar{f}_R f_L \right)
  \label{eq:fermion-mass}
\end{gather}
In Equation~(\ref{eq:fermion-mass}),
$f$ refers to the lower field of the fermion's SU$(2)_L$ doublet.
The Lagrangian also includes terms for the upper field since the conjugate of $\phi$
has the same symmetries as $\phi$.

The Lagrangian showing fermion-Higgs interactions in
Equation~(\ref{eq:fermion-mass}) consists of two terms.
Since $v$ is constant, the first term is consistent with a fermion's mass,
assuming an appropriate coupling constant.
\begin{gather}
  g_f = \sqrt2 \frac{m_f}{v}
\end{gather}
The second term is the coupling of the fermion to the Higgs field
with the same coupling constant.
This is the mechanism by which the Higgs give fermions their masses,
and also why the Higgs couples more strongly to massive particles.
The Feynman rule for the interaction vertex between the Higgs and fermions
is proportional to the fermion's mass.
Of the quarks, the $b$ quark is the second most massive.
The most massive $t$ quark is too massive to be the final decay product of an on-shell Higgs.
The required off-shell Higgs would need a center of mass energy, $E$, at least \SI{350}{GeV},
and the cross section for this production drops off as the tail of a relativistic Breit-Wigner function
in Equation~(\ref{eq:breitwigner}) with a mass $M$ of \SI{125}{GeV}.
The Higgs can also decay to two vector bosons.
In this case, instead of requiring the Higgs to be off-shell,
one of the unstable vector bosons can be less massive than its resonance mass.
However, this still results cross section suppression in the form of Equation~(\ref{eq:breitwigner}).

As the decay mode with the highest coupling requiring no off-shell particles,
the predicted branching ratio of $H \rightarrow \bb$ is 57.8\%.
Therefore, measuring $H \rightarrow \bb$ is the most direct measurement to confirm
this theory of quark masses.
The diagram for this decay can be combined with the Feynman diagrams in
Figure~\ref{fig:associated-production}, Figure~\ref{fig:w-production} or \ref{fig:z-production},
and one of the decays in Figure~\ref{fig:v-decay} in order to generate the full
Feynman diagrams for the processes being measured in this analysis.
One such full diagram is shown in Figure~\ref{fig:two-lep-diagram}.
\begin{figure}
  \centering
  \begin{fmffile}{two_lep_diagram}
    \fmfframe(0,0)(0, 20){
    \begin{fmfgraph*}(250, 150)
      \fmfleft{i0,i1}
      \fmfright{o0,o1,o2,o3}
      \fmf{quark}{i1,v0,i0}
      \fmflabel{$\bar{f}$}{i0}
      \fmflabel{$f$}{i1}
      \fmf{boson, label=$Z$}{v0,v1,v2}
      \fmf{fermion}{o0,v2,o1}
      \fmflabel{$\ell^+$}{o0}
      \fmflabel{$\ell^-$}{o1}
      \fmf{dashes, label=$H$}{v1,v3}
      \fmf{fermion}{o2,v3,o3}
      \fmflabel{$\bar{b}$}{o2}
      \fmflabel{$b$}{o3}
    \end{fmfgraph*}
    }
  \end{fmffile}
  \caption[Full Feynman diagram for the two lepton process]{
    Above is the full Feynman diagram for $ZH \rightarrow \ell^+\ell^- \bb$.
  }
  \label{fig:two-lep-diagram}
\end{figure}
