\chapter{Theory} \label{ch:theory}

Before diving into the description of the experimental apparatus,
an explanation of why it is expected to work is needed.
There are many textbooks that cover the Standard Model, as there are many students who study it.
Much of what follows is taken from the book by Mark Thompson \cite{Thomson:1529540}.

The Standard Model Lagrangian can be defined as a sum of Lagrangians that each describe
the interactions between different fermions and bosons.
Equations of motion can be extracted from a Lagrangian $\mathcal{L}$
for a particle field $\phi_i$ using the Euler-Lagrange equations.

\begin{gather}
  \delta_\mu\left(\frac{\delta \mathcal{L}}{\delta(\delta_\mu \phi_i)}\right) -
  \frac{\delta \mathcal{L}}{\delta \phi_i} = 0
\end{gather}

In the measurement of $H\rightarrow \bb$ in associated production,
many components of the Standard Model are of interest.
These will be introduced as needed.
First, I will give a brief explanation of Higgs field's non-zero vacuum energy,
a trait that makes the Higgs one of the central keystones to Standard Model.
After that, the electro-weak Lagragian will be described since
the cross section of associated production depends on
the coupling of the Higgs Boson to the $W$ and $Z$ vector bosons.
The coupling of the electro-weak force to fermions is also important to
understand both the generation of these intermediate states and
the resulting final state that the CMS detector records.
Another important factor for this work is a the decay of the Higgs boson itself into bottom quarks.
This depends on the Higgs directly coupling to fermions.
Finally, we will briefly consider the part of the Lagrangian describing the strong force.
Since the LHC is a hadron collider,
understanding of the strong force is required to extract data from LHC collisions.

\section{The Higgs Mechanism}

In both of these components of the Standard Model Lagrangian,
the Higgs coupling actually gives the vector bosons and massive fermions their mass.
(In this work, neutrinos can be treated as massless.)
The granting of mass happens for two reasons:
the Higgs field has a non-zero vacuum expectation value,
and the Higgs field couples to vector boson and massive fermion fields.

The Higgs can be described as a scalar boson with a quartic potential.
The Lagrangian for a free Higgs is then

\begin{gather}
  \mathcal{L} = (\delta_\mu \phi)^* (\delta^\mu \phi) - (\mu^2(\phi^*\phi) + \lambda(\phi^*\phi)^2) \label{eq:free}
\end{gather}

The complex scalar field $\phi$ then has a minimum potential at

\begin{gather}
  \phi^*\phi = \frac{-\mu^2}{\lambda} = v^2 \label{eq:vacuum}
\end{gather}

This potential of the Higgs field breaks the U(1) symmetry of the Standard Model Lagrangian,
which would otherwise be invariant under transformations of
$\phi \rightarrow \phi' = e^{i\alpha}\phi$ for all particle fields.
Through this non-zero vacuum expectation value, the Higgs then has a constant influence
in other parts of the Standard Model Lagrangian.
For this measurement, three interactions that the Higgs makes
with this influence need to be considered:
the Higgs interacting with itself,
the Higgs interacting with the electroweak vector bosons,
and the Higgs interacting with quarks.

The first two interactions manifest in the Lagrangian when
we force the U(1) symmetry on the Lagrangian in Equation~\ref{eq:free}.
The derivatives must be replaced via $\delta_\mu \rightarrow D_\mu = \delta_\mu + igB_\mu$.
Adding the kinetic terms of the gauge field $B_\mu$, the Lagrangian now becomes

\begin{gather}
  \mathcal{L} = -\frac14 F^{\mu\nu} F_{\mu\nu} + (D_\mu \phi)^* (D^\mu \phi) - \mu^2 \phi^2 - \lambda \phi^4
\end{gather}

Expanding the scalar field $\phi$ around the vacuum expectation value and choosing the proper gauge

\begin{gather}
  \phi(x) = \frac1{\sqrt{2}} (v + h(x) + i\epsilon(x)) \\
  B_\mu(x) \rightarrow B'_\mu(x) = B_\mu(x) + \frac1{gv}\delta_\mu\epsilon(x)
\end{gather}

leads to the expansion

\begin{align}
  \mathcal{L} = & \frac12(\delta_\mu h)(\delta^\mu h) - \lambda v^2 h^2 - \frac14 F_{\mu\nu} F^{\mu\nu} + \frac12 g^2 v^2 B_\mu B^\mu \nonumber \\
                & + g^2 v B_\mu B^\mu h + \frac12 g^2 B_\mu B^\mu h^2 - \lambda v h^3 - \frac14 \lambda h^4
\end{align}


\section{Associated Production}

\begin{gather}
  \mathcal{L}_{EWK}
\end{gather}

\subsection{Production Mechanisms}

\subsection{Decay Channels of Vector Bosons}

\section{Characteristics of the Higgs}

\subsection{Energy Spectrum}

\subsection{Decay to \bb}

\begin{gather}
  \mathcal{L}_{f}
\end{gather}

\section{Other Relevant Standard Model Processes}
