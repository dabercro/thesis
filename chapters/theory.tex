\chapter{Theory} \label{ch:theory}

Before diving into the description of the experimental apparatus,
an explanation of why it is expected to work is needed.
There are many textbooks that cover the Standard Model, as there are many students who study it.
Much of what follows is taken from the book by Mark Thompson \cite{Thomson:1529540}.

The Standard Model Lagrangian can be defined as a sum of Lagrangians that each describe
the interactions between different fermions and bosons.
Equations of motion can be extracted from a Lagrangian $\mathcal{L}$
for a particle field $\phi_i$ using the Euler-Lagrange equations.

\begin{gather}
  \delta_\mu\left(\frac{\delta \mathcal{L}}{\delta(\delta_\mu \phi_i)}\right) -
  \frac{\delta \mathcal{L}}{\delta \phi_i} = 0
\end{gather}

In the measurement of $H\rightarrow \bb$ in associated production,
many components of the Standard Model are of interest.
These will be introduced as needed.
First, I will give a brief explanation of Higgs field's non-zero vacuum energy,
a trait that makes the Higgs one of the central keystones to Standard Model.
After that, the electroweak Lagrangian will be described since
the cross section of associated production depends on
the coupling of the Higgs Boson to the $W$ and $Z$ vector bosons.
The coupling of the electroweak force to fermions is also important to
understand both the generation of these intermediate states and
the resulting final state that the CMS detector records.
Another important factor for this work is a the decay of the Higgs boson itself into bottom quarks.
This depends on the Higgs directly coupling to fermions.
Finally, we will briefly consider the part of the Lagrangian describing the strong force.
Since the LHC is a hadron collider,
understanding of the strong force is required to extract data from LHC collisions.

\section{The Higgs Mechanism}

In both of these components of the Standard Model Lagrangian,
the Higgs coupling actually gives the vector bosons and massive fermions their mass
\cite{PhysRevLett.13.321, PhysRevLett.13.508, PhysRevLett.13.585}.
(In this work, neutrinos can be treated as massless.)
The granting of mass happens for two reasons:
the Higgs field has a non-zero vacuum expectation value,
and the Higgs field couples to vector boson and massive fermion fields.

The Higgs can be described as two complex scalar fields in a weak isospin doublet
with a quartic potential.
The Lagrangian for a free Higgs is then
\begin{gather}
  \mathcal{L} = (\delta_\mu \phi)^\dagger (\delta^\mu \phi) - (\mu^2(\phi^\dagger\phi) + \lambda(\phi^\dagger\phi)^2) \label{eq:free}
\end{gather}
Through the virial theorem, the potential has a minimum value when
\begin{gather}
  \phi^\dagger\phi = \frac{-\mu^2}{2\lambda} = \frac{v^2}{2} \label{eq:vacuum}
\end{gather}

This potential of the Higgs field breaks the SU(2) $\times$ U(1)
symmetry of the Standard Model Lagrangian.
Through this non-zero vacuum expectation value, the Higgs then has a constant influence
in other parts of the Standard Model Lagrangian.
For this measurement, three interactions that the Higgs makes
with this influence need to be considered:
the Higgs interacting with itself,
the Higgs interacting with the electroweak vector bosons,
and the Higgs interacting with quarks.

The first two interactions manifest in the Lagrangian when
we force the SU(2) $\times$ U(1) symmetry on the Lagrangian in Equation~(\ref{eq:free}).
The derivatives must be replaced.
\begin{gather}
  \delta_\mu \rightarrow D_\mu = \delta_\mu + i\frac{g_W}{2} {\boldsymbol \sigma} \cdot {\bf W_\mu} + ig'\frac{Y}{2}B_\mu
\end{gather}
$\phi$ can be rewritten to satisfy the vacuum expectation value in the gauge that will give us the massless neutral boson known as a photon.
\begin{gather}
  \phi(x) = \frac{1}{\sqrt{2}}
  \left(
  \begin{matrix}
    0 \\
    v + h(x)
  \end{matrix}
  \right) \label{eq:higgs-doublet}
\end{gather}
This leads to the following expansion for the kinetic term of the Lagrangian.
\begin{align}
  (D_\mu \phi)^\dagger(D^\mu \phi) = & \frac12 (\delta_\mu h)(\delta^\mu h)
  + \frac18 g_W^2 (W^{(1)}_\mu + iW^{(2)}_\mu)(W^{(1)\mu} - iW^{(2)\mu})(v + h)^2 \nonumber \\
  & + \frac18 (g_W W^{(3)}_\mu - g'B_\mu)(g_W W^{((3)\mu} - g' B^\mu)(v + h)^2 \label{eq:expanded}
\end{align}

Terms that are quadratic in terms of the gauge boson fields reveal the mass of the fields.
Taking $h(x) \rightarrow 0$, the terms for $W^{(1)}$ and $W^{(2)}$ are the just
\[
\frac14 g_W^2 v^2 W^{(1)}_\mu W^{(1)\mu} \quad \mathrm{and} \quad
\frac14 g_W^2 v^2 W^{(2)}_\mu W^{(2)\mu},
\]
giving the mass.
\begin{gather}
  m_W = \frac12 g_W v
\end{gather}
The quadratic terms for $W^{(3)}$ and $B$ mix to give a non-diagonal mass matrix ${\bf M}$.
\begin{gather}
  \frac{v^2}{8}
  \left(
  \begin{matrix}
  W^{(3)}_\mu & B_\mu
  \end{matrix}
  \right) {\bf M} \left(
  \begin{matrix}
  W^{(3)\mu} \\ B^\mu
  \end{matrix}
  \right) = 
  \frac{v^2}{8}
  \left(
  \begin{matrix}
  W^{(3)}_\mu & B_\mu
  \end{matrix}
  \right)
  \left(
  \begin{matrix}
    g_W^2 & -g_W g' \\
    -g_W g' & g'^2
  \end{matrix}
  \right)
  \left(
  \begin{matrix}
  W^{(3)\mu} \\ B^\mu
  \end{matrix}
  \right) \label{eq:mixing}
\end{gather}
The non-diagonal matrix allow $W^{(3)}$ and $B$ to mix.
Physical states must be represented by a diagonal Hamiltonian.
Diagonalizing the term above gives masses of the physical states.
\begin{gather}
  \frac18 v^2
  \left(
  \begin{matrix}
  A_\mu & Z_\mu
  \end{matrix}
  \right)
  \left(
  \begin{matrix}
    0 & 0 \\
    0 & g^2_W + g'^2
  \end{matrix}
  \right)
  \left(
  \begin{matrix}
  A^\mu \\ Z^\mu
  \end{matrix}
  \right) = 
  \frac12
  \left(
  \begin{matrix}
  A_\mu & Z_\mu
  \end{matrix}
  \right)
  \left(
  \begin{matrix}
    m_A^2 & 0 \\
    0 & m_Z^2
  \end{matrix}
  \right)
  \left(
  \begin{matrix}
  A^\mu \\ Z^\mu
  \end{matrix}
  \right) \label{eq:zmass}
\end{gather}
This gives us the masses of the neutral gauge bosons.
\begin{gather}
  m_A = 0 \quad \mathrm{and} \quad m_Z = \frac12 v \sqrt{g^2_W + g'^2}
\end{gather}

From the simple act of requiring SU(2) $\times$ U(1) symmetry on the Lagrangian of a scalar
doublet with non-zero vacuum expectation value,
the masses of all the electroweak gauge bosons have been produced.

\section{Associated Production}

The next thing to consider is the couplings also produced by this process.
The couplings will allow us to determine more precisely the parameters above
by measuring cross sections.

The physical states of $W^+$ and $W^-$ bosons can be written as
the raising and lowering operators for isospin.
\begin{gather}
  W^\pm = \frac{1}{\sqrt2}\left(W^{(1)} \mp i W^{(2)}\right) \label{eq:w-form}
\end{gather}
The second term of Equation~(\ref{eq:expanded}) can be further expanded.
\begin{gather}
  \frac14 g^2_W W^-_\mu W^{+\mu} (v + h)^2 = \frac14 g^2_W v^2 W^-_\mu W^{+\mu} +
  \frac12 g^2_W v W^-_\mu W^{+\mu} h + \frac14 g^2_W W^-_\mu W^{+\mu} h^2 \label{eq:wwh}
\end{gather}
The second term on the right hand side of Equation~(\ref{eq:wwh}) gives us the coupling
strength of a vertex with a Higgs and two $W$ bosons.
\begin{gather}
  g_{HWW} = \frac12 g^2_W v = g_W m_W
\end{gather}
The coupling to the $Z$ boson can also be found from
Equation~(\ref{eq:zmass}) by substituting $(v + h)^2$ back in for $v^2$ and
extracting the terms proportional to $h Z_\mu Z^\mu$.
\begin{gather}
  g_{HZZ} = \frac12 \left(g^2_W + g'^2\right) v = \sqrt{g^2_W + g'^2} m_Z
\end{gather}
When arranged in a way that the $W$ or $Z$ boson radiates the Higgs,
as opposed to a Higgs decaying into a pair of $W$ or $Z$ bosons,
the process is called associated production or \emph{Higgstrahlung}.
The vertex showing associated production is pictured in Figure~\ref{fig:associated-production}.
\begin{figure}
  \centering
  \begin{fmffile}{associated_production}
    \begin{fmfgraph*}(200, 120)
      \fmfleft{i0}
      \fmfright{o0,o1}
      \fmf{dashes, label=$H$}{v0,o1}
      \fmf{boson, label=$W/Z$}{i0,v0,o0}
    \end{fmfgraph*}
  \end{fmffile}
  \caption[Feynman diagram of associated production]
          {
            Above is the Feynman diagram for associated production.
            The $W$ or $Z$ boson radiates a Higgs boson.
            Both bosons later decay into particles detected by CMS.
          }
  \label{fig:associated-production}
\end{figure}


\subsection{Production Mechanisms of Vector Bosons} \label{sec:produce-vector}

The $W$ and $Z$ bosons are themselves intermediate states,
never existing in a directly observable manner.
They must be produced through interacts with stable fermions.
Since the LHC is a hadron collider,
considering the vector bosons' couplings with quarks would be most relevant.

Quarks are fermions that couple to each other through the strong force,
resulting from a SU(3) symmetry.
There are three generations of quarks each consisting of a pair of quark types.
Their mass eigenstates are denoted as down-type or up-type.
Table~\ref{tab:quarks} displays some of the characteristics of these quarks.
\begin{table}
  \centering
  \caption[Quark flavors]{The quarks and some stuff about them.}
  {\renewcommand{\arraystretch}{1.5}
  \begin{tabular}{|l|c c c c c|}
    \hline
    & 1st gen. & 2nd gen. & 3rd gen. & $Q$ & $I_W^{(3)}$ \\
    \hline
    down-type & $d$ & $s$ & $b$ & $-\frac13$ & $-\frac12$ \\
    up-type & $u$ & $c$ & $t$ & $+\frac23$ & $+\frac12$ \\
    \hline
  \end{tabular}}
  \label{tab:quarks}
\end{table}
A feature of quarks is that their mass eigenstates do not match their weak eigenstates.
There is a mixing among the down-type quarks that is parametrized by the
Cabibbo-Kobayashi-Maskawa (CKM) matrix.
\begin{gather}
  \left(
  \begin{matrix}
    d' \\
    s' \\
    b'
  \end{matrix}
  \right)
  =
  \left(
  \begin{matrix}
    V_{ud} & V_{us} & V_{ub} \\
    V_{cd} & V_{cs} & V_{cb} \\
    V_{td} & V_{ts} & V_{tb}
  \end{matrix}
  \right)
  \left(
  \begin{matrix}
    d \\
    s \\
    b
  \end{matrix}
  \right)
\end{gather}
The mass eigenstates are denoted as $d, s$, and $b$,
while $d', s'$, and $b'$ are the weak eigenstates.
This mixing allows quarks to change generations through interaction with $W^\pm$ bosons,
which raise or lower the weak isospin.
The following is the charge current vertex interaction.
\[
-i \frac{g_W}{\sqrt{2}}
\left(
\begin{matrix}
\bar{u} & \bar{c} & \bar{t}
\end{matrix}
\right)
\gamma^\mu \frac12 (1 - \gamma^5)
\left(
\begin{matrix}
  V_{ud} & V_{us} & V_{ub} \\
  V_{cd} & V_{cs} & V_{cb} \\
  V_{td} & V_{ts} & V_{tb}
\end{matrix}
\right)
\left(
\begin{matrix}
d \\ s \\ b
\end{matrix}
\right)
\]
The vertices for this interaction is shown in Figure~\ref{fig:w-production}
arranged in a way to show the processes of generating a $W^+$ or $W^-$ boson
from annihilating quarks.
\begin{figure}
  \centering
  \begin{fmffile}{w_plus_production}
    \begin{fmfgraph*}(180, 120)
      \fmfleft{i0,i1}
      \fmfright{o0}
      \fmf{quark}{i0,v0,i1}
      \fmf{boson, label=$W^+$}{v0,o0}
      \fmflabel{$u$}{i0}
      \fmflabel{$\bar{d}$}{i1}
      \fmflabel{$V_{ud}\frac{g_W}{2}$}{v0}
    \end{fmfgraph*}
  \end{fmffile}
  \begin{fmffile}{w_minus_production}
    \begin{fmfgraph*}(180, 120)
      \fmfleft{i0,i1}
      \fmfright{o0}
      \fmf{quark}{i1,v0,i0}
      \fmf{boson, label=$W^-$}{v0,o0}
      \fmflabel{$\bar{u}$}{i0}
      \fmflabel{$d$}{i1}
      \fmflabel{$V_{ud}\frac{g_W}{2}$}{v0}
    \end{fmfgraph*}
  \end{fmffile}
  \caption[Feynman diagram of generating $W^\pm$]
          {
            Above are diagrams for generating $W^+$ and $W^-$ bosons.
            the $u$ and $d$ quarks in the diagram can be replaced with
            any up-type or down-type quark, respectively.
            The CKM matrix element would in the vertex element would be changed accordingly.
          }
  \label{fig:w-production}
\end{figure}
The $\gamma$ matrices in the interaction are present because the SU(2) component of
the Standard Model only interacts with left-handed fermions and right-handed anti-fermions.
For this reason, the SU(2) component is more accurately labelled SU$(2)_L$.
From Equation~(\ref{eq:w-form}), the $W^\pm$ bosons are completely made up of the
$W^{(1)}$ and $W^{(2)}$ components of the SU$(2)_L$, so they also only interact with
left-handed fermions and right-handed anti-fermions.

Both the photon and the $Z$ boson mix the SU$(2)_L$ and U(1) components of the Standard Model.
Production of the $Z$ needs to be directly understood for this measurement,
but it is more straightforward to determine the strength of the $Z$ couplings to left-
and right-handed fermions by exploiting the symmetry of photon interactions.
That is, the photon interacts the same with left and right handed charged fermions,
and not at all with neutral fermions.
This is shown directly with experiments with leptons.
The charged leptons, electrons, muons, and taus, interact with photons,
while the respective neutrinos do not.
From the mixing in Equation~(\ref{eq:mixing}),
the photon and $Z$ fields can be expressed as the following.
\begin{gather}
  A_\mu = B_\mu \cos \theta_W + W_\mu^{(3)} \sin \theta_W \\
  Z_\mu = - B_\mu \sin \theta_W + W_\mu^{(3)} \cos \theta_W \label{eq:z-force}
\end{gather}
$\theta_W$ is known as the weak mixing angle.
The relative strengths of the $B$ and $W^{(3)}$ couplings
are determined directly through lepton electro-magnetic characteristics,
keeping in mind that $W^{(3)}$ only interacts with left handed particles.
The following are the electro-magnetic interaction strengths of left- and right-handed electrons and neutrinos.
\begin{align}
  e_L:& \qquad Qe = \frac12 g' Y_{e_L} \cos \theta_W - \frac12 g_W \sin \theta_W \\
  \nu_L:& \qquad \phantom{Q}0 = \frac12 g' Y_{\nu_L} \cos \theta_W - \frac12 g_W \sin \theta_W \\
  e_R:& \qquad Qe = \frac12 g' Y_{e_R} \cos \theta_W \\
  \nu_R:& \qquad \phantom{Q}0 = \frac12 g' Y_{\nu_R} \cos \theta_W
\end{align}
$Y_{e_L}$ and $Y_{\nu_L}$ must be equal to maintain SU$(2)_L$ symmetry.
To satisfy these contraints, the follow definition of $Y$ is needed.
\begin{gather}
  Y = 2\left(Q - I_W^{(3)}\right)
\end{gather}
The following relationship also arises from these experimental constraints.
\begin{gather}
  e = g_W \sin \theta_W = g' \cos \theta_W
\end{gather}

Returning to the $Z$ boson, from Equation~(\ref{eq:z-force}),
and defining
\begin{gather}
  g_Z = \frac{e}{\sin \theta_W \cos \theta_W},
\end{gather}
we have the following couplings to left- and right-handed fermions.
\begin{align}
  -\frac12 g' \sin \theta_W (Y_{f_L} \bar{u}_L \gamma^\mu u_L &+ Y_{f_R} \bar{u}_R \gamma^\mu u_R) + I_W^{(3)} g_W \cos \theta_W \left( \bar{u}_L \gamma^\mu u_L \right) = \nonumber \\
  & g_Z\left(\left( I^{(3)} - Q \sin^2 \theta_W \right)\bar{u}_L \gamma^\mu u_L
  - Q \sin^2 \theta_W \bar{u}_R \gamma^\mu u_R\right)
\end{align}
Now the coupling of the $Z$ to left- and right-handed quarks can be calculated from
Table~\ref{tab:quarks}, remembering that $I_W^{(3)}$ for right-handed fermions is 0.
Diagrams showing the interaction strengths of fermion-$Z$ vertices are shown
in Figure~\ref{fig:z-production}.
\begin{figure}
  \centering
  \begin{fmffile}{z_left_production}
    \begin{fmfgraph*}(150, 120)
      \fmfleft{i0,i1}
      \fmfright{o0}
      \fmf{quark}{i0,v0,i1}
      \fmf{boson, label=$Z$}{v0,o0}
      \fmflabel{$f_L$}{i0}
      \fmflabel{$\bar{f}_R$}{i1}
      \fmflabel{$g_Z\left( I^{(3)} - Q \sin^2 \theta_W \right)$}{v0}
    \end{fmfgraph*}
  \end{fmffile}
  \hspace{24pt}
  \begin{fmffile}{z_right_production}
    \begin{fmfgraph*}(150, 120)
      \fmfleft{i0,i1}
      \fmfright{o0}
      \fmf{quark}{i1,v0,i0}
      \fmf{boson, label=$Z$}{v0,o0}
      \fmflabel{$\bar{f}_L$}{i0}
      \fmflabel{$f_R$}{i1}
      \fmflabel{$- g_Z Q \sin^2 \theta_W$}{v0}
    \end{fmfgraph*}
  \end{fmffile}
  \vspace{6pt}
  \caption[Feynman diagram of generating $Z$]
          {
            Above are diagrams for generating $Z$ bosons.
            Left- and right-handed fermions are both coupled to,
            but with different coupling strengths.
          }
  \label{fig:z-production}
\end{figure}

Thus vector bosons couple to quarks, the constituents of hadrons,
which means they can be produced at the LHC.
As mentioned earlier in this section,
quarks interact through an SU(3) symmetry that results in the strong force.
The three states that this symmetry supports are known as color states,
and they are labelled red, green, and blue, or $r$, $g$, and $b$.
The resulting gauge bosons are known as gluons, and they carry the following color states.
\[
r\bar{g}, g\bar{r}, r\bar{b}, b\bar{r}, g\bar{b}, b\bar{g}, \frac{1}{\sqrt{2}}(r\bar{r} - g\bar{g}) \mathrm{and} \frac{1}{\sqrt{6}}(r\bar{r} + g\bar{g} - 2b\bar{b})
\]
At low energy, the coupling constant for the strong force is on the order of unity.
This leads to color confinement,
so that quarks an appreciable distance apart do not interact with each other.
To achieve this, all observable hadronic states are color singlets.
The most common hadronic states are mesons, made of a quark/anti-quark pair
with the color singlet state
\begin{gather}
  \psi(q\bar{q}) = \frac1{\sqrt{3}} (r\bar{r} + g\bar{g} + b\bar{b}),
\end{gather}
and baryons, made of three quarks with the following color singlet state.
\begin{gather}
  \psi(qqq) = \frac{1}{\sqrt{6}}(rgb - rbg + gbr  grb + brg - bgr) \label{eq:baryon-singlet}
\end{gather}
Baryons can also be composed of three anti-quarks,
which has a state corresponding to Equation~(\ref{eq:baryon-singlet}),
but with anti-color.

For this measurement, protons are collided at the LHC.
The proton consists of two $u$ quarks, and one $d$ quark.
Since the three quarks inside the proton interact strongly,
there are also many virtual gluons and quark/anti-quark pairs present at all times.
The quantity and energies of all these partons are not able to be calculated
since QCD is non-perturbative.
They can be measured in deep inelastic scattering experiements though.
In these, electrons are scattered off of protons,
and parton distribution functions (PDFs) can be measured.
The PDFs for protons are shown in Figure~\ref{fig:pdf}.
\begin{figure}
  \centering
  \includegraphics[width=0.7\linewidth]{figures/CTEQ6_parton_distribution_functions.png}
  \caption{The P.D.F.}
  \label{fig:pdf}
\end{figure}
% https://en.wikipedia.org/wiki/Parton_(particle_physics)#/media/File:CTEQ6_parton_distribution_functions.png

From these things, we can predict the cross section of generating $W$ and $Z$ bosons at the LHC.
These need to be very massive though, since they're going to radiate a Higgs.
This complicates the calculation a bit.

% \cite{PhysRevD.2.1285}

\subsection{Decay Channels of Vector Bosons} \label{sec:v-decay}

Due to the couplings described in Section~\ref{sec:produce-vector},
the vector bosons decay into quarks.
However, in the hadronic environment produced at the LHC
these are not the best indicators of a vector boson intermediate state.
This measurement uses leptonic decays in the final state
since the contributions of the vector boson intermediate states to leptonic final states
of appropriate kinematics are larger compared to other contributions to this final state.

There are three generations of leptons.
Each generation consists of a charged lepton, and a neutral neutrino.
The left-handed charged lepton and neutrino of each generation form an electroweak SU(2) doublet.
In order of increasing mass, the three generations are called electron, muon, and tau.
Heavier charged leptons decay into lighter leptons via the weak force.
Two neutrinos result from this decay, as shown in Figure~\ref{fig:tau-decay},
making the characteristics of the parent lepton's parent difficult to reconstruct.
The tau lepton has enough mass to consistently decay before reaching the CMS detector.
The tau lepton is also massive enough to also decay into quarks,
making its measurement even more complicated.
Muons have an average lifetime long enough to penetrate the entire detector,
and electrons are stable particles.
As a result, only muons and electrons are considered in this analysis.
\begin{figure}
  \centering
  \begin{fmffile}{tau_decay}
    \begin{fmfgraph*}(150, 120)
      \fmfleft{i0}
      \fmfright{o0,o1,o2}
      \fmf{fermion}{i0,v0,o0}
      \fmf{boson, label=$W^-$}{v0,v1}
      \fmf{fermion}{o1,v1,o2}
      \fmflabel{$\tau^-$}{i0}
      \fmflabel{$\nu_\tau$}{o0}
      \fmflabel{$\bar{\nu}_\mu$}{o1}
      \fmflabel{$\mu^-$}{o2}
    \end{fmfgraph*}
  \end{fmffile}
  \caption[Tau decay]{
    Heavier leptons can decay to lighter leptons while emitting two neutrinos.
    Above is an example of a decay of $\tau \rightarrow \nu_\tau\mu\bar{\nu}_\mu$.
    The neutrinos cannot be measured at CMS, so it is better to avoid such decays
    in the analysis.
  }
  \label{fig:tau-decay}
\end{figure}
The Feynman diagrams for the decay channels of interest are shown in Figure~\ref{fig:v-decay}.
\begin{figure}
  \centering
  \begin{fmffile}{z_zero_lep}
    \begin{fmfgraph*}(150, 120)
      \fmfleft{i0}
      \fmfright{o0,o1}
      \fmf{boson, label=$Z$}{i0,v0}
      \fmf{fermion}{o0,v0,o1}
      \fmflabel{$\bar{\nu}_e/\bar{\nu}_\mu$}{o0}
      \fmflabel{$\nu_e/\nu_\mu$}{o1}
    \end{fmfgraph*}
  \end{fmffile}
  \begin{fmffile}{z_two_lep}
    \begin{fmfgraph*}(150, 120)
      \fmfleft{i0}
      \fmfright{o0,o1}
      \fmf{boson, label=$Z$}{i0,v0}
      \fmf{fermion}{o0,v0,o1}
      \fmflabel{$e^+/\mu^+$}{o0}
      \fmflabel{$e^-/\mu^-$}{o1}
    \end{fmfgraph*}
  \end{fmffile}
  \begin{fmffile}{w_one_lep}
    \begin{fmfgraph*}(150, 120)
      \fmfleft{i0}
      \fmfright{o0,o1}
      \fmf{boson, label=$W^-$}{i0,v0}
      \fmf{fermion}{o0,v0,o1}
      \fmflabel{$\bar{\nu}_e/\bar{\nu}_\mu$}{o0}
      \fmflabel{$e^-/\mu^-$}{o1}
    \end{fmfgraph*}
  \end{fmffile}
  \begin{fmffile}{wp_one_lep}
    \begin{fmfgraph*}(150, 120)
      \fmfleft{i0}
      \fmfright{o0,o1}
      \fmf{boson, label=$W^+$}{i0,v0}
      \fmf{fermion}{o0,v0,o1}
      \fmflabel{$\nu_e/\nu_\mu$}{o0}
      \fmflabel{$e^+/\mu^+$}{o1}
    \end{fmfgraph*}
  \end{fmffile}
  \caption[Vector Boson decays in the analysis]{
    Above are the three different vector boson decays we are interested in.
  }
  \label{fig:v-decay}
\end{figure}


\section{Decay Channels of the Higgs}

What we are ultimately interested in measuring is the contribution of the
Higgs intermediate state to the final state of \bb.
Since the Higgs is a SU$(2)_L$ doublet of scalar fields,
the term $-g_f(\bar{L}\phi R + \bar{R} \phi^\dagger L)$ in the
Standard Model Lagrangian is invariant under SU$(2)_L \times$ SU$(1)_Y$,
where $L$ is a left-handed fermion doublet, and $R$ is a right-handed singlet.
If the Higgs doublet is expanded around the vacuum expectation value,
as Equation~(\ref{eq:higgs-doublet}), the Lagrangian term becomes the following.
\begin{gather}
  \mathcal{L}_f =
  -\frac{g_f}{\sqrt2} v \left(\bar{f}_L f_R + \bar{f}_R f_L \right)
  -\frac{g_f}{\sqrt2} h \left(\bar{f}_L f_R + \bar{f}_R f_L \right)
  \label{eq:fermion-mass}
\end{gather}
In Equation~(\ref{eq:fermion-mass}),
$f$ refers to the lower field of the fermion's SU$(2)_L$ doublet.
The Lagrangian also includes terms for the upper field since the conjugate of $\phi$
has the same symmetries as $\phi$.

The Lagrangian showing fermion-Higgs interactions in
Equation~(\ref{eq:fermion-mass}) consists of two terms.
Since $v$ is constant, the first term would be consistent with a fermion's mass,
assuming an appropriate coupling constant.
\begin{gather}
  g_f = \sqrt2 \frac{m_f}{v}
\end{gather}
The second term is the coupling of the fermion to the Higgs field
with the same coupling constant.
This is the mechanism by which the Higgs give fermions their masses,
and also why the Higgs couples more strongly to massive particles.
The Feynman rule for the interaction vertex between the Higgs and fermions
is proportional to the fermion's mass.
Of the quarks, the $b$ quark is the second most massive.
The most massive $t$ quark is too massive to be the final decay product of an on-shell Higgs.
In fact, the predicted branching ratio of $H \rightarrow \bb$ is 57.8\%.
Therefore, measuring $H \rightarrow \bb$ is the most direct measurement to confirm
this theory of quark masses.
The diagram for this decay can be combined with the Feynman diagrams in
Figure~\ref{fig:associated-production}, Figure~\ref{fig:w-production} or \ref{fig:z-production},
and one of the decays in Figure~\ref{fig:v-decay} in order to generate the full
Feynman diagrams for the processes being measured in this analysis.
One such full diagram is shown in Figure~\ref{fig:two-lep-diagram}.
\begin{figure}
  \centering
  \begin{fmffile}{two_lep_diagram}
    \begin{fmfgraph*}(250, 150)
      \fmfleft{i0,i1}
      \fmfright{o0,o1,o2,o3}
      \fmf{quark}{i1,v0,i0}
      \fmflabel{$\bar{f}$}{i0}
      \fmflabel{$f$}{i1}
      \fmf{boson, label=$Z$}{v0,v1,v2}
      \fmf{fermion}{o0,v2,o1}
      \fmflabel{$\ell^+$}{o0}
      \fmflabel{$\ell^-$}{o1}
      \fmf{dashes, label=$H$}{v1,v3}
      \fmf{fermion}{o2,v3,o3}
      \fmflabel{$\bar{b}$}{o2}
      \fmflabel{$b$}{o3}
    \end{fmfgraph*}
  \end{fmffile}
  \caption[Full Feynman diagram for the two lepton process]{
    Above is the full Feynman diagram for $ZH \rightarrow \ell^+\ell^- \bb$.
  }
  \label{fig:two-lep-diagram}
\end{figure}

\section{Other Relevant Standard Model Processes}

There are other processes that must be acknowledged in order to explain how the CMS detector works.
I'll fill those in as needed?
