\chapter{Conclusions} \label{ch:conclusion}

Thanks to the theory, technologies, and techniques described,
an inclusive cross section is measured that is 2.1 standard deviations below the Standard Model.
The binned momentum spectrum of the associated vector bosons
has a $p$-value of 9.3\% given the inclusive cross section.
These measurements confirm the Standard Model.

\section{Comparison with ATLAS}

After the discovery of $V\!Hb\bar{b}$, announced simultaneously by both CMS and ATLAS,
a joint effort produced the Simplified Template Cross Section (STXS)
framework for study of the differential cross section of the Higgs boson \cite{berger2019simplified}.
ATLAS published a first STXS measurement using only the first two years of the LHC's Run 2 data \cite{Aaboud_2019}.
This publication, as well as the first publication with all three years of Run 2 data \cite{Aad:2727500}
only look at the two higest vector boson transverse momentum bins of
$250 < p_T(V) < \SI{400}{GeV}$ and $\SI{400}{GeV} < p_T(V)$ for $W\!H$ and $Z\!H$.
Lower momentum results using all three years were later published by ATLAS separately \cite{Aad:2723187}.
With the result of this thesis, we can compare results between CMS and these two papers.
This comparison is shown in Table~\ref{tab:compare-atlas},
along with Standard Model predictions for each STXS bin \cite{de2016handbook}.
%
\begin{table}
  \centering
  \caption[Comparison with ATLAS STXS measurement]{
    Below are the STXS measurements compared between the ATLAS collaboration
    and this analysis, along with cross section predictions given by the Standard Model.
    Note, in the analysis published by ATLAS the $Z\!H$ events with $150 < p_T(V) < \SI{250}{GeV}$
    are not split between events with and without additional jets, as is done by CMS.
    The systematic uncertainty applied to the thesis result in that phase space assumes that
    the uncertainties for the jet multiplicity bins are uncorrelated.
    A dedicated analysis would have greater precision.
  }
  \renewcommand{\arraystretch}{1.5}
  \begin{tabular}{|c|c|c|c|c|}
    \hline
    Process & $p_T(V)$ & Prediction [fb] & ATLAS [fb] & This Work [fb] \\
    \hline
    $W(\ell\nu)H(b\bar{b})$ & $150 \text{---} \SI{250}{GeV}$ & $24.0 \pm 1.1$ & $19.0 \pm 12.1$ & $5.8 \pm 14.4$ \\
    $W(\ell\nu)H(b\bar{b})$ & $250 \text{---} \SI{400}{GeV}$ & $5.83 \pm 0.26$ & $3.3 \pm 4.7$ & $7.3 \pm 3.2$ \\
    $W(\ell\nu)H(b\bar{b})$ & $\SI{400}{GeV} \text{---} \infty$ & $1.25 \pm 0.06$ & $2.1 \pm 1.2$ & $2.9 \pm 1.0$ \\
    \hline
    $Z(\ell\ell/\nu\nu)H(b\bar{b})$ & $75 \text{---} \SI{150}{GeV}$ & $50.6 \pm 4.1$ & $42.5 \pm 35.9$ & $-12.6 \pm 30.9$ \\
    $Z(\ell\ell/\nu\nu)H(b\bar{b})$ & $150 \text{---} \SI{250}{GeV}$ & $18.8 \pm 2.4$ & $20.5 \pm 6.2$ & $-3.8 \pm 7.7$ \\
    $Z(\ell\ell/\nu\nu)H(b\bar{b})$ & $250 \text{---} \SI{400}{GeV}$ & $4.12 \pm 0.45$ & $1.4 \pm 3.0$ & $2.5 \pm 1.8$ \\
    $Z(\ell\ell/\nu\nu)H(b\bar{b})$ & $\SI{400}{GeV} \text{---} \infty$ & $0.72 \pm 0.05$ & $0.2 \pm 0.7$ & $0.78 \pm 0.48$ \\
    \hline
  \end{tabular}
  \label{tab:compare-atlas}
\end{table}
%
The uncertainty bands for each analysis overlap in every STXS bin,
except for  $Z\!H$ events with $150 < p_T(V) < \SI{250}{GeV}$.
The analysis presented in this thesis has better sensitivity than the ATLAS result in most STXS bins.
Though overall the two analyses have similar sensitivity.

A combination result of these two independent experiments would reduce overall uncertainties.
In particular, the statistical uncertainties would decrease as the amount of data available for analysis
increases by a factor of two compared to the data gathered by a single experiment.
There are also ways to decrease the systematic uncertainties applied to the result, and
future measurements from both CMS and ATLAS will improve the sensitivity of STXS measurements
as the amount of available data increases and as measurement technique is refined.

\section{The Future of HEP}

Study of the Higgs boson is far from finished.
Many more analyses of the Higgs boson will follow this one, just as many preceded it.
The first published searches of the Higgs boson in associated production by CMS
were given in 2011 \cite{CMS-PAS-HIG-11-012}.
As mentioned in the beginning of this document, the search concluded successfully in 2018.
This analysis itself used the techniques used to obtain that previous result as a starting point.
There were some adjustments to the method used before, and
researchers will use lessons learned from both studies for future measurements at CMS and beyond.

While it is impossible to predict the future precisely,
there are multiple projects underway with designs to continue to study the Higgs after CMS and ATLAS.
The LHC itself is receiving an upgrade, in a project titled the High Luminosity LHC (HL-LHC).
New technologies will be used to create a machine that will gather
an order of magnitude more data suitable for Higgs studies and
other searches beyond the Standard Model \cite{osti_1365580}.
In addition, there are plans for other new colliders.
For example, the Circular Electron Positron Collider (CEPC) is currently being designed
for use as a Higgs factory for precision studies \cite{thecepcstudygroup2018cepc}.
There are also plans for the Future Circular Collider (FCC),
which is foreseen to go through phases of providing electron-positron collisions,
and follow with hadronic collisions after an upgrade \cite{benedikt2020future},
much like the LHC used the same tunnels as LEP.

Over time, experiments on future colliders will achieve significantly more precise measurements
of the Higgs differential cross section,
giving scientists the potential to uncover smaller discrepancies from the Standard Model.
However, we do not know what other particles and interactions might exist beyond the Standard Model.
Our approach at CMS must be combined with results of other physics experiments using entirely different techniques.
For example, the LHCb collaboration recently measured a violation of lepton universality
at 3.1 standard deviations outside of the Standard Model through measuring $b$ hadron decays alone
\cite{lhcbcollaboration2021test}.
Of course, along with the excitement generated by the LHCb results are calls for
even more precise measurements of $b$ decays.
Additional understanding of this process may reveal much about the full $V\!Hb\bar{b}$ process as well.
We will likely pick up hints of deviation from the Standard Model, if they are there to be found,
in many different experiments as a new level of understanding is gained from each new result.
