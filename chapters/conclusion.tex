\chapter{Conclusions} \label{ch:conclusion}

As a result of the presented analysis,
an inclusive cross section is measured that is 2.1 standard deviations below the Standard Model.
The binned spectrum measured has a $p$-value of 9.3\% given the inclusive cross section.
These measurements do not contradict the Standard Model, though they do not conclusively verify it either.
In particular, $Z\!H$ generation is measured to be 2.9 standard deviations below the Standard Model,
and the measured momentum spectrum of the associated vector bosons has higher energy in both $Z\!H$ and $W\!H$.
Study of the Higgs boson, and even study of $H\rightarrow b\bar{b}$ in associated production, is not finished.
This analysis is only one of many to characterize the Higgs boson,
and this cross section measurement itself will be improved as more data is gathered
and measurement techniques are refined.

The first published searches of the Higgs in associated production by CMS
were given in 2011 \cite{CMS-PAS-HIG-11-012},
As mentioned in the beginning of this document, the search concluded successfully in 2018.
This analysis overlaps with the previous CMS result in only 50\% of the data points gathered in 2016 and 2017.
The difference is caused by the lower vector boson momentum values that were used in the discovery analysis,
made up for by looser $b$-tagging selection criteria in this analysis.
Researchers will use lessons learned from both studies to refine future measurements at CMS.

After the discovery, confirmed also by ATLAS,
a joint effort produced the STXS framework for additional study of the
Higgs boson \cite{berger2019simplified}.
ATLAS published a first STXS measurement using only the first two years of the LHC's Run 2 data \cite{Aaboud_2019}.
This publication, as well as the first publication with all three years of Run 2 data \cite{Aad:2727500}
only look at the two higest $p_T(V)$ bins of $250 < p_T(V) < \SI{400}{GeV}$ and $\SI{400}{GeV} < p_T(V)$
for $W\!H$ and $Z\!H$.
Lower momentum results using all three years were later published by ATLAS separately \cite{Aad:2723187}.
Comparison between the ATLAS results and the results of this thesis are shown in Table~\ref{tab:compare-atlas}.
%
\begin{table}
  \centering
  \caption[Comparison with ATLAS STXS measurement]{
    Below are the STXS measurements compared between the ATLAS collaboration,
    and this analysis done by CMS.
    Note, in the analysis published by ATLAS the $Z\!H$ events with $150 < p_T(V) < \SI{250}{GeV}$
    are not split between events with and without additional jets.
    The systematic uncertainty applied to the thesis result in that phase space assumes
    the uncertainties for the jet multiplicity bins are uncorrelated.
  }
  \renewcommand{\arraystretch}{1.5}
  \begin{tabular}{|c|c|c|c|c|}
    \hline
    Process & $p_T(V)$ & SM Prediction [fb] & ATLAS [fb] & Measured [fb] \\
    \hline
    $W(\ell\nu)H(b\bar{b})$ & $150 \text{---} \SI{250}{GeV}$ & $24.0 \pm 1.1$ & $19.0 \pm 12.1$ & $5.8 \pm 14.4$ \\
    $W(\ell\nu)H(b\bar{b})$ & $250 \text{---} \SI{400}{GeV}$ & $5.83 \pm 0.26$ & $3.3 \pm 4.7$ & $7.3 \pm 3.2$ \\
    $W(\ell\nu)H(b\bar{b})$ & $\SI{400}{GeV} \text{---} \infty$ & $1.25 \pm 0.06$ & $2.1 \pm 1.2$ & $2.9 \pm 1.0$ \\
    \hline
    $Z(\ell\ell/\nu\nu)H(b\bar{b})$ & $75 \text{---} \SI{150}{GeV}$ & $50.6 \pm 4.1$ & $42.5 \pm 35.9$ & $-12.6 \pm 30.9$ \\
    $Z(\ell\ell/\nu\nu)H(b\bar{b})$ & $150 \text{---} \SI{250}{GeV}$ & $18.8 \pm 2.4$ & $20.5 \pm 6.2$ & $-3.8 \pm 7.7$ \\
    $Z(\ell\ell/\nu\nu)H(b\bar{b})$ & $250 \text{---} \SI{400}{GeV}$ & $4.12 \pm 0.45$ & $1.4 \pm 3.0$ & $2.5 \pm 1.8$ \\
    $Z(\ell\ell/\nu\nu)H(b\bar{b})$ & $\SI{400}{GeV} \text{---} \infty$ & $0.72 \pm 0.05$ & $0.2 \pm 0.7$ & $0.78 \pm 0.48$ \\
    \hline
  \end{tabular}
  \label{tab:compare-atlas}
\end{table}
%
The uncertainty bands for each analysis overlap in every STXS bins,
except for  $Z\!H$ events with $150 < p_T(V) < \SI{250}{GeV}$.
The analysis presented here has better sensitivity than the ATLAS result in most STXS bins,
but overall the two analyses have similar sensitivity.

Standard Model confirmation of this independent experiment also immediately serves to
reduce any uncertainties from previous analyses. \phil{Can you rewrite
  this sentence, I find it very confusing}
Though the binning and results of this analysis may appear rudimentary
in the context of what is required to dispute the Standard Model,
the development of the techniques is crucial for studies to
come. 

While it is impossible to predict the future precisely,
there are multiple large projects underway with designs to continue to study the Higgs.
The LHC is receiving an upgrade, in what is called the High Luminosity LHC (HL-LHC).
All cutting-edge technologies will be used to create a machine that will gather
an order of magnitude more data suitable for Higgs studies and
other searches beyond the Standard Model \cite{osti_1365580}.
Beyond that, the Circular Electron Positron Collider (CEPC) is currently being designed
for use as a Higgs factory for precision studies
\cite{thecepcstudygroup2018cepc}, as well has the Future Circular Collider (FCC),
which will go through phases of colliding electrons and hadrons \cite{benedikt2020future}.
These experiments will produce very fine-grained descriptions of
the Higgs differential cross section, and have the potential to uncover
discrepancies from the Standard Model.
They may allow us to understand just a little bit more about our universe,
along with any other exciting results from physics experiments looking elsewhere.

\phil{Can you rewrite the last paragraph. The words are a little
  jumbled, but the message is nice. It would be good to end your
  thesis on a high note!}
