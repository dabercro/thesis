\chapter{Introduction} \label{ch:intro}

One of the most curious features of physics at small scales, which will likely frustrate students for the rest of time,
is that certain events are not deterministic and only have a probability of happening.
There is no guarantee that an electron and a positron approaching each other at high energies will annihilate and produce a muon and an anti-muon.
However, this event might still occur at a later time at with the exact same initial conditions.
Furthermore, the observation of resonances, where this is more likely to happen when the electron and positron approach each other at particular speeds,
does not mean that a $Z$ boson was present in a given interaction.
It just means that the weak component of the electroweak force significantly increases the probability of the muonic final state, given the total energy of the initial state.
The sum of probabilities from different possible field interactions with particular initial conditions is the only thing we can measure.
This is also only possible when observing many events with the same initial conditions.

This point is difficult to convey concisely, so many laypeople, as well as some practicing physicists, are confused by the terminology adopted by the field.
But this distinction is relevant to the topic of this work.
This document presents a measurement of a cross section.
Cross section is the name given to the probability of an interaction occurring.
Reported cross sections can be split up to describe different contributions to final states,
and they can be collated into what are called ``production cross sections'' which describe
the probabilities of particular intermediate states ``occurring'' (even though intermediate states never exist in reality).

The main point is that if there exists some interesting particle, and it interacts with other particles,
you can see an increased probability of certain initial states resulting in certain final states.
This can teach the observer about the role of the interesting particle, without ever directly seeing it.

\section{Measurement of the Higgs Cross Section}

The purpose of the following document is to present the methods and results of measuring the strength of the coupling between the Higgs boson and bottom quarks.
In this context, the Higgs boson makes up one of the previously mentioned intermediate states that cannot be shown as present in a given event.
The cross section measurement relies on a number of physics processes that will be accounted for in this document.

To measure this coupling, the Higgs boson must first be ``produced'' before measuring its coupling strength to bottom quarks.
Since we are not technologically advanced enough to achieve this generation using bottom quarks directly,
we use measured Higgs generation rates from normal constituents of protons.
We constrain ourselves further by requiring that the Higgs is generated by associated production.

After the Higgs is generated, it can decay into a number of different particles.
This work is only concerned with one kind of decay.

The math that this all relies on is presented in Chapter~\ref{ch:theory}.

\section{Motivation for the Measurement}

This is Thomas Kuhn's ``normal science''.

Precision measurements are needed to be certain of what we think.

Precision measurements often lead to discrepancies that are explained by a fundamental shift in the model.

The Standard Model is a good one.
It will not be fully replaced, but at worst expanded upon.
Just like Newton's Laws are still a reasonable approximation for General Relativity, The Standard Model is a good approximation for most things we have been able to interact with so far.

The only lingering questions are Dark Matter and Dark Energy,
but there is no reason to assume that precise measurements of known phenomena will not lead to an explanation.

\section{Historic Context}

First, we have The Standard Model.

Parts were proven correct by the observation of the weak bosons.

The Higgs was observed in 2013.

The Higgs decaying to \bb\! was observed in 2018.
\cite{obs-18, Aaboud_2018}

\section{Using the CMS Detector}

The CMS detector is a general purpose detector used to make many observations of conditions
unattainable on Earth outside of the LHC.

It has many stationary parts, and a couple of moving ones too.

This device is described in detail in Chapter~\ref{ch:detector}.
