\chapter{Introduction} \label{ch:intro}

One of the most curious features of physics at small scales,
which will likely frustrate students for the rest of time,
is that certain events are not deterministic and only have a probability of happening.
There is no guarantee that the initial state of an electron and a positron approaching
each other will annihilate and produce the final state of a muon and an anti-muon,
even if their energies in the center-of-mass frame are adequate for muon production.
However, trying long enough with the same initial conditions will eventually
produce a muon and anti-muon pair.
Furthermore, the observation of resonances, where this is more likely to happen
when the electron and positron approach each other at particular energies,
does not mean that a $Z$ boson was present in a given interaction.
It just means that the weak component of the electroweak force significantly increases
the probability of the muonic final state,
when the total energy of the system is close to the mass of the $Z$ boson.
The sum of probabilities from different possible field interactions leading from a specific
initial condition to a particular final state is the only thing that can be measured.
This measurement is only done accurately
when observing many events with the same initial conditions.

This point is difficult to convey concisely, so many laypeople,
as well as some practicing physicists, are confused by the terminology adopted by the field.
But this distinction is relevant to the topic of this work.
This document presents a measurement of a cross section.
Cross section is the name given to the probability of an interaction occurring.
Reported cross sections can be split up to describe different contributions to final states,
and they can be collated into what are called ``production cross sections'' which describe
the probabilities of particular intermediate states ``occurring''
(even though intermediate states never exist in reality).
In the example above, the $Z$ boson would be a possible intermediate state.

The main point of the example above is that physicists can learn about the $Z$ boson
by observing only electrons and muons.
They never see the $Z$ boson itself.
This can be generalized to any interesting particle that interacts with other particles.
An increased probability of certain initial states resulting in certain final states
can teach the observer about the role of the interesting particle
without ever directly seeing it.

\section{Measurement of the Higgs Cross Section}

The purpose of the following document is to present the methods and results
of measuring the strength of the coupling between the Higgs boson and bottom quarks.
In this context, the Higgs boson makes up one of the previously mentioned
intermediate states that cannot be shown as present in a given event.
The coupling strength is directly related to the probability contribution
the Higgs field has on processes involving bottom quarks.
However, the cross section measurement also relies on a number of other physics processes.

To measure this coupling, the Higgs boson must first be ``produced''
before measuring its coupling strength to bottom quarks.
This analysis takes advantage of a process known as associated production,
where a vector boson, one of the intermediate particles of the weak nuclear force,
couples to and radiates a Higgs boson.
The vector boson is in turn produced by the collision of high energy protons.

The measurement is not complete once the intermediate state is generated.
The Higgs boson can decay into a number of different particles,
with bottom quarks being only one type.
The bottom quarks must be identified.
The vector bosons have multiple decay modes as well.
In this analysis, we only use the leptonic decays because
these give us the cleanest signature,
where enough of the contribution to the final state probability
is from associated production for it to be measured.

There are also other physics processes that create the same final states,
as well as processes that create final states that look similar enough
to be practically indistinguishable.
These processes must also be well understood before a Higgs boson cross section measurement
can be undertaken.

\section{Motivation for the Measurement}

The measurement of a cross section of a known particle is ``normal science,''
and that is the space in which this analysis operates \cite{Kuhn:1970}.
Much of the community of physics researchers have been operating under the paradigm
of the Standard Model for the better part of a century.
The Standard Model has known gaps,
such as missing explanations for neutrino mass,
as well as the origin of Dark Matter and Dark Energy.
However, none of these research fields have yielded any results
that will trigger a paradigm shift.
In fact, the most exciting discoveries of new particles,
such as the weak $W$ \cite{PhysRevLett.50.1738} and $Z$ \cite{dau1983ua1}
bosons and the Higgs boson \cite{Chatrchyan_2012,PhysRevD.86.032003} only confirmed
predictions by the Standard Model.

In the meanwhile, precision measurements are performed on
processes that we expect to already understand very well.
Repeating measurements while the state of the art is improving is interesting,
no matter the outcome.
Over time, the uncertainty in the measurement outcome shrinks,
leading to more precise knowledge of parameters of the Standard Model.
If the precise parameters cause excessive tension in that they cannot exist
assuming the Standard Model is true,
the discrepancies would need to be explained by a different or amended model.

The Higgs decaying to $b\bar{b}$, or a bottom quark and bottom anti-quark,
was observed in 2018 \cite{obs-18, Aaboud_2018}.
The measurement outlined in this document goes further in that
it measures the contribution of $H \rightarrow b\bar{b}$ in associated production
to final states possessing different energies.
This is called a differential cross section,
and places greater constraints on the parameters of the Standard Model.
These constraints lead to more precise measurements of the parameters,
and have the potential to discover discrepancies that have hitherto been missed.

If descrepancies arise, not only in the frequency of events,
but also the energy spectrum of the events,
that means there are interactions that are not accounted for.
The Standard Model describes all possible interactions between the particles we know of.
If there is evidence of additional interactions, then additional particles must exist to allow them.
Alternatively, the Higgs boson itself might not be the type of boson we expect it to be.

\section{Using the CMS Detector at the LHC}

This measurement is only possible due to massive efforts by the scientific community.
The Higgs boson is not typically generated in conditions on Earth.
There have been multiple colliders created over the years that attempted to find evidence of the Higgs boson.
The highest energy hadron collider before the LHC,
the Tevatron at Fermilab reached collision energies of nearly \SI{2}{TeV}
and ran for two decades \cite{Holmes_2011}, but was unable to discover the Higgs.
At CERN, the Large Electron-Positron (LEP) collider \cite{Myers:226776} ran at energies up to \SI{200}{GeV}.
In principle, this was enough energy to generate Higgs bosons,
but the collider was decommissioned with only hints of the Higgs boson at LEP.
The Large Hadron Collider (LHC) was built in the same tunnels as LEP, and
required the efforts of thousands of scientists and engineers,
as well as the funding from countries distributed all around the globe.
The LHC performs proton-proton collisions at \SI{13}{TeV},
which we now know is more than enough energy to produce Higgs bosons.

To observe the final states of collisions at the LHC,
multiple detectors have also been constructed,
due to the efforts of hundreds or thousands of individuals.
This analysis is done using data from the Compact Muon Solenoid (CMS).
Other experiments are ATLAS, ALICE, LHCb, TOTEM, LHCf, MoDEL, and FASER.
The CMS detector is a general purpose detector,
which was used in the discovery of the Higgs boson, along with ATLAS.
The detection capabilities of CMS make it instrumental in a number of state of the art
measurements of the Standard Model as well as in searches for
physics beyond the Standard Model.
