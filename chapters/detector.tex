\chapter{The CMS Detector} \label{ch:detector}

The Compact Muon Solenoid (CMS) detector, located at the LHC,
consists of multiple sub-detectors.
Ideally, time could be saved for the dedicated reader by only detailing
the parts of the detector that are relevant for the analysis presented in this work.
Unfortunately, a characteristic of hadron collider measurements is that
almost all sub-systems of the detector are used to make the final measurement.

A brief overview of all the detector subsystems are therefore presented in this chapter.
More can be learned about the design and motivations for the detector in the TDR \cite{cms-tdr}.
Information presented on the physical CMS design parameters are
taken directly from that document unless otherwise noted.

\section{Associated Production at the LHC}

The CMS detector only observes events.
Before describing the devices that are used to observe and record events,
the method of generating interesting events must be described.
The CMS detector is located at the Large Hadron Collider (LHC).
Described in detail in multiple publications \cite{Evans_2008},
a brief description is given here.

The LHC, with a circumference of 26.7 km,
is large enough to be considered located in multiple towns and countries,
but it will suffice to say it is near Geneva, Switzerland
at the European Organization for Nuclear Research (CERN),
the main campus of which is addressed in Meyrin, Switzerland.
This campus itself also spans the border between Switzerland and France.
This large circumference is needed since charged particles traveling
in a circular path with radius $r$ emit synchrotron radiation at the following rate.
\begin{gather}
  P = \frac{q^2 p^4}{6\pi \epsilon_0 m^4 c^5 r^2}
\end{gather}
The amount of power lost by the particles decreases quadratically
with the size of the collider.
In addition, the energy lost decreases with the mass of the accelerated particles
to the fourth power.
The LHC was built in the same tunnels that were used for LEP,
which was a collider for electrons and positrons that took much of its data at
$\sqrt{s} = \SI{98}{GeV}$.
The resulting LHC is designed to collide protons at energies of $\sqrt{s} = \SI{14}{TeV}$,
with the data for this analysis taken at \SI{13}{TeV}.
Though Figure~\ref{fig:pdf} showed that not all of the energy from each proton
goes into the interaction,
there still is adequate phase space available to generate the massive off-shell
vector bosons that are needed for \emph{Higgstrahlung},
via the mechanisms described in Section~\ref{sec:produce-vector}.

The removal of a quark from one proton and an anti-quark from the other leads to two
non-color-singlet states in close proximity to each other.
The resulting spray of hadronic particles generated from the vacuum to restore
color singlets are called jets.
The detectors are designed to distinguish these jets from more interesting decay products
in the interaction.

The luminosity of the LHC is given by the following formula.
\begin{gather}
  \mathcal{L} = \frac{N^2_B f_{\text{rev}} k_B}{4\pi B^* \epsilon_{xy}} \times F
\end{gather}
$N_B$ is the number of protons per bunch,
$f_{\text{rev}}$ is the frequency of beam revolutions,
$k_B$ is the number of bunches per beam,
$B^*$ and $\epsilon_{xy}$ describe the goodness of the beam,
and $F$ is a geometric collision factor.
\begin{gather}
  F = \frac{1}{\sqrt{1 + \frac{(\sigma_s \tan \phi)^2}{\epsilon_{xy} \beta^*}}}
\end{gather}
$\sigma_s$ is the length of each bunch,
and $\phi$ is the crossing angle.
In order to generate a large amount of data needed for measurements like associated production,
the LHC operates at a high frequency of collisions, and generates many simultaneous collisions.
For Run 2, there is a proton bunch crossing every \SI{25}{ns}.
The CMS detector must be able to read out and process data on that timescale.
Each proton bunch includes over 100 billion protons \cite{Bruce:2201447}.

\section{Detector Requirements}

One configuration of possible final state particles is shown in Figure~\ref{fig:two-lep-diagram}.
There, two oppositely charged leptons and two b quarks are the end decay products.
The $b$ quarks also hadronize form color singlets well before reaching the detector,
but the resulting jets can actually be distinguished well from the jets resulting
from the fragmenting protons.

Hadrons containing $b$ quarks decay through the weak force since they require a flavor change.
As mentioned before, the CKM matrix in Equation~\ref{eq:ckm} quantifies the mixing between the different quark flavors.
The value of $V_{tb}$ is close to unity, and since the CKM matrix is unitary,
$V_{cb}$ and $V_{ub}$ are small.
This means the matrix element weak decays of the $b$ hadrons is small.
This is the only decay channel available to the lightest $b$ hadrons,
so their lifetimes are relatively long.
The delayed decay results in a jet with a secondary vertex where many of its particles
are generated from the vacuum at a distance from the initial collision point.

Alternate signatures of interest can be seen by substituting other vector boson final states
from Figure~\ref{fig:v-decay}.
In these, there may be one or zero charged leptons,
with one or two neutral leptons, respectively.
Neutral particles are difficult to detect,
with neutral leptons being capable of passing through the entire Earth without
being part of a detectable interaction.
The CMS detector therefore ignores the neutrinos,
but their presence can still be inferred.
Even with the variation in momentum along the beam direction,
all partons in each proton have approximately zero momentum in the transverse direction.
Therefore, the sum of the transverse momenta of all final state particles must also be zero.
Many events in CMS have an overall imbalance in the transverse plane.
This imbalance is labelled Missing Transverse Energy, $E^\textrm{miss}_T$, or MET.
Large MET in an event is often a sign of high energy neutrinos that the detector cannot detect.

We need to identify all of these interesting particles,
as well as be able to reconstruct missing transverse momentum.
In addition, the additional hadronic activity in the event, called pileup, must be mitigated.
The energy of the decay products have energies on the scale of the masses of the parent particles.
The detector must be capable of measuring jets and leptons with energies on the
order of 10s or 100s of GeV.
Better energy resolution for each of these decay products allows better separation
of our signal process from background processes that generate very similar final states.

\section{Detector Design}

The CMS detector as a whole has cylindrical symmetry around the beam access.
It is 21 meters long and 15 meters in diameter.
There are gaps at either end to allow the beam,
but otherwise tries to cover the full solid angle around the collision point.
The azimuthal angle of a particle relative to the beam axis is described
by pseudorapidity, $\eta$.
\begin{gather}
  \eta = -\ln \left[\tan \left( \frac \theta 2 \right) \right]
\end{gather}
The barrel portion of CMS can detect particles with $|\eta| < 2.4$,
while the forward caps of the detector can reach $|\eta| < 5.0$.

Different technologies are better for measuring the energy
or other kinematics variables of different particles.
As a result, the CMS detector is made up of different sub-detector systems,
arranged in cylindrical layers.
Each layer consists of a ``barrel'' portion and two endcaps on either side.

The innermost layer is designed to extrapolate the tracks of charged particles
back to their point of origin.
This is called the Silicon Pixel Detector.
The next layer is designed to measure the energies of photons and electrons.
The third layer measures the energies of both charged and neutral hadrons.
Outside of these three layers is a superconducting solenoid,
which generates a magnetic field for the entire detector.
On the very outside of the detector are gas chambers designed to detect muons
interspersed with the iron return yoke for the solenoid.
A slice of the CMS detector showing the relative positions of each layer
is shown in Figure~\ref{fig:slice}.
\begin{figure}
  \centering
  \includegraphics[width=0.9\linewidth]{figures/CMSslice_whiteBackground.png}
  \caption[CMS detector slice]{
    A slice of the CMS detector is shown above \cite{Barney:2120661}.
    The four detector layers are labelled and show the penetration
    depths of various particles stable enough to travel a measureable distance.
    }
  \label{fig:slice}
\end{figure}

The magnet is described first since the magnetic field it produces is a key
part of most of the rest of the detector.
After that, the sub-detectors are summarized in the order of closest to farthest
from the beamline, since this is the order that particles would interact with the layers.
Each sub-detector section also describes the measured performance during Run 2.
Note that this measurement is an iterative process that depends on the event reconstruction
described in Section~\ref{sec:event-reco},
which in turn depends on the performance of the entire detector.
The performance numbers are presented with each sub-detector design though so that
it is immediately clear how effective each design has been.

\subsection{Solenoid Magnet}

Part of the CMS acronym acknowledges the role of the solenoid magnet.
The presence of a magnetic field is necessary for accurate measurements
of charged particles passing through the silicon pixel detector and muon chambers.

The magnetic field generated is designed to cause the path of
a muon with \SI{1}{TeV} of energy to bend enough to have a momentum resolution of 10\%.
Inside the solenoid, the magnetic field reaches \SI{4}{T},
with a return field that is large enough to cause muon tracks to curve throughout
the muon chambers outside the magnet.

A super-conducting solenoid enables the creation of a magnetic field
with the required strength.
A current of \SI{19.5}{kA} is sent through 2168 turns over \SI{12.9}{m}.
The magnetic field stores \SI{2.7}{GJ} of energy.
In order to hold this, the structural components holding the magnet and
the detector in place are strong enough to withstand \SI{64}{atm} of hoop pressure.

\subsection{Silicon Pixel Detector}

The layer closest to the beamline is designed to obtain a precise track pointing
to the orgin of particles passing into the detector.
It is made up of layers of many small pixels to do this.
As distance from the interaction point increases, the pixel size also increases.

The innermost three layers, with the closest layer being a distance of $r=\SI{4}{cm}$ from the interaction point,
are made of hybrid pixel detectors.
Each pixel has dimensions of $100 \times \SI{150}{\micro m}$.
At this size, only one out of every ten thousand inner layer pixels is triggered in a typical LHC bunch crossing.
Outside of the pixel detector layers,
silicon strip detectors are used.
These are placed in the region that is $20 < r < \SI{55}{cm}$ from the beamline.
Strip dimensions give a cell size of approximately $\SI{10}{cm} \times \SI{80}{\micro m}$.
2--3\% of cells are activiated during a typical bunch crossing.
The outermost layers are made of larger strips with cell sizes of $\SI{25}{cm} \times \SI{180}{\micro m}$.
About 1\% of these pixels are triggered each bunch crossing.

The active material of the silicon pixel detector is semi-conducting silicon.
When charged particles pass through, electron-hole pairs are generated and drift apart due to a bias voltage.
The voltage change when these pairs reach their respective electrodes indicates a charged particle passed through.
Because of this, the silicon pixel detector cannot detect any neutral particles,
but it gives a point of origin for charged particles that is accurate enough to identify pileup.

In the beginning of 2017, the pixel detector was upgraded to handle the higher radiation environment of Run 2 \cite{Dominguez:1481838}.
A layer was added to both the barrel and endcap sections of the pixel detector.
Firmware was also upgraded to keep the pixel detector operating at a frequency higher than the Run 2 collision frequency.
With this upgrade, the detector operated with a 97\% hit efficiency for all layers at the highest instantaneous luminosity.
Layers beyond the first performed with greater than 99\% hit efficiency \cite{Modak:2712284}.

\subsection{Electromagnetic Calorimeter} \label{sec:ecal}

The next layer of the detector is called the Electromagnetic Calorimeter or ECAL.
This layer is designed to fully capture and measure the energy of photons and electrons.
The ECAL is made of crystals of the scintillating material Lead Tungstate (\ch{PbWO4}).
Each crystal is placed in the detector so that its smallest face is facing the collision point.
These small faces have dimensions of $22\times\SI{22}{mm}$.
The length of each crystal is \SI{230}{mm},
and the far face is slightly larger at $26\times\SI{26}{mm}$.
\ch{PbWO4} has a radiation length of $\chi_0 = \SI{8.9}{mm}$ and a Moliere radius of \SI{21}{mm}.
This means each crystal is 25.8 radiation lengths, containing the full shower within the ECAL,
and each shower is also localized to within one crystal from the initial ionization.

The scintillating properties of \ch{PbWO4} are also desireable for observing LHC collisions.
4.5 photons for every MeV of deposited energy are ultimately detected by the photodiodes at the far end of the crystals.
This is a low number for most experiments, but the only photons and electrons of interest in this measurement deposit at least 10s of GeV of energy.
This gives the ECAL energy resolutions in the range of $5-10\%$.
More importantly, the scintillation is very fast.
80\% of the light from an interaction is emitted within the \SI{25}{ns} between bunch crossings,
making it easy to associate the readouts with the appropriate collision.

During the operation of the CMS, the ECAL crystals are damaged by radiation.
Some recovery occurs as the crystal structure falls back into the ground state,
but over time, the performance of the crystals degrade.
That degradation happens at different rates in different areas of the detector.
Lasers are used to calibrate the ECAL online during the gaps between beams \cite{Monti:2653861}.
Resolution is measured by looking at $Z\rightarrow ee$ events.
For Run 2, the barrel region of the ECAL performed with 1.6\% resolution, and the other regions had a 5\% resolution \cite{Bartosik:2712238}.

\subsection{Hadronic Calorimeter}

The Hadronic Calorimeter (HCAL) has the same goal as the ECAL,
where it contains particles and measures the energy emitted by them.
However, it tries to do this for hadrons, such as protons, neutrons, and stable mesons.
Since hadrons are much more massive than electrons,
the ionizing collisions in a typical scintillator does not slow them down enough to contain them.
Instead, they must interact via nuclear collisions to be attentuated.
CMS uses brass for its HCAL due to its relatively short interaction length,
the fact that it is non-magnetic, and it affordability.

The barrel of the HCAL is jacketed in stainless steel for structural support.
This layer is \SI{61}{mm} on the layer immediately next to the ECAL and \SI{75}{mm} thick on the outer edge.
The inside of the HCAL consists of brass absorber plates interspersed with plastic scintillator tiles.
The layers closer to the beamline alterate \SI{50.5}{mm} brass plates with \SI{3.7}{mm} scintillator plates.
Farther away, the brass plates are instead \SI{56.5}{mm} thick.
Wavelength shifting fibers are run through the scintillator tiles
to allow photons to travel to the outside of the HCAL where they are detected by photodiodes.

The HCAL performance also degrades as it is exposed to radiation, like the ECAL in Section~\ref{sec:ecal}.
The calibration for HCAL is performed using an embedded radioactive source,
lasers and LEDs, and an \emph{in situ} calibration using assumed symmetry in $\phi$.
With these methods, a response within 3.4\% was maintained in the HCAL barrel
and within 2.6\% in the HCAL endcap \cite{Chadeeva_2018}}.

\subsection{Muon Chambers}

Muons are the most penetrative particles that CMS attempts to detect.
They are heavier than electrons, so they are not stopped in the ECAL.
They do not interact via the strong nuclear force,
so the high density of the HCAL also does not cause measurable interactions.
Instead of stopping and measuring muons in calorimeters,
CMS tracks their trajectory with both the pixel detectors on the inside of the detector
and the muon chambers that make up the outer layer of the detector.

This is the only subdetector system outside of the solenoid,
but the returning magnetic field is still present,
allowing the momentum of the muons to be extracted from the curvature of their trajectory.
Layers of muon chambers act much like the pixel detectors, but at a larger and more distant scale.
The muon chambers in the barrel region of $|\eta| < 1.2$ consist of drift tube chambers.
In the endcaps, cathode strip chambers are used.
The difference is to account for higher neutron backgrounds in the endcap, as well as a greater magnetic field.
In both regions, resistive plate chambers are spaced between the layers of the other muon chambers.

The muon detectors were triggered above 90\% of the time a muon passed through.
The overall efficiency of the muon trigger, which relies heavily on the muon system
and is described in more detail in Section~\ref{sec:trigger},
increases as a function of muon $p_T$ and plateaus around 90\% efficiency.
The timing of the muon system assigns 1\% of muons to the wrong event \cite{Pozzobon_2019}.

\section{Event Reconstruction} \label{sec:event-reco}

Event reconstruction is done using a combination of tracks and calorimeter deposits.
These are then linked across the different detector components to identify particles.

\cite{Sirunyan_2017}

\section{Trigger} \label{sec:trigger}

Collisions happen every \SI{25}{ns}, corresponding to a frequency of \SI{400}{MHz}.
The amount of data that the detector generates at each collision is
too large to store all of it at this frequency.
Luckily, most collisions result in uninteresting data.
A trigger system is used to identify interesting events
and reduce the frequency of event writing to \SI{1}{kHz}.
This is done using two stages.
The Level-1 (L1) trigger passes events with a frequency of \SI{100}{kHz},
and the High Level Trigger (HLT) picks from the remaining events
with a frequency of \SI{1}{kHz} \cite{Tosi:2290106}.

The L1 trigger is implemented in hardware.
It was upgraded for Run 2 of the LHC to run on FPGAs on an Advanced Mezzanine Card.
There are two main components of the L1 trigger.
One considers calorimeter deposits, and the other examines the muon chambers.
The overall L1 trigger fires when there are high energy, resolved calorimeter hits
or if a possible muon is reconstructed.
Due to the flexibility of FPGAs,
the exact conditions of the firing are configurable \cite{Cadamuro_2017}.

After the L1 is fired, the data is sent to the High-Level Trigger (HLT),
which is a computing farm that makes a final decision on whether or not to save the data
using a rough event reconstruction.
The use of 30,000 cores in the HLT allows for buffering data
so the HLT has plenty of time to make this decision \cite{Sert:2712275}.

For this analysis, only a few of the possible HLT paths are of interest.
The exact trigger names are given in Chapter~\ref{ch:selection},
but for the most part, they only depend on three different identifiable objects.
Figures~\ref{fig:v-decay} and \ref{fig:two-lep-diagram}
show the different final states of interest.
$b$ jets are difficult to identify quickly since we must rely on the pixel detector's
reconstruction of the secondary vertex,
but the decay mode of the vector boson can be used for the trigger.
In that case, only events with one of the following are
worth saving and examining for $VH\rightarrow\bb$.
\begin{itemize}
  \item ECAL deposits consistent with a high $p_T$ electron
  \item muon chamber hits consistent with a high $p_T$ muon
  \item an overall energy imbalance consistent with MET from neutrinos
\end{itemize}
For the specific decay channel in Figure~\ref{fig:two-lep-diagram},
the HLT also includes paths where two electrons or two muons are identified.

\section{Accessing Data}

XRootD
