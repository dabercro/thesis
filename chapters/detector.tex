\chapter{The CMS Detector} \label{ch:detector}

The Compact Muon Solenoid (CMS) detector, located at the LHC,
consists of multiple sub-detectors.
Ideally, time could be saved for the dedicated reader by only detailing
the parts of the detector that are relevant for the analysis presented in this work.
Unfortunately, a characteristic of hadron collider measurements is that
almost all sub-systems of the detector are used to make the final measurement.

A brief overview of all the detector subsystems are therefore presented in this chapter.
More can be learned about the design and motivations for the detector in the TDR \cite{cms-tdr}.

\section{Associated Production at the LHC}

The CMS detector only observes events.
Before describing the devices that are used to observe and record events,
the method of generating interesting events must be described.
The CMS detector is located at the Large Hadron Collider (LHC).
Described in detail in multiple publications \cite{Evans_2008},
a brief description is given here.

The LHC, with a circumference of 26.7 km,
is large enough to be considered located in multiple towns and countries,
but it will suffice to say it is near Geneva, Switzerland
at the European Organization for Nuclear Research (CERN),
the main campus of which is addressed in Meyrin, Switzerland.
This campus itself also spans the border between Switzerland and France.
This large circumference is needed since charged particles traveling
in a circular path with radius $r$ emit synchrotron radiation at the following rate.
\begin{gather}
  P = \frac{q^2 p^4}{6\pi \epsilon_0 m^4 c^5 r^2}
\end{gather}
The amount of power lost by the particles decreases quadratically
with the size of the collider.
In addition, the energy lost decreases with the mass of the accelerated particles
to the fourth power.
The LHC was built in the same tunnels that were used for LEP,
which was a collider for electrons and positrons that took much of its data at
$\sqrt{s} = \SI{98}{GeV}$.
The resulting LHC is designed to collide protons at energies of $\sqrt{s} = \SI{14}{TeV}$,
with the data for this analysis taken at \SI{13}{TeV}.

Though Figure~\ref{fig:pdf} showed that not all of the energy from each proton
goes into the interaction,
there still is adequate phase space available to generate the massive off-shell
vector bosons that are needed for \emph{Higgstrahlung},
via the mechanisms described in Section~\ref{sec:produce-vector}.

The removal of a quark from one proton and an anti-quark from the other leads to two
non-color-singlet states in close proximity to each other.
The resulting spray of hadronic particles generated from the vacuum to restore
color singlets are called jets.
The detectors are designed to distinguish these jets from more interesting decay products
in the interaction.

\section{Detector Requirements}

One configuration of possible final state particles is shown in Figure~\ref{fig:two-lep-diagram}.
There, two oppositely charged leptons and two b quarks are the end decay products.
The $b$ quarks also hadronize form color singlets well before reaching the detector,
but the resulting jets can actually be distinguished well from the jets resulting
from the fragmenting protons.

Hadrons containing $b$ quarks decay through the weak force since they require a flavor change.
As mentioned before, the CKM matrix in Equation~\ref{eq:ckm} quantifies the mixing between the different quark flavors.
The value of $V_{tb}$ is close to unity, and since the CKM matrix is unitary,
$V_{cb}$ and $V_{ub}$ are small.
This means the matrix element weak decays of the $b$ hadrons is small.
This is the only decay channel available to the lightest $b$ hadrons,
so their lifetimes are relatively long.
The delayed decay results in a jet with a secondary vertex where many of its particles
are generated from the vacuum at a distance from the initial collision point.

Alternate signatures of interest can be seen by substituting other vector boson final states
from Figure~\ref{fig:v-decay}.
In these, there may be one or zero charged leptons,
with one or two neutral leptons, respectively.
Neutral particles are difficult to detect,
with neutral leptons being capable of passing through the entire Earth without
being part of a detectable interaction.
The CMS detector therefore ignores the neutrinos,
but their presence can still be inferred.
Even with the variation in momentum along the beam direction,
all partons in each proton have approximately zero momentum in the transverse direction.
Therefore, the sum of the transverse momenta of all final state particles must also be zero.
Many events in CMS have an overall imbalance in the transverse plane.
This imbalance is labelled Missing Transverse Energy, $E^\textrm{miss}_T$, or MET.
Large MET in an event is often a sign of high energy neutrinos that the detector cannot detect.

We need to identify all of these interesting particles,
as well as be able to reconstruct missing transverse momentum.
In addition, the additional hadronic activity in the event, called pileup, must be mitigated.
The energy of the decay products have energies on the scale of the masses of the parent particles.
The detector must be capable of measuring jets and leptons with energies on the
order of 10s or 100s of GeV.
Better energy resolution for each of these decay products allows better separation
of our signal process from background processes that generate very similar final states.

\section{Detector Design}

The CMS detector as a whole has cylindrical symmetry around the beam access.
It is 21 meters long and 15 meters in diameter.
There are gaps at either end to allow the beam,
but otherwise tries to cover the full solid angle around the collision point.
The azimuthal angle of a particle relative to the beam axis is described
by pseudorapidity, $\eta$.
\begin{gather}
  \eta = -\ln \left[\tan \left( \frac \theta 2 \right) \right]
\end{gather}
The barrel portion of CMS can detect particles with $|\eta| < 2.4$,
while the forward caps of the detector can reach $|\eta| < 5.0$.

Different technologies are better for measuring the energy
or other kinematics variables of different particles.
As a result, the CMS detector is made up of different sub-detector systems,
arranged in cylindrical layers.

The innermost layer is designed to extrapolate the tracks of charged particles
back to their point of origin.
This is called the Silicon Pixel Detector.
The next layer is designed to measure the energies of photons and electrons.
The third layer measures the energies of both charged and neutral hadrons.
Outside of these three layers is a superconducting solenoid,
which generates a magnetic field for the entire detector.
On the very outside of the detector are gas chambers designed to detect muons
interspersed with the iron return yoke for the solenoid.
A slice of the CMS detector showing the relative positions of each layer
is shown in Figure~\ref{fig:slice}.
\begin{figure}
  \centering
  \includegraphics[width=\linewidth]{figures/CMSslice_whiteBackground.png}
  \caption[CMS detector slice]{
    A slice of the CMS detector is shown above\cite{Barney:2120661}.
    The four detector layers are labelled and show the penetration
    depths of various particles stable enough to travel a measureable distance.
    }
  \label{fig:slice}
\end{figure}

\subsection{Solenoid Magnet}

\subsection{Silicon Pixel Detector}

\subsection{Electromagnetic Calorimeter}

\subsection{Hadronic Calorimeter}

\subsection{Muon Chambers}

\section{Detector Performance}

\subsection{Test Beam Performance}

\subsection{Trigger}

\subsection{Online Calibration}

\section{Data Format}

\subsection{Event Reconstruction} \label{sec:event-reco}

\subsection{Offline Calibration}

\section{Accessing Data}
