\chapter{The CMS Detector} \label{ch:detector}

The Compact Muon Solenoid (CMS) detector, located at the LHC,
consists of multiple sub-detectors.
The analysis in this work is quite complex,
and depends on all parts of the detector.
Therefore, a full description of CMS is presented in this chapter.

First, a brief description of the LHC is given in Section~\ref{sec:lhc}.
Then design requirements and considerations are outlined for the CMS detector
in Section~\ref{sec:requirements}.
Specific design decisions and descriptions of subdetectors are given
in Section~\ref{sec:design}.
Section~\ref{sec:event-reco} describes event reconstruction algorithms,
Section~\ref{sec:trigger} describes the triggers used to collect data,
and Section~\ref{sec:simulation} outlines simulation techniques used for CMS.
Finally, Section~\ref{sec:accessing-data} describes how data is stored and accessed
by members of the collaboration.
More can be learned about the design and motivations for the detector
in the TDR \cite{cms-tdr}.
Information presented on the physical CMS design parameters are
taken directly from that document unless otherwise noted.

\section{The Large Hadron Collider} \label{sec:lhc}

The CMS detector only observes events.
Before describing the devices that are used to observe and record events,
the method of generating interesting events must be described.
The CMS detector is located at the Large Hadron Collider (LHC).
Described in detail in multiple publications \cite{Evans_2008},
a brief description is given here.

The LHC, with a circumference of 26.7 km,
is large enough to be considered located in multiple towns and countries,
but it will suffice to say it is near Geneva, Switzerland
at the European Organization for Nuclear Research (CERN),
the main campus of which is addressed in Meyrin, Switzerland.
This campus itself also spans the border between Switzerland and France.
This large circumference is needed since charged particles traveling
in a circular path with radius $r$ emit synchrotron radiation at the following rate.
\begin{gather}
  P = \frac{q^2 p^4}{6\pi \epsilon_0 m^4 c^5 r^2}
\end{gather}
The amount of power lost by the particles decreases quadratically
with the size of the collider.
In addition, the energy lost decreases with the mass of the accelerated particles
to the fourth power.
The LHC was built in the same tunnels that were used for LEP,
which was a collider for electrons and positrons that took much of its data at
$\sqrt{s} = \SI{91}{GeV}$ in order to study the $Z$ boson resonance.
The resulting LHC is designed to collide protons at energies of $\sqrt{s} = \SI{14}{TeV}$,
with the data for this analysis taken at $\sqrt{s} = \SI{13}{TeV}$.
This is more than enough energy to generate the massive off-shell
vector bosons that are needed for \emph{Higgstrahlung},
as well as many accompanying jets,
via the mechanisms described previously in Chapter~\ref{ch:theory}.

The luminosity of the LHC is given by the following formula.
\begin{gather}
  \mathcal{L} = \frac{N^2_B f_{\text{rev}} k_B}{4\pi B^* \epsilon_{xy}} \times F
\end{gather}
$N_B$ is the number of protons per bunch,
$f_{\text{rev}}$ is the frequency of beam revolutions,
$k_B$ is the number of bunches per beam,
$B^*$ and $\epsilon_{xy}$ describe the goodness of the beam,
and $F$ is a geometric collision factor.
\begin{gather}
  F = \frac{1}{\sqrt{1 + \frac{(\sigma_s \tan \phi)^2}{\epsilon_{xy} \beta^*}}}
\end{gather}
$\sigma_s$ is the length of each bunch,
$\phi$ is the crossing angle,
and $\beta^*$ is the value of the amplitude function at the focal point.
In order to generate as much collision data as possible,
the LHC operates at a high frequency of collisions,
and generates many simultaneous collisions.
For Run 2, there is a proton bunch crossing every \SI{25}{ns}.
The CMS detector must be able to read out and process data on that timescale.
Each proton bunch includes over 100 billion protons \cite{Bruce:2201447}.

\section{Detector Requirements} \label{sec:requirements}

One configuration of possible final state particles was shown previously
in Figure~\ref{fig:two-lep-diagram}.
There, two oppositely charged leptons and two b quarks are the end decay products.
The $b$ quarks also hadronize form color singlets well before reaching the detector,
but the resulting jets can actually be distinguished well from the jets resulting
from the fragmenting protons.

Hadrons containing $b$ quarks decay through the weak force
since they require a flavor change.
As mentioned before, the CKM matrix in Equation~\ref{eq:ckm}
quantifies the mixing between the different quark flavors.
The value of $V_{tb}$ is close to unity, and since the CKM matrix is unitary,
$V_{cb}$ and $V_{ub}$ are small.
This means the matrix element weak decays of the $b$ hadrons is small.
This is the only decay channel available to the lightest $b$ hadrons,
so their lifetimes are relatively long.
The delayed decay results in a jet with a secondary vertex where many of its particles
are generated from the vacuum at a distance from the initial collision point.

Alternate signatures of interest can be seen by substituting other vector boson final states
from Figure~\ref{fig:v-decay}.
In these, there may be one or zero charged leptons,
with one or two neutral leptons, respectively.
Neutral particles are difficult to detect,
with neutral leptons being capable of passing through the entire Earth without
being part of a detectable interaction.
The CMS detector therefore ignores the neutrinos,
but their presence can still be inferred.
Even with the variation in momentum along the beam direction,
all partons in each proton have approximately zero momentum in the transverse direction.
Therefore, the sum of the transverse momenta of all final state particles must also be zero.
Many events in CMS have an overall imbalance in the transverse plane.
This imbalance is labelled Missing Transverse Energy, $E^\textrm{miss}_T$, or MET.
Large MET in an event is often a sign of high energy neutrinos
that the detector cannot detect.

We need to identify all of these interesting particles,
as well as be able to reconstruct missing transverse momentum.
In addition, the additional hadronic activity in the event, called pileup, must be mitigated.
The energy of the decay products have energies on the scale of
the masses of the parent particles.
The detector must be capable of measuring jets and leptons with energies on the
order of 10s or 100s of GeV.
Better energy resolution for each of these decay products allows better separation
of our signal process from background processes that generate very similar final states.

\section{Detector Design} \label{sec:design}

The CMS detector as a whole has cylindrical symmetry around the proton beams.
It is 21 meters long and 15 meters in diameter.
There are gaps at either end to allow the beams to enter and leave,
but otherwise the design tries to cover the full solid angle around the collision point.
The azimuthal angle of a particle relative to the beam axis is described
by pseudorapidity, $\eta$.
\begin{gather}
  \eta = -\ln \left[\tan \left( \frac \theta 2 \right) \right]
\end{gather}
The barrel portion of CMS detects particles up to $|\eta| < 1.5$,
while the forward caps of the detector can reach $|\eta| < 5.0$.
The muon and pixel trackers reach up to $|\eta| < 2.5$,
with additional space covered by calorimetry.

Different technologies are better for measuring the energy
or other kinematics variables of different particles.
As a result, the CMS detector is made up of different sub-detector systems,
arranged in cylindrical layers.
Each layer consists of a ``barrel'' portion and two end caps on either side.

The innermost sub-detector is designed to extrapolate the tracks of charged particles
back to their point of origin.
This is called the Silicon Tracker.
One key design feature of the pixel detector is that it is non-destructive.
Particles it detects pass through to the rest of the detector for additional
measurement and identification.
The next sub-detector encountered by most particles is the Electromagnetic Calorimeter,
which is designed to measure the energies of photons and electrons.
The next sub-detector, the Hadronic Calorimeter,
measures the energies of both charged and neutral hadrons.
The two calorimeters are destructive because absorb the particles that interact with them
in order to measure their full energy.
Outside of these three sub-detectors is a superconducting solenoid,
which generates a magnetic field for the entire detector.
On the very outside of the detector are gas chambers designed to detect muons
interspersed with the iron return yoke for the solenoid.
A slice of the CMS detector showing the relative positions of each layer
is shown in Figure~\ref{fig:slice}.
\begin{figure}
  \centering
  \includegraphics[width=0.9\linewidth]{figures/CMSslice_whiteBackground.png}
  \caption[CMS detector slice]{
    A slice of the CMS detector is shown above \cite{Barney:2120661}.
    The four detector layers are labelled and show the penetration
    depths of various particles stable enough to travel a measurable distance.
    }
  \label{fig:slice}
\end{figure}

The magnet is described first since the magnetic field it produces is a key
part of most of the rest of the detector.
After that, the sub-detectors are summarized in the order of closest to farthest
from the beamline, since this is the order that particles would interact with the layers.
Each sub-detector section also describes the measured performance during Run 2.
Note that this measurement is an iterative process that depends on the event reconstruction
described in Section~\ref{sec:event-reco},
which in turn depends on the performance of the entire detector.
The performance numbers are presented with each sub-detector design though so that
it is immediately clear how effective each design has been.

\subsection{Solenoid Magnet}

Part of the CMS acronym acknowledges the role of the solenoid magnet.
The presence of a magnetic field is paramount for accurate measurements
of charged particles passing through the silicon detector and the muon chambers.

The magnetic field generated is designed to cause the path of
a muon with \SI{1}{TeV} of energy to bend enough to have a momentum resolution of 10\%.
Inside the solenoid, the magnetic field operates at \SI{3.8}{T},
with the solenoid design being capable of achieving \SI{4}{T}.
The return field is large enough to cause muon tracks to curve throughout
the muon chambers outside the magnet.

A super-conducting solenoid enables the creation of a magnetic field
with the required strength.
A current of \SI{19.5}{kA} is sent through 2168 turns over \SI{12.9}{m}.
The magnetic field stores \SI{2.7}{GJ} of energy.
In order to hold this, the structural components holding the magnet and
the detector in place are strong enough to withstand \SI{64}{atm} of hoop pressure.

\subsection{Silicon Tracker}

The layer closest to the beamline is designed to obtain a precise track pointing
to the origin of particles passing into the detector.
It is made up of layers of many small pixels to do this.
As distance from the interaction point increases, the pixel size also increases
since the absolution spacial resolution does not need to be as fine.

The innermost three layers, with the closest layer being a distance of
$r=\SI{4}{cm}$ from the interaction point,
are made of hybrid pixel detectors.
Each pixel has dimensions of $100 \times \SI{150}{\micro m}$
in order to achieve fine resolution of where the detected particles originiated.
The TDR also claims an occupancy of $10^{-4}$ per pixel per LHC bunch crossing,
which improves the pixel's longevity and reduces problems from detector deadtime.
Outside of the pixel detector layers,
silicon strip detectors are used.
These are placed in the region that is $20 < r < \SI{55}{cm}$ from the beamline.
Strip dimensions give a cell size of approximately $\SI{10}{cm} \times \SI{80}{\micro m}$.
2--3\% of cells are activiated during a typical bunch crossing.
The outermost layers are made of larger strips with cell sizes of
$\SI{25}{cm} \times \SI{180}{\micro m}$.
About 1\% of these pixels are triggered each bunch crossing.

The active material of the silicon pixel detector is semi-conducting silicon.
When charged particles pass through,
electron-hole pairs are generated and drift apart due to a bias voltage.
The voltage change when these pairs reach their respective electrodes
indicates a charged particle passed through.
Because of this, the silicon pixel detector cannot detect neutral particles,
but it gives a precise point of origin for charged particles.
The points of origin allow for the determination of locations of primary and secondary
collision vertices, which plays an important role in the identification of pileup.

In the beginning of 2017, the pixel detector was upgraded to handle
the higher radiation environment of Run 2 \cite{Dominguez:1481838}.
A layer was added to both the barrel and endcap sections of the pixel detector.
Firmware was also upgraded to keep the pixel detector operating
at a frequency higher than the Run 2 collision frequency.
With this upgrade, the detector operated with a 97\% hit efficiency
for all layers at the highest instantaneous luminosity.
Layers beyond the first performed with greater than 99\% hit efficiency \cite{Modak:2712284}.

\subsection{Electromagnetic Calorimeter} \label{sec:ecal}

The next layer of the detector is called the Electromagnetic Calorimeter or ECAL.
This layer is designed to fully capture and accurately measure
the energy of photons and electrons.
The ECAL is made of crystals of the scintillating material Lead Tungstate (\ch{PbWO4}).
Each crystal is placed in the detector so that its smallest face
is facing the collision point.
These small faces have dimensions of $22\times\SI{22}{mm}$.
The length of each crystal is \SI{230}{mm},
and the far face is slightly larger at $26\times\SI{26}{mm}$.
\ch{PbWO4} has a radiation length of $\chi_0 = \SI{8.9}{mm}$ and
a Moliere radius of \SI{21}{mm}.
This means each crystal is 25.8 radiation lengths,
containing the full shower within the ECAL,
and each shower is also localized to within one crystal from the initial ionization.
There is are gaps in active detecting volume of the ECAL,
which are needed to accommodate various electronics and structural components.
The gaps are located in symmetric locations on either side of the ECAL from
$|\eta| = 1.4442$ to $|\eta| = 1.5660$.

The scintillating properties of \ch{PbWO4} are also desirable for observing LHC collisions.
The photodiodes at the far end of the crystals
ultimately detect 4.5 photons for every MeV of energy deposited in the ECAL.
This is a low number for other experiments, but the only photons and electrons of
interest in this measurement deposit at least tens of GeV of energy.
This gives the ECAL energy resolutions in the range of $5-10\%$.
More importantly, the scintillation is very fast.
About 80\% of the light from an interaction is emitted within
the \SI{25}{ns} between bunch crossings,
making it easy to associate the readouts with the appropriate bunch crossing.

When exposed to the high radiation environment of the LHC,
the ECAL crystals are damaged by radiation.
Damage to the crystal structure causes it to become more opaque to the scintillated light.
Much of this damage happens within the first 30 minutes of operation.
Some recovery occurs as the crystal structure falls back into the ground state,
but over time, the performance of the crystals degrade.
That degradation happens at different rates in different areas of the detector,
but, aside from the initial darkening,
is slow enough to be able to correct for it during the run.
Lasers are used to calibrate the ECAL online during
the gaps between beams \cite{Monti:2653861}.
Resolution is measured by looking at $Z\rightarrow ee$ events.
For Run 2, the barrel region of the ECAL performed with 1.6\% resolution,
and the other regions had a 5\% resolution \cite{Bartosik:2712238}.

\subsection{Hadronic Calorimeter}

The ECAL absorbs electromagnetic particles and measures their energies
which are dispersed in electromagnetic showers.
Hadrons deposit energy in hadronic showers,
which require a different mechanism to contain and measure.
The Hadronic Calorimeter (HCAL) does this.
Like the ECAL it contains particles and measures their energy destructively.
However, it does this for hadrons, such as protons, neutrons, and stable mesons.
Since hadrons are much more massive than electrons,
the ionizing collisions in a typical scintillator does not slow them down
enough to contain them.
Instead, they must interact via nuclear collisions to be attenuated.
CMS uses brass for its HCAL due to its relatively short interaction length,
the fact that it is non-magnetic, and its affordability.

The barrel of the HCAL is jacketed in stainless steel for structural support.
This layer is \SI{61}{mm} thick on the layer immediately next to the ECAL and
\SI{75}{mm} thick on the outer edge.
The inside of the HCAL consists of brass absorber plates
interspersed with plastic scintillator tiles.
The layers closer to the beamline alternate \SI{50.5}{mm}
brass plates with \SI{3.7}{mm} scintillator plates.
Farther away, the brass plates are instead \SI{56.5}{mm} thick.
Wavelength shifting fibers are run through the scintillator tiles
to allow photons to travel to the outside of the HCAL where they are detected by photodiodes.

Like the ECAL, the HCAL performance also degrades as it is exposed to radiation.
The calibration for HCAL is performed using an embedded radioactive source,
lasers and LEDs, and an \emph{in situ} calibration using assumed symmetry in $\phi$.
With these methods, a response within 3.4\% was maintained in the HCAL barrel
and within 2.6\% in the HCAL endcap up to $|\eta| < 2$ \cite{Chadeeva_2018}.

\subsection{Muon Chambers}

Muons are the most penetrative particles that CMS detects.
Through the calorimeters, muons act as minimum-ionizing particles \cite{James:927392}.
They are heavier than electrons, so they are not stopped in the ECAL.
They do not interact via the strong nuclear force,
so the high density of the HCAL also does not cause significant interactions.
Instead of stopping and measuring muons in calorimeters,
CMS tracks their trajectory with both the silicon tracker on the inside of the detector
and the muon chambers that make up the outer layer of the detector.

This is the only sub-detector system outside of the solenoid,
but the returning magnetic field is still present outside of the return yoke
\cite{828264,Klyukhin:2062915},
allowing the momentum of the muons to be extracted from the curvature of their trajectory.
Layers of muon chambers act much like the pixel detectors,
but at a larger and more distant scale.
The muon chambers in the barrel region of $|\eta| < 1.2$ consist of drift tube chambers.
In the endcaps, cathode strip chambers are used.
The difference is to account for higher neutron backgrounds in the endcap,
as well as a greater magnetic field.
In both regions, resistive plate chambers are spaced
between the layers of the other muon chambers.

Each muon chamber has a detection efficiency greater than 95\%.
The overall efficiency of the muon trigger, which relies heavily on the muon system
and is described in more detail in Section~\ref{sec:trigger},
increases as a function of muon $p_T$ and plateaus around 90\%.
The timing of the muon system leads to 1\% of muons
to be assigned as originating from the wrong bunch crossing \cite{Pozzobon_2019}.

\section{Event Reconstruction} \label{sec:event-reco}

Each sub-detector reconstructs the particles that passes through it.
The independent reconstructions are then linked
across the different detector components to identify particles.
This overview is taken largely from reference \cite{Sirunyan_2017}.

\subsection{Charged Particle Tracks} \label{sec:tracks}

Both the Silicon Tracker and the Muon Chambers are designed
for charged particles to leave tracks.
In both sub-detectors, the basic steps for track reconstruction are the same.
First, a track must be seeded.
Usually, this is done by finding hits in consecutive layers that are consistent
with a particle coming from the beamline.
Particles loose energy as they pass through matter,
and they can also be redirected through multiple scattering,
so the extrapolation is non-trivial.
A Kalman Filter is used to find hits in the other layers of the appropriate sub-detector
that are consistent with the initial seed.
Once more hits have been found, a fit is performed for the precise trajectory of the track.

Tracks are kept or discarded based on the number of layers that are missing hits
and on the momentum of a charged track in the magnetic field of the solenoid.
Making these parameters looser results in better recovery of tracks,
but the high activity within the detector results in a combinatorial background.
This background increases exponentially when the momentum cut is reduced, for example.
To help reduce this background, tracks with missing hits use an iterative fit.
Different seeds are found for each track to make sure that the resulting collections
of hits remain the same.

Additional complications arise for each the pixel and muon detectors.
The Silicon Tracker is the only tracking detector that deals with electrons.
Because of their small mass, electrons are likely to radiate energy
while travelling through the magnetic field of CMS.
This leads to complications in the calorimeters described in Section~\ref{sec:calorimeters}.
It also means that the radius of curvature of an electron track can decrease appreciably
within the pixel detector.
This can lead to the Kalman Filter approach missing tracks entirely,
depending on the number of hits required.
Tracks with a large $\chi^2$ and a certain number of hits are fit again
using a Gaussian-Sum Filter (GSF).
The GSF allows for fitting tracks that have significant energy loss,
recovering electron and positron tracks.

A change of trajectory may also happen in the muon chambers,
but this is due to multiple scattering in the return yoke.
No specialized tracking algorithm is used to account for it.
The muon tracking performs best when the tracks in the muon chambers are successfully
linked to a track in the pixel detector.
The most common backgrounds in the muon chambers
is caused when hadrons manage to punch through the HCAL.
This is often mitigated by considering the amount of energy deposited along the particle
track in the other sub-detector systems.

\subsection{Calorimeters} \label{sec:calorimeters}

Clusters are identified in calorimeters, also using a seeding algorithm.
First clusters with a large energy deposit are identified,
and then nearby crystals are checked against noise thresholds.
The energy deposition is assumed to have a Gaussian profile,
and a fit is performed to disentangle overlapping energy depositions.

The dimensions of the ECAL crystals
are comparable to the Moliere radius, keeping clusters localized.
Although a complication arises due to curvature of the electromagnetic shower
caused by the magnetic field, leading to the need for using GSF to find tracks.
Superclusters that are linked to electrons have
a larger allowed range over $\phi$ to account for this.

The calibration of HCAL is complicated by the fact that particles reaching it
have to first pass through the ECAL.
Initial calibration was done with a \SI{50}{GeV} pion test beam,
but the actual response is non-linear in energy
as well as different for charged and neutral particles.
Reasons for this difficulty include particles losing energy in the region
between the ECAL and the HCAL, in addition to the energy lost in the ECAL.
Therefore, there are calibration coefficients that are used
depending on if the energy deposits are all in the HCAL,
or in preceding sub-detectors as well.

\subsection{Linking and Particle Identification}

An important step in making sense of the various sub-detector readouts is linking
tracks to calorimeter clusters.
The general procedure is to extrapolate tracks from the inner tracker
out to each calorimeter.
A shower that originates within one radiation or interaction length in the calorimeter
along that track is linked with the track.

The bremsstrahlung from GSF electrons is linked to the track by looking along track tangents.
A dedicated conversion finder is used to identify pair production within the pixel detector
caused by either bremsstrahlung or prompt photons in order to not mistakenly link
a charged particle track with what should otherwise be measured as a photon.
This step, in addition to some ECAL clusters that do not have a track
make it possible to identify isolated photons.
On the other hand, it is still difficult to determine
whether an electron track is well isolated or not.
The large number of variables that go into identifying an electron
leads to the training of a Boosted Decision Tree (BDT) to identify an electron.
Separate BDTs are needed for the barrel and endcap regions of the detector.

For accurate HCAL readings,
the linking algorithm also ECAL clusters and HCAL clusters along a path.
These may not always be along a charged particle track.
Multiple calorimeter links of this nature may be found,
but only a single link is kept based on a distance assigned to each link.
HCAL hits without a track are identified as neutral hadrons.
HCAL hits with a linked track are likely charged hadrons.
Though the ECAL clusters must be linked in order to determine the energy
coefficient to calibrate the HCAL,
ECAL hits without tracks are still identified as photons because photons
carry some of the energy of jets.

Of particular interest to this analysis is also the secondary vertex step of linking.
Charged particle tracks that do not go back to the interaction vertex
are linked together if they share a common secondary vertex.
These tracks must have a mass greater than \SI{0.2}{GeV} to be kept.
There must also be a track from the secondary vertex to the primary vertex,
which would belong to a long-lived hadron.
As mentioned in Section~\ref{sec:requirements}, this is the signature of a $b$ jet.
It is possible, however, that the secondary vertex is generated
by a nuclear scattering, pair production, or other long-lived particles
like $K_S$ or $\Lambda$ within the pixel detector,
so additional analysis is needed for each secondary vertex.

The final link is made between tracks in the muon chambers and tracks in the inner tracker.
Muons are identified as tracker muons if they only leave tracks in the inner tracker.
This can often happen with low energy muons.
They are called standalone muons when only the track in the muon chamber is identified.
When tracks are successfully linked in both sub-detectors,
the resulting reconstructed particle is called a global muon.
When a global muon is not well-isolated from other energy deposits,
it must have tighter requirements on how it behaves in the muon chambers.
This is to prevent energy from a jet from being attributed to a muon or vice versa.
This is important for $b$ jets because the decay that happens at a displaced vertex
is a decay through the weak nuclear force,
which can result in leptons being present inside of a jet.

\section{Trigger} \label{sec:trigger}

Bunch crossings happen every \SI{25}{ns},
with each bunch crossing producing on average 20 collisions.
The amount of data that the detector generates for each bunch crossing is
too large to store all of it at this rate.
Luckily, most collisions result in uninteresting data.
A trigger system is used to identify interesting events
and reduce the frequency of event writing to \SI{1}{kHz}.
This is done using two stages.
The Level-1 (L1) trigger passes events with a frequency of \SI{100}{kHz},
and the High Level Trigger (HLT) picks from the remaining events
with a frequency of \SI{1}{kHz} \cite{Tosi:2290106}.

The L1 trigger is implemented in hardware.
It was upgraded for Run 2 of the LHC to run on FPGAs on an Advanced Mezzanine Card.
There are two main components of the L1 trigger.
One considers calorimeter deposits, and the other examines the muon chambers.
The overall L1 trigger fires when there are high energy, resolved calorimeter hits
or if a possible muon is reconstructed.
Due to the flexibility of FPGAs,
the exact conditions of the firing are configurable \cite{Cadamuro_2017}.

After the L1 is fired, the data is sent to the High-Level Trigger (HLT),
which is a computing farm that makes a final decision on whether or not to save the data
using a rough event reconstruction.
The use of 30,000 cores in the HLT allows for buffering data
so the HLT has plenty of time to make this decision \cite{Sert:2712275}.

For this analysis, only a few of the possible HLT paths are of interest.
The exact trigger names are given in Chapter~\ref{ch:selection},
but for the most part, they only depend on three different identifiable objects.
Figures~\ref{fig:v-decay} and \ref{fig:two-lep-diagram}
show the different final states of interest.
$b$ jets are difficult to identify quickly because we must rely on the Silicon Tracker's
reconstruction of the secondary vertex,
but the decay mode of the vector boson can be used for the trigger.
More boosted vector bosons leave a signature with a higher trigger efficiency.
They also will cause the $b$ jets to have a higher $p_T$,
leading to easier identification and measurement.
In that case, only events with one of the following are
worth saving and examining for $VH\rightarrow\bb$.
\begin{itemize}
  \item ECAL deposits consistent with a high $p_T$ electron
  \item muon chamber hits consistent with a high $p_T$ muon
  \item an overall energy imbalance consistent with MET from neutrinos
\end{itemize}
For the specific decay channel in Figure~\ref{fig:two-lep-diagram},
the HLT also includes paths where two electrons or two muons are identified.

\section{Simulation} \label{sec:simulation}

After the detector is well understood,
predictions on how it responds in the LHC environment can be made.
The number of ways the detector could possibly respond are nearly infinite.
Therefore, simulation is performed using Monte Carlo methods,
and the resulting analysis is statistical in nature.
The data format for simulation results is similar to the data format for
data collected from the detector.
Unobservable information about intermediate steps
in the simulation is also stored,
but otherwise the data is the predicted output of a collection of events.

The simulation itself consists of several steps
each outlined in a separate section of this chapter.
First the background processes that will appear in our analysis must be known.
Identifying all of these processes is necessary to quantify and characterize
the signal events that are also mixed in to our selection.
Then each of these processes must be simulated to determine the final state particles
that the detector will observe.
Each process will looks slightly different in our signal selection.
Events outside of the selection must also be simulated so that they can be studied
for accuracy in separate phase spaces that do not include the Higgs events.
After the final state particles are predicted,
the detector response to those particles passing through must be simulated.
This allows researchers to compare the physical readouts
they can observe to predicted detector signals.
Finally, using the phase spaces outside of the desired signal process,
minor corrections to the simulation can be made.
Simulated energies from the detector model might not be the exact same as
what the physical detector produces, for example,
and they must be made to match to make the signal process cleanly appear in the analysis.

The physical processes that occur at the LHC all contain QCD-driven phenomena.
As a result, the part of the simulation that predicts the particles present in the detector
has two distinct parts.
QCD is perturbative at small distances,
and other forces are perturbative at all distances.
The collisions themselves are in this regime,
so the initial- and final-state particles over a distance of femtometers
can be simulated using typical calculations using
perturbative rules described by Feynman diagrams.
The decay of unstable particles can also be simulated this way.
Once particles interacting through QCD exceed this distance,
well before reaching the detector,
hadronization, or parton showers, must be simulated differently.
The following two sub-sections describing these techniques.
The exact generators and configurations used to simulate each process for this analysis
are detailed in Appendix~\ref{app:generator}.
After final state particles are generated,
their propagation through the detector is simulated in a third step.

\subsection{Short-Scale Simulation}

Events are generated by selecting results and assigning weights in a way proportional
to the phase space and the matrix element squared of the event.
The phase space integral has the following form \cite{Peskin:257493}.
\begin{gather}
  \int d\Pi_n = \left( \prod_f \int \frac{d^3p_f}{(2\pi)^3} \frac{1}{2E_f} \right)
                (2\pi)^4 \delta^{(4)}(P - \sum p_f)
\end{gather}
$P$ is the total initial 4-momentum and $f$ runs over all final state particles.
This phase space integral is Lorentz invariant.
Once an event has been selected, the available phase space can then be used to assign
directions to the final particles.

The proportional matrix elements are described in Chapter~\ref{ch:theory},
but most of the diagrams described were Leading Order (LO).
Generators used in this analysis can also simulate Next to Leading Order (NLO) processes
thanks to the FKS method of subtracting particles to avoid double counting them during the
showering calculation.
However, the option to use NLO simulation is not always used.
The results of NLO calculations more accurately predict physical processes,
but they also take more computational resources,
resulting in larger measurement uncertainties due to statistical limitations.

The generators used for short-scale simulation in this analysis
are POWHEG \cite{Oleari_2010} and MadGraph5 \cite{hirschi2015automated}.

\subsection{Parton Showers}

The final state particles from the short-scale simulation
include a number of free quarks and gluons.
As mentioned in Section~\ref{sec:produce-vector},
it is energetically favorable for color-charged particles to create additional
quark/anti-quark pairs to screen the color charge.
This is known as hadronization or parton showering and
is simulated separately from the calculation of tree-level processes.
Hadronization happens well before particles reach the CMS detector,
so the results are needed to predict the detector response.
Accurate simulation of this process is important for all collisions at the LHC,
which produces much QCD background.
It is also important to accurately simulate the constituents of individual jets
because this analysis includes detailed inspection of each jet
in order to identify $b$-jets and to estimate the amount of energy carried away by neutrinos.

To be able to analyze the simulation in the same way data is processed,
Monte Carlo simulation is used to predict precise final states of the jets.
CMS uses the Lund model \cite{ANDERSSON198331} as implemented in
\texttt{PYTHIA8} \cite{SJOSTRAND2015159}.

\subsection{Detector Simulation}

After determining all of the final state particles that will reach the detector,
the interaction between these particles and the detector components must be simulated.
Multiple simulations of $pp$ collisions are combined to simulate pileup,
and then the particle propagation through various materials is done with
\texttt{GEANT4} \cite{AGOSTINELLI2003250}.
The full CMS detector geometry is maintained within CMSSW \cite{Hildreth_2015}
using a framework written in the Unified Modelling Language \cite{Lefebure:687188}.
To be able to process the simulated data in the exact same fashion as the measured data,
the readout of the electronics is also simulated.

\section{Accessing Data} \label{sec:accessing-data}

The final important piece of the CMS detector is its offline computing resources.
The data that is gathered by the detector must be processed and stored.
This is done by using computing resources spread around the world.
They are grouped into Tier-1 and Tier-2 sites,
based largely on the geographic space that they are meant to provide computing services for.
Tier-1 sites typically provide 30,000 CPU cores, while the more numerous Tier-2 sites provide
another 60,000 \cite{Bloom_2017}.
Together they also provide around 100 Petabytes of space \cite{Ratnikova_2014}.
Keeping these services running smoothly is important for the CMS collaboration to function,
and two projects that aid in this are outlined in Appendix~\ref{app:project}.
Since the data is event-based, they are stored in n-tuple format.
These n-tuples are created and read by the CMS Software (CMSSW) \cite{INNOCENTE200131}.
