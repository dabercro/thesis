\chapter{The CMS Detector} \label{ch:detector}

The Compact Muon Solenoid (CMS) detector is a complex device located at the LHC.
Ideally, a lot of time could be saved for the reader by only detailing the parts of the detector that are relevant for the analysis presented in this work.
Unfortunately, a characteristic of contemporary collider measurements is that almost all sub-systems of the detector are used to make the final measurement.

A brief overview of all the detector subsystems are therefore presented in this chapter.
A curious reader can learn more about the design and motivations for said design in the TDR \cite{cms-tdr}.


\section{Associated Production at the LHC}

Before describing the devices that are used to observe and record events,
the method of generating interesting events must be described.
The CMS detector is located at the Large Hadron Collider (LHC).
Described in detail in multiple publications \cite{Evans_2008},
a brief description is given here.

The LHC, with a circumference of 26.7 km, is large enough to be considered located in multiple towns and countries,
but it will suffice to say it is near Geneva, Switzerland at the European Organization for Nuclear Research (CERN), the main campus of which is addressed in Meyrin, Switzerland.
This campus itself also spans the border between Switzerland and France.

Quark interactions can lead to Vector boson production,
since the protons are at high energy.

Virtual vector bosons can radiate a Higgs.

This happens along with additional event stuff,
caused by breaking apart the protons and pileup.

\section{Observables of Associated Production}

We can count charged leptons or infer neutrinos to tag vector bosons.

We are interested in Higgs to \bb.

\section{Detector Requirements}

We need to identify interesting particles,
be able to reconstruct missing transverse momentum,
and remove pileup.

\subsection{Particle Types and Energies}

\subsection{Pileup Conditions at the LHC}

\section{Detector Design}

\subsection{Silicon Pixel Detector}

\subsection{ECAL}

\subsection{HCAL}

\subsection{Muon Chambers}

\section{Detector Performance}

\subsection{Test Beam Performance}

\subsection{Trigger}

\subsection{Online Calibration}

\section{Data Format}

\subsection{Event Reconstruction} \label{sec:event-reco}

\subsection{Offline Calibration}

\section{Accessing Data}
