\chapter{The CMS Detector} \label{ch:detector}

The Compact Muon Solenoid (CMS) detector, located at the LHC,
consists of multiple sub-detectors.
Ideally, time could be saved for the dedicated reader by only detailing
the parts of the detector that are relevant for the analysis presented in this work.
Unfortunately, a characteristic of hadron collider measurements is that
almost all sub-systems of the detector are used to make the final measurement.

A brief overview of all the detector subsystems are therefore presented in this chapter.
More can be learned about the design and motivations for the detector in the TDR \cite{cms-tdr}.

\section{Associated Production at the LHC}

The CMS detector only observes events.
Before describing the devices that are used to observe and record events,
the method of generating interesting events must be described.
The CMS detector is located at the Large Hadron Collider (LHC).
Described in detail in multiple publications \cite{Evans_2008},
a brief description is given here.

The LHC, with a circumference of 26.7 km,
is large enough to be considered located in multiple towns and countries,
but it will suffice to say it is near Geneva, Switzerland
at the European Organization for Nuclear Research (CERN),
the main campus of which is addressed in Meyrin, Switzerland.
This campus itself also spans the border between Switzerland and France.
This large circumference is needed since charged particles traveling
in a circular path with radius $r$ emit synchrotron radiation at the following rate.
\begin{gather}
  P = \frac{q^2 p^4}{6\pi \epsilon_0 m^4 c^5 r^2}
\end{gather}
The amount of power lost by the particles decreases quadratically
with the size of the collider.
In addition, the energy lost decreases with the mass of the accelerated particles
to the fourth power.
The LHC was built in the same tunnels that were used for LEP,
which was a collider for electrons and positrons that took much of its data at
$\sqrt{s} = \SI{98}{GeV}$.
The resulting LHC is designed to collide protons at energies of $\sqrt{s} = \SI{14}{TeV}$,
with the data for this analysis taken at \SI{13}{TeV}.

Though Figure~\ref{fig:pdf} showed that not all of the energy from each proton
goes into the interaction,
there still is adequate phase space available to generate the massive off-shell
vector bosons that are needed for \emph{Higgstrahlung},
via the mechanisms described in Section~\ref{sec:produce-vector}.

The removal of a quark from one proton and an anti-quark from the other leads to two
non-color-singlet states in close proximity to each other.
The resulting spray of hadronic particles generated from the vacuum to restore
color singlets are called jets.
The detectors are designed to distinguish these jets from more interesting decay products
in the interaction.

\section{Observables of Associated Production}

One configuration of possible final state particles is shown in Figure~\ref{fig:two-lep-diagram}.
There, two oppositely charged leptons and two b quarks are the end decay products.
The $b$ quarks also hadronize form color singlets well before reaching the detector,
but the resulting jets can actually be distinguished well from the jets resulting
from the fragmenting protons.

Alternate signatures of interest can be seen by substituting other vector boson final states
from Figure~\ref{fig:v-decay}.
In these, there may be one or zero charged leptons.
Neutral particles are difficult to detect,
with neutral leptons being capable of passing through the entire Earth without
being part of a detectable interaction.
The CMS detector therefore ignores the neutrinos,
but their presence can still be inferred.


\section{Detector Requirements}

We need to identify interesting particles,
be able to reconstruct missing transverse momentum,
and remove pileup.

\subsection{Particle Types and Energies}

\subsection{Pileup Conditions at the LHC}

\section{Detector Design}

\subsection{Silicon Pixel Detector}

\subsection{ECAL}

\subsection{HCAL}

\subsection{Muon Chambers}

\section{Detector Performance}

\subsection{Test Beam Performance}

\subsection{Trigger}

\subsection{Online Calibration}

\section{Data Format}

\subsection{Event Reconstruction} \label{sec:event-reco}

\subsection{Offline Calibration}

\section{Accessing Data}
