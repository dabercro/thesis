\chapter{Analysis Results} \label{ch:results}

\section{Run 2 Data Collection}

The CMS detector collected proton-proton collision data at $\sqrt{s} = \SI{13}{TeV}$
over three years during Run 2 of the LHC.
In 2016, CMS collected \SI{35.90}{fb^{-1}} of data \cite{CMS-PAS-LUM-17-001}.
In 2017, \SI{41.53}{fb^{-1}} of data was collected \cite{CMS-PAS-LUM-17-004}.
In 2018, CMS collected \SI{58.83}{fb^{-1}} of collision data \cite{CMS-PAS-LUM-18-002}.

\section{Control Region Plots}

A number of variables are compared in all of the control regions
to ensure that the data and simulation are in reasonable agreement.

\begin{figure}
  \centering
  \includegraphics[width=0.42\linewidth]{figures/2016_Zll/ttbar_low_Zll__kinFit_V_pt_.pdf} ~
  \includegraphics[width=0.42\linewidth]{figures/2016_Zll/ttbar_med_Zll__kinFit_V_pt_.pdf}
  \includegraphics[width=0.42\linewidth]{figures/2016_Zll/Zhf_low_Zll__kinFit_V_pt_.pdf} ~
  \includegraphics[width=0.42\linewidth]{figures/2016_Zll/Zhf_med_Zll__kinFit_V_pt_.pdf}
  \includegraphics[width=0.42\linewidth]{figures/2016_Zll/Zlf_low_Zll__kinFit_V_pt_.pdf} ~
  \includegraphics[width=0.42\linewidth]{figures/2016_Zll/Zlf_med_Zll__kinFit_V_pt_.pdf}
  \caption[Control region plots for two leptons in 2016]{
    The vector boson $p_T$ is plotted for multiple control regions and bins in
    an inclusive selection of both the $Z\rightarrow \mu\mu$ and $Z\rightarrow ee$
    two lepton channels.
    The top row shows the $t\bar{t}$ control region,
    the middle row shows the $Z$+ heavy jets,
    and the bottom row shows the $Z$+ light jets control region.
    The left column shows the low $p_T(V)$ bin of 75 to \SI{150}{GeV},
    and the right column shows the medium $p_T(V)$ bin of 150 to \SI{250}{GeV}.
  }
  \label{fig:2016_Zll_vpt}
\end{figure}

\begin{figure}
  \centering
  \includegraphics[width=0.42\linewidth]{figures/2016_Wln/ttbar_med_Wln__Hj0_pt_.pdf} ~
  \includegraphics[width=0.42\linewidth]{figures/2016_Wln/ttbar_high_Wln__Hj0_pt_.pdf}
  \includegraphics[width=0.42\linewidth]{figures/2016_Wln/Whf_med_Wln__Hj0_pt_.pdf} ~
  \includegraphics[width=0.42\linewidth]{figures/2016_Wln/Whf_high_Wln__Hj0_pt_.pdf}
  \includegraphics[width=0.42\linewidth]{figures/2016_Wln/Wlf_med_Wln__Hj0_pt_.pdf} ~
  \includegraphics[width=0.42\linewidth]{figures/2016_Wln/Wlf_high_Wln__Hj0_pt_.pdf}
  \caption[Control region plots for one lepton in 2016]{
    The $p_T$ of the jet with a higher deepCSV value
    is plotted for multiple control regions and bins in
    an inclusive selection of both the $W\rightarrow \mu\nu$ and $W\rightarrow e\nu$
    one lepton channels.
    The top row shows the $t\bar{t}$ control region,
    the middle row shows the $W$+ heavy jets,
    and the bottom row shows the $W$+ light jets control region.
    The left column shows the medium $p_T(V)$ bin of 150 to \SI{250}{GeV},
    and the right column shows the high $p_T(V)$ bin of \SI{250}{GeV} and above.
  }
  \label{fig:2016_Wln_j0pt}
\end{figure}

\begin{figure}
  \centering
  \includegraphics[width=0.42\linewidth]{figures/2017_Znn/ttbar_med_Znn__Hj0_DeepCSV_.pdf} ~
  \includegraphics[width=0.42\linewidth]{figures/2017_Znn/ttbar_high_Znn__Hj0_DeepCSV_.pdf}
  \includegraphics[width=0.42\linewidth]{figures/2017_Znn/Zhf_med_Znn__Hj0_DeepCSV_.pdf} ~
  \includegraphics[width=0.42\linewidth]{figures/2017_Znn/Zhf_high_Znn__Hj0_DeepCSV_.pdf}
  \includegraphics[width=0.42\linewidth]{figures/2017_Znn/Zlf_med_Znn__Hj0_DeepCSV_.pdf} ~
  \includegraphics[width=0.42\linewidth]{figures/2017_Znn/Zlf_high_Znn__Hj0_DeepCSV_.pdf}
  \caption[Control region plots for zero lepton in 2017]{
    The deepCSV value of the jet with a higher deepCSV value
    is plotted for multiple control regions and bins in
    an inclusive selection of both the $Z\rightarrow \mu\nu$ and $Z\rightarrow e\nu$
    one lepton channels.
    The top row shows the $t\bar{t}$ control region,
    the middle row shows the $Z$+ heavy jets,
    and the bottom row shows the $Z$+ light jets control region.
    The left column shows the medium $p_T(V)$ bin of 150 to \SI{250}{GeV},
    and the right column shows the high $p_T(V)$ bin of \SI{250}{GeV} and above.
  }
  \label{fig:2017_Znn_j0csv}
\end{figure}


\section{Multivariate Discriminator}

In each STXS bin, a multivariate discriminator is plotted which separates
the signal events from background events.
A Deep Neural Network (DNN) is trained for the resolved selection,
and a Boosted Decision Tree (BDT) is trained for the boosted selection.
Using fewer discriminating variables in the boosted events
leads to this difference in architecture.

\subsection{Resolved DNN}

The DNN classifier for distinguishing background and signal events is prepared using
Keras with a Tensorflow backend using an Adam optimizer.
It has five hidden layers.
The number of nodes in each layer, from input to output, is 512, 256, 128, 64, 64, and 64.
The final layer is a softmax layer with the target of predicting the probability
of each event belonging to a particular class.

Each channel of 0-, 1-, and 2-leptons is trained separately,
and has slightly different input variables.
The list of input variables is given in Table~\ref{tab:dnn-inputs}.
All variables that are affected by the kinematic fit in the 2-lepton region
use the values calculated by the fit.

\begin{table}
  \caption[Resolved DNN inputs]{
    The list of input variables used for each DNN training is shown.
  }
  \centering
  \begin{tabularx}{\textwidth}{|l|X|c|c|c|}
    \hline
    Variable & Explanation & 0-lepton & 1-lepton & 2-lepton \\
    \hline\hline
    $m_{jj}$ & Di-jet mass & $\checkmark$ & $\checkmark$ & $\checkmark$ \\
    $p_{T,jj}$ & Di-jet transverse momentum & $\checkmark$ & $\checkmark$ & $\checkmark$ \\
    MET & Missing transverse energy & $\checkmark$ & $\checkmark$ & $\checkmark$ \\
    \hline
    $m_{T,V}$ & Vector boson transverse mass & & $\checkmark$ & \\
    $p_T(V)$ & Vector boson $p_T$ & & $\checkmark$ & $\checkmark$ \\
    $p_{T,jj}/p_T(V)$ & Redundant ratio & & $\checkmark$ & $\checkmark$ \\
    \hline
    $\Delta\phi(V, jj)$ & Azimuthal angle between vector boson and di-jet & $\checkmark$ & $\checkmark$ & $\checkmark$ \\
    $b$-tag$_\mathrm{max}$ WP & 1, 2, or 3 if higher $b$-tag discriminate meets the tight, medium, or loose working point respectively & $\checkmark$ & $\checkmark$ & $\checkmark$ \\
    $b$-tag$_\mathrm{min}$ WP & 1, 2, or 3 if lower $b$-tag discriminate meets the tight, medium, or loose working point respectively & $\checkmark$ & $\checkmark$ & $\checkmark$ \\
    \hline
    $\Delta\eta(jj)$ & $\eta$ difference between jets & $\checkmark$ & $\checkmark$ & $\checkmark$ \\
    $\Delta\phi(jj)$ & Azimuthal angle between jets & $\checkmark$ & $\checkmark$ & \\
    $p_{T, \mathrm{lead}}$ & Leading jet $p_T$ & $\checkmark$ & $\checkmark$ & $\checkmark$ \\
    \hline
    $p_{T, \mathrm{trail}}$ & Trailing jet $p_T$ & $\checkmark$ & $\checkmark$ & $\checkmark$ \\
    SA5 & Number of soft jets, $p_T > \SI{5}{GeV}$ & $\checkmark$ & $\checkmark$ & $\checkmark$ \\
    N_{aj} & Number of additional jets & $\checkmark$ & $\checkmark$ & \\
    \hline
    $b$-tag$_\mathrm{add}$ & Maximum $b$-tag of additional jets & $\checkmark$ & & \\
    $p_{T,\mathrm{add}}$ & Maximum $p_T$ of additional jets & $\checkmark$ & & \\
    $\Delta\phi(\mathrm{add, MET})$ & Azimuthal angle between additional jet and MET & $\checkmark$ & & \\
    \hline
    $\Delta\phi(\ell, \mathrm{MET})$ & Azimuthal angle between lepton and MET & & $\checkmark$ & \\
    $m_t$ & Reconstruction top mass & & $\checkmark$ & \\
    $m_V$ & Vector boson mass & & & $\checkmark$ \\
    \hline
    $\Delta R(V, jj)$ & Separation between vector boson and di-jet & & & $\checkmark$ \\
    $\Delta R_{jj}$ & Separation between jets & & & $\checkmark$ \\
    \hline
  \end{tabularx}
  \label{tab:dnn-inputs}
\end{table}

\subsection{Boosted BDT}

The BDT used to classify signal and background events in the boosted region
was trained using ROOT.
The model uses 100 trees with 20 cuts and a minimum node size of 0.05.
The QCD multi-jet backgrounds were not used in the training since the sample's large weights
of individual events affected the training.

The list of input variables for the BDT is the following:
\begin{itemize}
\item Soft-drop mass of the reconstructed fat jet
\item Transverse momentum of the fat jet
\item Transverse momentum of the reconstructed vector boson
\item Number of soft-track jets with $p_T > \SI{5}{GeV}$
\item Double $b$-tagger output node for boosted jets
\end{itemize}
All of these variables were all used in the 0-, 1-, and 2-lepton regions,
even though the regions were trained separately.

\section{Combination Fit}

A simultaneous fit is run over all channels, control regions, and the signal selection region
in order to determine the most likely values for all
parameters with systematic uncertainties, called nuisance parameters,
as well as the most likely scale factors for all the MC backgrounds and signal.
The fit is done by using the \texttt{combine} tool \cite{cmsdocumentation} as part of
\texttt{CMSSW\_10\_2\_13} \cite{cmssw_doxygen}.
It is commonly used in searches of new physics processes within the CMS collaboration.
Part of the method for discovery involves finding the most likely values
for all nuisance parameters and signal strength simultaneously.
This maximum likelihood feature is used for the measurement of signal strength in
the STXS bins.

In addition to the most likely nuisance parameter and scale factor values,
the output of \texttt{combine} includes likelihood values at
multiple signal strengths with and without variation of
nuisance parameters from their most likely values.
This allows separation of systematic and statistical uncertainties from the global fit.
The \texttt{combine} tool also evaluated the impacts of each individual
nuisance parameter on the likelihood by varying them individually.

\subsection{$V\!Z$ Cross Check Analysis}

\begin{figure}
  \centering
  \includegraphics[width=0.8\linewidth]{figures/201118_inclVZ/scan_nominal_r.pdf}
  \caption[Inclusive likelihood scan of $V\!Z$]{
    The likelihood scan of the inclusive signal strength of the
    $V\!Z$ cross check analysis.
  }
  \label{fig:vz-inclusive}
\end{figure}

\begin{figure}
  \centering
  \includegraphics[width=0.8\linewidth]{figures/201118_STXS_VZ/summary_stxs.pdf}
  \caption[Measured STXS values of $V\!Z$]{
    The measured most likely values of all STXS bins in the
    $V\!Z$ cross check analysis.
  }
  \label{fig:vz_stxs}
\end{figure}
