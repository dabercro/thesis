\chapter{Analysis Results} \label{ch:results}

With the selection and correction procedure outlined in the previous chapters,
data gathered at CMS is gathered and compared to predictions of the Standard Model.
Before looking at the signal events, various distributions are plotted
in each of the control regions to ensure that the background events are simulated accurately.
After some example distributions, discriminating variables to be used in the signal selections are defined.
Right before getting the final measurement, a cross-check analysis is done using the same techniques
on the more common $V\!Z$ process.

\section{Run 2 Data Collection}

The CMS detector collected proton-proton collision data at $\sqrt{s} = \SI{13}{TeV}$
over three years during Run 2 of the LHC.
In 2016, CMS collected \SI{35.9}{fb^{-1}} of data \cite{CMS-PAS-LUM-17-001}.
In 2017, data corresponding to \SI{41.5}{fb^{-1}} were collected \cite{CMS-PAS-LUM-17-004}.
In 2018, CMS collected \SI{60}{fb^{-1}} of collision data \cite{CMS-PAS-LUM-18-002}.
This data, totaling about \SI{137}{fb^{-1}} was used for the analysis.

As noted in defining the MET selection of 2018, the CMS detector performance does
change over time as the systems are exposed to radiation from the LHC environment.
Therefore, each of the three years is treated separately with many systematic uncertainties related to the
detector uncorrelated between the years.
Even though these experimental values float separately, the final result is that of a combination of all three years.

\section{Control Region Plots}

A number of variables are compared in all of the control regions
to ensure that the data and simulation are in reasonable agreement.
A selection of these plots are shown in Figures~\ref{fig:2016_Zll_vpt} through \ref{fig:2018_Zmm_vpt}.
%,fig:2016_Wln_j0pt,fig:2017_Znn_j0csv,fig:2017_Zee_j1pt,fig:2018_Wln_j1csv,
These are the distributions before fitting the control and signal regions to data,
so agreements improve after the fit.
The number of variables, selections, and separation of the data collection years
means that this is just a fraction of the total number of plots produced.
However, this subset shown demonstrate several discrepancies that needed to be accounted for in our systematics,
such as the $p_T(V)$ trend which can be seen in Figure~\ref{fig:2016_Zll_vpt},
the $p_T$ of individual jets in Figures~\ref{fig:2016_Wln_j0pt} and \ref{fig:2017_Zee_j1pt},
and normalization of certain background samples, like the $Z$ + jets samples in Figure~ \ref{fig:2018_Zmm_vpt}.
Figure~\ref{fig:2017_Znn_j0csv} shows how difficult it is to model the $b$-tagging variables.
Systematic uncertainties are applied to all of these variables to account for the modeling difficulties,
but the plots in  Figures~\ref{fig:2016_Zll_vpt} through \ref{fig:2018_Zmm_vpt} only show the statistical
uncertainties in the simulation.

\begin{figure}
  \centering
  \includegraphics[width=0.42\linewidth]{figures/2016_Zll/ttbar_low_Zll__kinFit_V_pt_.pdf} ~
  \includegraphics[width=0.42\linewidth]{figures/2016_Zll/ttbar_med_Zll__kinFit_V_pt_.pdf}
  \includegraphics[width=0.42\linewidth]{figures/2016_Zll/Zhf_low_Zll__kinFit_V_pt_.pdf} ~
  \includegraphics[width=0.42\linewidth]{figures/2016_Zll/Zhf_med_Zll__kinFit_V_pt_.pdf}
  \includegraphics[width=0.42\linewidth]{figures/2016_Zll/Zlf_low_Zll__kinFit_V_pt_.pdf} ~
  \includegraphics[width=0.42\linewidth]{figures/2016_Zll/Zlf_med_Zll__kinFit_V_pt_.pdf}
  \caption[Control region plots for two leptons in 2016]{
    The vector boson $p_T$ is plotted for multiple control regions and bins in
    an inclusive selection of both the $Z\rightarrow \mu\mu$ and $Z\rightarrow ee$
    two lepton channels with 2016 data.
    The top row shows the $t\bar{t}$ control region,
    the middle row shows the $Z$+ heavy jets,
    and the bottom row shows the $Z$+ light jets control region.
    The left column shows the low $p_T(V)$ bin of 75 to \SI{150}{GeV},
    and the right column shows the medium $p_T(V)$ bin of 150 to \SI{250}{GeV}.
  }
  \label{fig:2016_Zll_vpt}
\end{figure}

\begin{figure}
  \centering
  \includegraphics[width=0.42\linewidth]{figures/2016_Wln/ttbar_med_Wln__Hj0_pt_.pdf} ~
  \includegraphics[width=0.42\linewidth]{figures/2016_Wln/ttbar_high_Wln__Hj0_pt_.pdf}
  \includegraphics[width=0.42\linewidth]{figures/2016_Wln/Whf_med_Wln__Hj0_pt_.pdf} ~
  \includegraphics[width=0.42\linewidth]{figures/2016_Wln/Whf_high_Wln__Hj0_pt_.pdf}
  \includegraphics[width=0.42\linewidth]{figures/2016_Wln/Wlf_med_Wln__Hj0_pt_.pdf} ~
  \includegraphics[width=0.42\linewidth]{figures/2016_Wln/Wlf_high_Wln__Hj0_pt_.pdf}
  \caption[Control region plots for one lepton in 2016]{
    The $p_T$ of the jet with a higher DeepCSV value
    is plotted for multiple control regions and bins in
    an inclusive selection of both the $W\rightarrow \mu\nu$ and $W\rightarrow e\nu$
    one lepton channels with 2016 data.
    The top row shows the $t\bar{t}$ control region,
    the middle row shows the $W$+ heavy jets,
    and the bottom row shows the $W$+ light jets control region.
    The left column shows the medium $p_T(V)$ bin of 150 to \SI{250}{GeV},
    and the right column shows the high $p_T(V)$ bin of \SI{250}{GeV} and above.
  }
  \label{fig:2016_Wln_j0pt}
\end{figure}

\begin{figure}
  \centering
  \includegraphics[width=0.42\linewidth]{figures/2017_Znn/ttbar_med_Znn__Hj0_DeepCSV_.pdf} ~
  \includegraphics[width=0.42\linewidth]{figures/2017_Znn/ttbar_high_Znn__Hj0_DeepCSV_.pdf}
  \includegraphics[width=0.42\linewidth]{figures/2017_Znn/Zhf_med_Znn__Hj0_DeepCSV_.pdf} ~
  \includegraphics[width=0.42\linewidth]{figures/2017_Znn/Zhf_high_Znn__Hj0_DeepCSV_.pdf}
  \includegraphics[width=0.42\linewidth]{figures/2017_Znn/Zlf_med_Znn__Hj0_DeepCSV_.pdf} ~
  \includegraphics[width=0.42\linewidth]{figures/2017_Znn/Zlf_high_Znn__Hj0_DeepCSV_.pdf}
  \caption[Control region plots for zero lepton in 2017]{
    The DeepCSV value of the jet with a higher DeepCSV value
    is plotted for multiple control regions and bins in
    the $Z\rightarrow \nu\bar{\nu}$ zero lepton channel with 2017 data.
    The top row shows the $t\bar{t}$ control region,
    the middle row shows the $Z$+ heavy jets,
    and the bottom row shows the $Z$+ light jets control region.
    The left column shows the medium $p_T(V)$ bin of 150 to \SI{250}{GeV},
    and the right column shows the high $p_T(V)$ bin of \SI{250}{GeV} and above.
  }
  \label{fig:2017_Znn_j0csv}
\end{figure}

\begin{figure}
  \centering
  \includegraphics[width=0.42\linewidth]{figures/2017_Zee/ttbar_low_Zee__Hj1_pt_reg_.pdf} ~
  \includegraphics[width=0.42\linewidth]{figures/2017_Zee/ttbar_med_Zee__Hj1_pt_reg_.pdf}
  \includegraphics[width=0.42\linewidth]{figures/2017_Zee/Zhf_low_Zee__Hj1_pt_reg_.pdf} ~
  \includegraphics[width=0.42\linewidth]{figures/2017_Zee/Zhf_med_Zee__Hj1_pt_reg_.pdf}
  \includegraphics[width=0.42\linewidth]{figures/2017_Zee/Zlf_low_Zee__Hj1_pt_reg_.pdf} ~
  \includegraphics[width=0.42\linewidth]{figures/2017_Zee/Zlf_med_Zee__Hj1_pt_reg_.pdf}
  \caption[Control region plots for two electrons in 2017]{
    The $p_T$ of the jet with a lower DeepCSV score is plotted for
    multiple control regions and bins in
    the $Z\rightarrow ee$ two lepton channel with 2017 data.
    The top row shows the $t\bar{t}$ control region,
    the middle row shows the $Z$+ heavy jets,
    and the bottom row shows the $Z$+ light jets control region.
    The left column shows the low $p_T(V)$ bin of 75 to \SI{150}{GeV},
    and the right column shows the medium $p_T(V)$ bin of 150 to \SI{250}{GeV}.
  }
  \label{fig:2017_Zee_j1pt}
\end{figure}

\begin{figure}
  \centering
  \includegraphics[width=0.42\linewidth]{figures/2018_Wln/ttbar_medhigh_Wen__Hj1_DeepCSV_.pdf} ~
  \includegraphics[width=0.42\linewidth]{figures/2018_Wln/ttbar_medhigh_Wmn__Hj1_DeepCSV_.pdf}
  \includegraphics[width=0.42\linewidth]{figures/2018_Wln/Whf_medhigh_Wen__Hj1_DeepCSV_.pdf} ~
  \includegraphics[width=0.42\linewidth]{figures/2018_Wln/Whf_medhigh_Wmn__Hj1_DeepCSV_.pdf}
  \includegraphics[width=0.42\linewidth]{figures/2018_Wln/Wlf_medhigh_Wen__Hj1_DeepCSV_.pdf} ~
  \includegraphics[width=0.42\linewidth]{figures/2018_Wln/Wlf_medhigh_Wmn__Hj1_DeepCSV_.pdf}
  \caption[Control region plots for one lepton in 2018]{
    The DeepCSV value of the jet with a higher DeepCSV value
    is plotted for multiple control regions in
    an inclusive selection that contains all STXS bins with
    $p_T(V)$ greater than \SI{150}{GeV}
    in the one lepton channels with 2018 data.
    The top row shows the $t\bar{t}$ control region,
    the middle row shows the $W$+ heavy jets,
    and the bottom row shows the $W$+ light jets control region.
    The left column shows the  $W\rightarrow e\nu$ channel,
    and the right column shows the $W\rightarrow \mu\nu$ channel.
  }
  \label{fig:2018_Wln_j1csv}
\end{figure}

\begin{figure}
  \centering
  \includegraphics[width=0.42\linewidth]{figures/2018_Zmm/ttbar_low_Zmm__Vpt_.pdf} ~
  \includegraphics[width=0.42\linewidth]{figures/2018_Zmm/ttbar_medhigh_Zmm__Vpt_.pdf}
  \includegraphics[width=0.42\linewidth]{figures/2018_Zmm/Zhf_low_Zmm__Vpt_.pdf} ~
  \includegraphics[width=0.42\linewidth]{figures/2018_Zmm/Zhf_medhigh_Zmm__Vpt_.pdf}
  \includegraphics[width=0.42\linewidth]{figures/2018_Zmm/Zlf_low_Zmm__Vpt_.pdf} ~
  \includegraphics[width=0.42\linewidth]{figures/2018_Zmm/Zlf_medhigh_Zmm__Vpt_.pdf}
  \caption[Control region plots for two muons in 2018]{
    The vector boson $p_T$ is plotted for multiple control regions and bins in
    the $Z\rightarrow \mu\mu$ two lepton channel with 2018 data.
    The top row shows the $t\bar{t}$ control region,
    the middle row shows the $Z$+ heavy jets,
    and the bottom row shows the $Z$+ light jets control region.
    The left column shows the low $p_T(V)$ bin of 75 to \SI{150}{GeV},
    and the right column shows an inclusive selection of
    the medium and high $p_T(V)$ bins of greater than \SI{150}{GeV}.
  }
  \label{fig:2018_Zmm_vpt}
\end{figure}

\section{Multivariate Discriminator}

In each STXS bin, a multivariate discriminator is evaluated for the signal region which separates
the signal events from background events.
A Deep Neural Network (DNN) is trained for the resolved selection,
and a Boosted Decision Tree (BDT) is trained for the boosted selection.
The DNN was also used to train a multi-classifier for use in the resolved $Z$ + heavy jets control region.
Half of the simulated events are used in these trainings.
To prevent biases, as well as to avoid over-training, the other half of the simulated events are used
to compare classifier distributions with data.

\subsection{Resolved DNN}

The DNN classifier for distinguishing background and signal events is prepared using
Keras with a Tensorflow back-end using an Adam optimizer.
It has five hidden layers.
The number of nodes in each layer, from input to output, is 512, 256, 128, 64, 64, and 64.
The final layer is a softmax layer with the target of predicting the probability
of each event belonging to a particular class.
The cross entropy loss function is used for the training.

A separate training using the same architecture is performed on the $V$ + heavy jets control regions.
Each class in the output is designed to control the normalization and input variable shapes
of different background processes.
There are five classes that attempt to separate the following processes:
\begin{itemize}
\item $V$ + light-flavored ($udsg$) jets
\item $V$ + $c$ jets
\item $V$ + $b$ jets
\item Single-top events
\item $t\bar{t}$ events
\end{itemize}

Each channel of 0-, 1-, and 2-leptons is trained separately,
and has slightly different input variables.
The list of input variables is given in Table~\ref{tab:dnn-inputs}.
All variables that are affected by the kinematic fit in the 2-lepton region
use the values calculated by the fit.

\begin{table}
  \caption[Resolved DNN inputs]{
    The list of input variables used for each DNN training is shown.
  }
  \centering
  \begin{tabularx}{\textwidth}{|l|X|c|c|c|}
    \hline
    Variable & Explanation & 0-lepton & 1-lepton & 2-lepton \\
    \hline\hline
    $m_{jj}$ & Di-jet mass & $\checkmark$ & $\checkmark$ & $\checkmark$ \\
    $p_{T,jj}$ & Di-jet transverse momentum & $\checkmark$ & $\checkmark$ & $\checkmark$ \\
    MET & Missing transverse energy & $\checkmark$ & $\checkmark$ & $\checkmark$ \\
    \hline
    $m_{T,V}$ & Vector boson transverse mass & & $\checkmark$ & \\
    $p_T(V)$ & Vector boson $p_T$ & & $\checkmark$ & $\checkmark$ \\
    $p_{T,jj}/p_T(V)$ & Redundant ratio & & $\checkmark$ & $\checkmark$ \\
    \hline
    $\Delta\phi(V, jj)$ & Azimuthal angle between vector boson and di-jet & $\checkmark$ & $\checkmark$ & $\checkmark$ \\
    $b$-tag$_\mathrm{max}$ WP & 1, 2, or 3 if higher $b$-tag discriminate meets the tight, medium, or loose working point respectively & $\checkmark$ & $\checkmark$ & $\checkmark$ \\
    $b$-tag$_\mathrm{min}$ WP & 1, 2, or 3 if lower $b$-tag discriminate meets the tight, medium, or loose working point respectively & $\checkmark$ & $\checkmark$ & $\checkmark$ \\
    \hline
    $\Delta\eta(jj)$ & $\eta$ difference between jets & $\checkmark$ & $\checkmark$ & $\checkmark$ \\
    $\Delta\phi(jj)$ & Azimuthal angle between jets & $\checkmark$ & $\checkmark$ & \\
    $p_{T, \mathrm{lead}}$ & Leading jet $p_T$ & $\checkmark$ & $\checkmark$ & $\checkmark$ \\
    \hline
    $p_{T, \mathrm{trail}}$ & Trailing jet $p_T$ & $\checkmark$ & $\checkmark$ & $\checkmark$ \\
    SA5 & Number of soft jets, $p_T > \SI{5}{GeV}$ & $\checkmark$ & $\checkmark$ & $\checkmark$ \\
    $N_{aj}$ & Number of additional jets & $\checkmark$ & $\checkmark$ & \\
    \hline
    $b$-tag$_\mathrm{add}$ & Maximum $b$-tag of additional jets & $\checkmark$ & & \\
    $p_{T,\mathrm{add}}$ & Maximum $p_T$ of additional jets & $\checkmark$ & & \\
    $\Delta\phi(\mathrm{add, MET})$ & Azimuthal angle between additional jet and MET & $\checkmark$ & & \\
    \hline
    $\Delta\phi(\ell, \mathrm{MET})$ & Azimuthal angle between lepton and MET & & $\checkmark$ & \\
    $m_t$ & Reconstruction top mass & & $\checkmark$ & \\
    $m_V$ & Vector boson mass & & & $\checkmark$ \\
    \hline
    $\Delta R(V, jj)$ & Separation between vector boson and di-jet & & & $\checkmark$ \\
    $\Delta R_{jj}$ & Separation between jets & & & $\checkmark$ \\
    \hline
  \end{tabularx}
  \label{tab:dnn-inputs}
\end{table}

\subsection{Boosted BDT}

The BDT used to classify signal and background events in the boosted region
was trained using ROOT.
The model uses 100 trees with 20 cuts and a minimum node size of 0.05.
The QCD multi-jet backgrounds were not used in the training since the sample's large weights
of individual events affected the training.

The list of input variables for the BDT is the following:
\begin{itemize}
\item Soft-drop mass of the reconstructed fat jet
\item Transverse momentum of the fat jet
\item Transverse momentum of the reconstructed vector boson
\item Number of soft-track jets with $p_T > \SI{5}{GeV}$
\item Double $b$-tagger output node for boosted jets
\end{itemize}
All of these same variables were used in the 0-, 1-, and 2-lepton regions,
even though the regions were trained separately.
Because of the decreased complexity of the boosted BDT compared to the DNN,
there is no need to attempt to simultaneously control the input variables from the backgrounds
in the $V$ + heavy flavor control regions.

\section{Combination Fit}

A simultaneous fit is run over all channels, control regions, and the signal selection region
in order to determine the most likely values for all
parameters with systematic uncertainties, called nuisance parameters,
as well as the most likely scale factors for all the MC backgrounds and signal.
The fit is done by using the \texttt{combine} tool \cite{cmsdocumentation} as part of
\texttt{CMSSW\_10\_2\_13} \cite{cmssw_doxygen}.
Included in this fit is the strength of the signal in the STXS bins.
This is how the final result of this analysis is measured.

There are a total of 243 distributions that make up the fit.
Table~\ref{tab:num-hists} shows how the different channels and control regions contribute to this number.
%
\begin{table}
  \caption[Counts of distributions for each selection]{
    Below, a table indicates the distributions from each control region (CR) and signal region (SR) selection.
    Each of the three control regions contributes the same number of distributions to the fit.
    The channels containing charged leptons can either be electron or muon flavored.
    These two considerations determine the multipliers shown in the table.
  }
  \begin{tabular}{|c|c|c|c|c|c|c|c|c|c|c|c|c|}
    \hline
    \multirow{3}{*}{$p_T(V)$ [GeV]} & \multicolumn{4}{c|}{0-leptons} & \multicolumn{4}{c|}{1-lepton} & \multicolumn{4}{c|}{2-leptons} \\
    & \multicolumn{2}{c|}{Resolved} & \multicolumn{2}{c|}{Boosted} & \multicolumn{2}{c|}{Resolved} & \multicolumn{2}{c|}{Boosted} & \multicolumn{2}{c|}{Resolved} & \multicolumn{2}{c|}{Boosted} \\
    & CR & SR & CR & SR & CR & SR & CR & SR & CR & SR & CR & SR \\
    \hline
    75 -- 150     &   &   &   &   &   &   &   &   & X & X &   &   \\
    \hline
    150 -- 250    & X &   &   &   & X & X &   &   & X &   &   &   \\
    with jet      &   & X &   &   &   &   &   &   &   & X &   &   \\
    without       &   & X &   &   &   &   &   &   &   & X &   &   \\
    \hline
    250 -- $\infty$ & X &   & X &   & X &   & X &   & X &   & X &   \\
    250 -- 400      &   & X &   & X &   & X &   & X &   & X &   & X \\
    400 -- $\infty$ &   & X &   & X &   & X &   & X &   & X &   & X \\
    \hline
    number X      & 2 & 4 & 1 & 2 & 2 & 3 & 1 & 2 & 3 & 5 & 1 & 2 \\
    multiplier    & 3 & 1 & 3 & 1 & 6 & 2 & 6 & 2 & 6 & 2 & 6 & 2 \\
    regions/year  & 6 & 4 & 3 & 2 & 12 & 6 & 6 & 4 & 18 & 10 & 6 & 4 \\
   \hline
  \end{tabular}
  \label{tab:num-hists}
\end{table}
%
As mentioned before, there are a number of uncertainties that
also produce variations in the fit histograms.
A separate fit is run with each corresponding nuisance parameter frozen to their most likely value,
which gives the purely statistical uncertainty on the measurement result.
The systematic uncertainty is then extracted.

Running this fit without data, but instead toy distributions gathered directly
from the initial simulated distributions gives the expected sensitivity of the analysis.
Analyzers can see the statistical and systematic uncertainties before measuring the final result in this way.
It is under this limitation that the analysis strategy is developed and optimized for sensitivity.
For Higgs analyses, there is an additional measurement which is done before the fit is run
with measured data to give the unblinded result.
In addition to the ability to fit distributions , \texttt{combine} includes a variety of
diagnostic tools to ensure the assumed model is adequate.
Before unblinding results, multiple toy distributions are generated using the best-fit
nuisance parameter values and comparing a test statistic,
which is of a generalized $\chi^2$ test \cite{cousins2013generalization},
of these toys to that of the observed data.
The distributions of the values from the set of toys allows the evaluation of the $p$-value
of the test statistic observed in data.
The \texttt{combine} tool also evaluates the impacts of each
nuisance parameter on the likelihood by varying them individually.
If nuisance parameters with large impacts are pulled far from their initial value,
that means that an important aspect of the measurement was not well understood.
Both of these tests can be run without revealing the measured values of cross sections.
This prevents biases in the analysis strategy which is finalized in response to these tests.

The final results will therefore show comparison of data and best-fit toys,
called Goodness of Fit, as well as the final scan results.
Post-fit plots will also reveal the way that the fit moves the background and signal distributions
within uncertainties in order for the background to describe the data.
Impacts plots are shown in Appendix~\ref{app:impacts} due to the high multiplicity of nuisance parameters.
Before running these checks and measurements for the $V\!H$ measurement,
they were first tested on $V\!Z$.

\subsection{Test of Methodology Using a $V\!Z$ Selection}

Before unblinding the $V\!H$ signal region and results, the techniques used in the analysis are
verified by performing a similar analysis.
The process $V\!Zb\bar{b}$, where a vector boson radiates a $Z$ boson that decays into $b$ quarks,
looks very similar to the $V\!Hb\bar{b}$ process.
The di-jet mass is just in a different location.
The $V\!Z$ cross check analysis uses a strategy that is similar to what has been presented so far.
The primary differences are that a di-boson sample with generator-level $b$ hadrons produced
is used as the signal sample, and the signal mass window is moved from $\SI{90}{GeV} < m_{jj} < \SI{150}{GeV}$
to  $\SI{60}{GeV} < m_{jj} < \SI{120}{GeV}$.
However, other differences should be mentioned that may affect the final result.
Additional measured electroweak corrections are applied to the di-boson samples for each STXS bin.
The values of these corrections vary from as low as 4\% for the low $p_T(V)$ bins
to as high as 23\% for the high $p_T(V)$, but all will increase the measured signal strength compared to $V\!H$.
Other differences, the effect of which are more difficult to quantify, include the dedicated $V\!Z$ DNN training and
the fact that the $Z$ boson will decay into light flavored jets at a higher proportional rate than the Higgs boson.
The DNN training for $V\!H$ has to account for the similar $V\!Z$ background,
while the $V\!Z$ is not influenced as much by the rare $V\!H$ process.
Also, the light flavor $V\!Z$ process will have more influence over
the control region fits than the $V\!H$ process has.

Data from all three years of Run 2 are used so that any differences that may affect the $V\!H$ analysis
can also be investigated using $V\!Z$.
The Goodness of Fit plots are shown in Figure~\ref{fig:vz-gof}.
The high $p$-value means that the simulated processes were able to match the data very well
while staying within the fit constraints of the systematic uncertainties.
This suggests that the systematic uncertainties applied may be much larger than needed for this cross check analysis.

\begin{figure}
  \centering
  \includegraphics[width=0.65\linewidth]{figures/210308_STXS_VZ_XbbVZ_e4179c95_inclusive_gof/Gof_inclusive_.pdf}
  \caption[Goodness of Fit for $V\!Z$]{
    The Goodness of Fit test results are shown for $V\!Z$.
    The fit is performed using STXS bins,
    and the test statistic is generated by toys for all control regions and signal regions.
    The high $p$-value indicates that the model may be over-fitting to data due to large systematic uncertainties.
    This is not the case for the $V\!H$ analysis.
  }
  \label{fig:vz-gof}
\end{figure}

Due to the splitting of STXS bins, the multiple channels for each bin, and the combination of three different years,
243 distributions are fit in the $V\!Z$ cross check analysis.
For the sake of brevity, the effect of the fit will only be shown on eight of these distributions for the $V\!Z$
cross-check analysis.
These are split into two figures, each showing distributions closely related to each other.
On the left of each figure are simulation values and uncertainties
before the nuisance parameters are fit to match the simulation to data.
On the right are the simulated values after fitting the nuisance parameters.
In the lower parts of each plot, ratios between observed data and expected yields improve as a result of the fit.
The uncertainties of the simulations also decrease as a result of the fit.
Some nuisance parameters end up with smaller individual uncertainties from the fit,
including the initially large migration uncertainties.
Figure~\ref{fig:impact-ex} shows this decrease in bin migration uncertainties,
especially in $t\bar{t}$ in line 19.
%
\begin{figure}
  \centering
  \includegraphics[width=0.8\linewidth,page=1]{figures/impacts/impacts_r_whhi1.pdf}
  \caption[Example impacts for WH $250 < p_T(V) < \SI{400}{GeV}$]{
    The most significant impacts in on WH $250 < p_T(V) < \SI{400}{GeV}$.
    Note the decrease in migration uncertainties, such as \texttt{CMS\_vhbb\_Vpt250\_TT\_1lep\_13TeV2016}.
  }
  \label{fig:impact-ex}
\end{figure}
%
The overall uncertainty also decreases slightly since each named nuisance parameter is at first
assumed to be uncorrelated to the other nuisance parameters.
The fit reveals many of the nuisance parameters to be correlated, decreasing the overall possible variation.

Figure~\ref{fig:vz-zmm-low-2017} shows all of the distributions fit for the low $p_T$ bin in the
$Z\rightarrow\mu\mu$ channel in 2017.
Several additional features aside from the pre-fit and post-fit differences are worth noting.
There is only one bin in the $t\bar{t}$ control region,
so it is only used to constrain the normalization of the $t\bar{t}$ background.
In the $Z$ + heavy jets control region,
the multi-classifier DNN  with the same inputs as the signal region DNN is shown.
The signal distributions are binned so that the output of the DNN is flat in the $V\!H$ signal.
The $V\!Z$ signal distribution is close to flat when following the same binning scheme.
The important feature in that distribution is that the backgrounds are all falling as the DNN value increases.
In all of the regions, we see improved agreement between the simulation and data after the fit, and
the combination fit also decreases the uncertainty on each bin.
The relative contributions of the main background processes can also be seen at a glance at the signal region plots.
%
\begin{figure}
  \centering
  \includegraphics[width=0.35\linewidth]{figures/210414_STXS_VZ_unblinded_XbbVZ_e4179c95_postfitplots/plot_shapes_vhbb_Zmm_2_13TeV2017_prefit}
  \includegraphics[width=0.35\linewidth]{figures/210414_STXS_VZ_unblinded_XbbVZ_e4179c95_postfitplots/plot_shapes_vhbb_Zmm_2_13TeV2017_postfit} \\
  \includegraphics[width=0.35\linewidth]{figures/210414_STXS_VZ_unblinded_XbbVZ_e4179c95_postfitplots/plot_shapes_vhbb_Zmm_3_13TeV2017_prefit}
  \includegraphics[width=0.35\linewidth]{figures/210414_STXS_VZ_unblinded_XbbVZ_e4179c95_postfitplots/plot_shapes_vhbb_Zmm_3_13TeV2017_postfit} \\
  \includegraphics[width=0.35\linewidth]{figures/210414_STXS_VZ_unblinded_XbbVZ_e4179c95_postfitplots/plot_shapes_vhbb_Zmm_4_13TeV2017_prefit}
  \includegraphics[width=0.35\linewidth]{figures/210414_STXS_VZ_unblinded_XbbVZ_e4179c95_postfitplots/plot_shapes_vhbb_Zmm_4_13TeV2017_postfit} \\
  \includegraphics[width=0.35\linewidth]{figures/210414_STXS_VZ_unblinded_XbbVZ_e4179c95_postfitplots/plot_shapes_vhbb_Zmm_1_13TeV2017_prefit_logy}
  \includegraphics[width=0.35\linewidth]{figures/210414_STXS_VZ_unblinded_XbbVZ_e4179c95_postfitplots/plot_shapes_vhbb_Zmm_1_13TeV2017_postfit_logy} \\
  \caption[$Z\rightarrow \mu\mu$ $V\!Z$ distributions for low $p_T$ in 2017]{
    Above are $Z\rightarrow \mu\mu$ pre-fit (left) and post-fit distributions (right)
    for the low $p_T$ STXS bin in 2017 in the $V\!Z$ cross-check analysis.
    The top row shows the $Z$ + light jets control region,
    the second row shows the $Z$ + heavy jets control region,
    and the third row shows the $t\bar{t}$ control region.
    The bottom row shows the signal region.
  }
  \label{fig:vz-zmm-low-2017}
\end{figure}
%
These distributions can be compared to those in Figure~\ref{fig:vz-wen-boost-2018},
which shows distributions from the boosted selection of the $W\rightarrow e\nu$ channel in 2018.
In this case, the $W$ + heavy jets control region is not shown because the related process contributes less
in the signal region.
Instead, the $t\bar{t}$ background is dominant in this boosted topology.
The control regions use the double $b$-tag score of the fat jet instead of $p_T(V)$ for the distribution.
The signal region is also split into two higher $p_T(V)$ bins, while the control regions stay combined
to constrain both STXS bins.
As in the low $p_T$ bin, though, both agreement between simulation and data and estimated uncertainties
are improved as a result of the fit.
%
\begin{figure}
  \centering
  \includegraphics[width=0.35\linewidth]{figures/210414_STXS_VZ_unblinded_XbbVZ_e4179c95_postfitplots/plot_shapes_vhbb_Wen_18_13TeV2018_prefit}
  \includegraphics[width=0.35\linewidth]{figures/210414_STXS_VZ_unblinded_XbbVZ_e4179c95_postfitplots/plot_shapes_vhbb_Wen_18_13TeV2018_postfit} \\
  \includegraphics[width=0.35\linewidth]{figures/210414_STXS_VZ_unblinded_XbbVZ_e4179c95_postfitplots/plot_shapes_vhbb_Wen_20_13TeV2018_prefit}
  \includegraphics[width=0.35\linewidth]{figures/210414_STXS_VZ_unblinded_XbbVZ_e4179c95_postfitplots/plot_shapes_vhbb_Wen_20_13TeV2018_postfit} \\
  \includegraphics[width=0.35\linewidth]{figures/210414_STXS_VZ_unblinded_XbbVZ_e4179c95_postfitplots/plot_shapes_vhbb_Wen_22_13TeV2018_prefit_logy}
  \includegraphics[width=0.35\linewidth]{figures/210414_STXS_VZ_unblinded_XbbVZ_e4179c95_postfitplots/plot_shapes_vhbb_Wen_22_13TeV2018_postfit_logy} \\
  \includegraphics[width=0.35\linewidth]{figures/210414_STXS_VZ_unblinded_XbbVZ_e4179c95_postfitplots/plot_shapes_vhbb_Wen_24_13TeV2018_prefit_logy}
  \includegraphics[width=0.35\linewidth]{figures/210414_STXS_VZ_unblinded_XbbVZ_e4179c95_postfitplots/plot_shapes_vhbb_Wen_24_13TeV2018_postfit_logy} \\
  \caption[$W\rightarrow e\nu$ $V\!Z$ distributions for the boosted selection in 2018]{
    Above are $W\rightarrow e\nu$ pre-fit (left) and post-fit (right) distributions in the boosted selections in 2018
    for the $V\!Z$ analysis.
    The top row shows the $W$ + light jets control region,
    and the second row shows the  $t\bar{t}$ control region.
    The third row shows the signal region for the $\SI{250}{GeV} < p_T(V) < \SI{400}{GeV}$ bin,
    and the bottom row shows the signal region for the bin of $p_T(V) > \SI{400}{GeV}$.
  }
  \label{fig:vz-wen-boost-2018}
\end{figure}

From these post-fit distributions, the strength and uncertainty of the signal samples are extracted.
A value of 1.0 corresponds to the expectation based on simulation of the Standard Model.
By correlating all of the signal samples across STXS bins,
an inclusive signal strength is obtained.
The likelihood scan of this strength is shown in Figure~\ref{fig:vz-inclusive}.
The maximum likelihood with respect to the Standard Model is
$1.182^{+0.118}_{-0.112} \mathrm{(stat)}^{+0.098}_{-0.097} \mathrm{(sys)}$.
The uncertainty of this measurement means the result has a $p$-value of 22\%, assuming Standard Model couplings,
or within 1.2 standard deviations of the Standard Model.
Figure~\ref{fig:vz-stxs} shows the measured kinematic distributions of the various $V\!Z$ processes.
Assuming the Standard Model, the measurement has a $p$-value of 23\%.
The $W\!Z$ STXS bins are consistently slightly above the Standard Model prediction.
When the $V\!Z$ process is split into inclusive $W\!Z$ and $Z\!Z$ parameters though,
the $W\!Z$ process is seen to be 2.1 standard deviations away from the Standard Model.
Overall the $V\!Z$ cross-check is in agreement with the Standard Model.

\begin{figure}
  \centering
  \includegraphics[width=0.7\linewidth]{figures/210309_inclVZ_unblinded_XbbVZ_e4179c95_a866aef8/scan_nominal_r.pdf}
  \caption[Inclusive likelihood scan of $V\!Z$]{
    The likelihood scan of the inclusive signal strength of the
    $V\!Z$ cross check analysis is shown above.
    The dashed line is generated by freezing all nuisance parameters to their most likely value,
    so the likelihood variation for that curve is from statistical uncertainties only.
    Using quadratic subtraction between the two curves allows the
    systematic uncertainties to be evaluated.
  }
  \label{fig:vz-inclusive}
\end{figure}

\begin{figure}
  \centering
  \includegraphics[width=0.8\linewidth]{figures/210308_STXS_VZ_unblinded_XbbVZ_e4179c95_a866aef8/summary_stxs.pdf}
  \caption[Measured STXS values of $V\!Z$]{
    The measured most likely values of all STXS bins in the
    $V\!Z$ cross check analysis.
  }
  \label{fig:vz-stxs}
\end{figure}

\begin{figure}
  \centering
  \includegraphics[width=0.8\linewidth]{figures/210404_VZ_unblinded_XbbVZ_df06b22d_a866aef8/summary_stxs.pdf}
  \caption[Measured values of $W\!Z$ and $Z\!Z$]{
    The measured most likely values of $W\!Z$ and $Z\!Z$ in the
    $V\!Z$ analysis.
  }
  \label{fig:vz-wzzz}
\end{figure}

\subsection{$V\!H$ Combination Fit Results}

To extract the $V\!H$ results, the same procedure is followed as was done in the $V\!Z$ cross-check analysis.
The first step to show that the background processes for $V\!H$ in the detector are well-modeled
by simulation is the Goodness of Fit test, which is shown in Figure~\ref{fig:vh-gof}.
The observed value for the saturated test statistic in data is in the bulk of the distribution created by throwing
post-fit toys, with an associated $p$-value of 70\%.
This suggests both good modeling, since the $p$-value is not low,
and reasonable values for post-fit uncertainties, since the $p$-value is not too high.
%
\begin{figure}
  \centering
  \includegraphics[width=0.65\linewidth]{figures/210308_STXSfine_400split_Xbb_8f854f5a_inclusive_gof/Gof_inclusive_.pdf}
  \caption[Goodness of Fit for $V\!H$]{
    The Goodness of Fit test results are shown for $V\!H$.
    The fit is performed using STXS bins,
    and the test statistic is generated by toys for all control regions and signal regions.
  }
  \label{fig:vh-gof}
\end{figure}

Again, only a sub-set of the 243 pre-fit and post-fit distributions are shown.
Like, the $V\!Z$ analysis, there are too many distributions to show all of them in this work,
so what follows are only a few illustrative examples.
Figure~\ref{fig:vh-znn-high-2016} shows the zero-lepton channel in the high $p_T(V)$ regime.
Unlike the two lepton selections,
the signal distributions display contamination from $W$ + jets background processes.
There is no zero-lepton control region for $W$ + jets.
Instead, these backgrounds are controlled only by one-lepton control regions,
so the five channels are correlated in the fit.
%
\begin{figure}
  \centering
  \includegraphics[width=0.35\linewidth]{figures/210322_STXSfine_400split_unblinded_Xbb_025349b6_postfitplots/plot_shapes_vhbb_Znn_14_13TeV2016_prefit.pdf}
  \includegraphics[width=0.35\linewidth]{figures/210322_STXSfine_400split_unblinded_Xbb_025349b6_postfitplots/plot_shapes_vhbb_Znn_14_13TeV2016_postfit.pdf} \\
  \includegraphics[width=0.35\linewidth]{figures/210322_STXSfine_400split_unblinded_Xbb_025349b6_postfitplots/plot_shapes_vhbb_Znn_15_13TeV2016_prefit.pdf}
  \includegraphics[width=0.35\linewidth]{figures/210322_STXSfine_400split_unblinded_Xbb_025349b6_postfitplots/plot_shapes_vhbb_Znn_15_13TeV2016_postfit.pdf} \\
  \includegraphics[width=0.35\linewidth]{figures/210322_STXSfine_400split_unblinded_Xbb_025349b6_postfitplots/plot_shapes_vhbb_Znn_21_13TeV2016_prefit_logy.pdf}
  \includegraphics[width=0.35\linewidth]{figures/210322_STXSfine_400split_unblinded_Xbb_025349b6_postfitplots/plot_shapes_vhbb_Znn_21_13TeV2016_postfit_logy.pdf} \\
  \includegraphics[width=0.35\linewidth]{figures/210322_STXSfine_400split_unblinded_Xbb_025349b6_postfitplots/plot_shapes_vhbb_Znn_23_13TeV2016_prefit_logy.pdf}
  \includegraphics[width=0.35\linewidth]{figures/210322_STXSfine_400split_unblinded_Xbb_025349b6_postfitplots/plot_shapes_vhbb_Znn_23_13TeV2016_postfit_logy.pdf} \\
  \caption[$Z\rightarrow \nu\nu$ $V\!H$ distributions for high $p_T$ in 2016]{
    Above are $Z\rightarrow \nu\nu$ pre-fit (left) and post-fit distributions (right)
    for the high $p_T$ STXS bins in 2016 in the $V\!H$ measurement using the resolved selection.
    The top row shows the $Z$ + light jets control region, and
    the second row shows the $Z$ + heavy jets control region.
    The third row shows the signal region for the $\SI{250}{GeV} < p_T(V) < \SI{400}{GeV}$ bin,
    and the bottom row shows the signal region for the bin of $p_T(V) > \SI{400}{GeV}$.
  }
  \label{fig:vh-znn-high-2016}
\end{figure}
%
Figure~\ref{fig:vh-wmn-med-2017} shows the $W\rightarrow \mu\nu$ distributions for the medium $p_T(V)$ STXS bin.
Like in the $V\!Z$ cross-check analysis, the single lepton signal selection is
largely made up of $t\bar{t}$ background.
%
\begin{figure}
  \centering
  \includegraphics[width=0.35\linewidth]{figures/210322_STXSfine_400split_unblinded_Xbb_025349b6_postfitplots/plot_shapes_vhbb_Wmn_6_13TeV2017_prefit.pdf}
  \includegraphics[width=0.35\linewidth]{figures/210322_STXSfine_400split_unblinded_Xbb_025349b6_postfitplots/plot_shapes_vhbb_Wmn_6_13TeV2017_postfit.pdf} \\
  \includegraphics[width=0.35\linewidth]{figures/210322_STXSfine_400split_unblinded_Xbb_025349b6_postfitplots/plot_shapes_vhbb_Wmn_7_13TeV2017_prefit.pdf}
  \includegraphics[width=0.35\linewidth]{figures/210322_STXSfine_400split_unblinded_Xbb_025349b6_postfitplots/plot_shapes_vhbb_Wmn_7_13TeV2017_postfit.pdf} \\
  \includegraphics[width=0.35\linewidth]{figures/210322_STXSfine_400split_unblinded_Xbb_025349b6_postfitplots/plot_shapes_vhbb_Wmn_8_13TeV2017_prefit.pdf}
  \includegraphics[width=0.35\linewidth]{figures/210322_STXSfine_400split_unblinded_Xbb_025349b6_postfitplots/plot_shapes_vhbb_Wmn_8_13TeV2017_postfit.pdf} \\
  \includegraphics[width=0.35\linewidth]{figures/210322_STXSfine_400split_unblinded_Xbb_025349b6_postfitplots/plot_shapes_vhbb_Wmn_5_13TeV2017_prefit_logy.pdf}
  \includegraphics[width=0.35\linewidth]{figures/210322_STXSfine_400split_unblinded_Xbb_025349b6_postfitplots/plot_shapes_vhbb_Wmn_5_13TeV2017_postfit_logy.pdf} \\
  \caption[$W\rightarrow \mu\nu$ $V\!H$ distributions for medium $p_T$ in 2017]{
    Above are $W\rightarrow \mu\nu$ pre-fit (left) and post-fit distributions (right)
    for the medium $p_T$ STXS bins in 2017 in the $V\!H$ measurement.
    The top row shows the $W$ + light jets control region, and
    the second row shows the $W$ + heavy jets control region, and
    the third row shows the $t\bar{t}$ control region.
    The bottom row shows the signal region.
  }
  \label{fig:vh-wmn-med-2017}
\end{figure}
%
Finally, Figure~\ref{fig:vh-zee-med-2018} shows the $150 < p_T(V) < \SI{250}{GeV}$ STXS bin
for $Z\rightarrow ee$ in 2018.
In this $p_T(V)$ bin, the $Z\!H$ the signal region is split between events with no extra jets,
and events with extra jets, which can be seen in the distributions.
The other thing to note is that the post-fit signal values are pulled to low values,
and disappear in the post-fit plot.
A deficit of signal-like events is observed in data for events with jets.
Even though there is an excess of signal-like events in events without jets,
the signal strength is affected by the fit in other channels ($\mu\mu$ and $\nu\nu$) and other years.
%
\begin{figure}
  \centering
  \includegraphics[width=0.35\linewidth]{figures/210322_STXSfine_400split_unblinded_Xbb_025349b6_postfitplots/plot_shapes_vhbb_Zee_6_13TeV2018_prefit.pdf}
  \includegraphics[width=0.35\linewidth]{figures/210322_STXSfine_400split_unblinded_Xbb_025349b6_postfitplots/plot_shapes_vhbb_Zee_6_13TeV2018_postfit.pdf} \\
  \includegraphics[width=0.35\linewidth]{figures/210322_STXSfine_400split_unblinded_Xbb_025349b6_postfitplots/plot_shapes_vhbb_Zee_7_13TeV2018_prefit.pdf}
  \includegraphics[width=0.35\linewidth]{figures/210322_STXSfine_400split_unblinded_Xbb_025349b6_postfitplots/plot_shapes_vhbb_Zee_7_13TeV2018_postfit.pdf} \\
  \includegraphics[width=0.35\linewidth]{figures/210322_STXSfine_400split_unblinded_Xbb_025349b6_postfitplots/plot_shapes_vhbb_Zee_5_13TeV2018_prefit_logy.pdf}
  \includegraphics[width=0.35\linewidth]{figures/210322_STXSfine_400split_unblinded_Xbb_025349b6_postfitplots/plot_shapes_vhbb_Zee_5_13TeV2018_postfit_logy.pdf} \\
  \includegraphics[width=0.35\linewidth]{figures/210411_STXSfine_400split_unblinded_Xbb_8f854f5a_postfitplots/plot_shapes_vhbb_Zee_9_13TeV2018_prefit_logy.pdf}
  \includegraphics[width=0.35\linewidth]{figures/210411_STXSfine_400split_unblinded_Xbb_8f854f5a_postfitplots/plot_shapes_vhbb_Zee_9_13TeV2018_postfit_logy.pdf} \\
  \caption[$Z\rightarrow ee$ $V\!H$ distributions for medium $p_T$ in 2018]{
    Above are $Z\rightarrow ee$ pre-fit (left) and post-fit distributions (right)
    for the medium $p_T$ STXS bins in 2018 in the $V\!H$ measurement.
    The top row shows the $Z$ + light jets control region, and
    the second row shows the $Z$ + heavy jets control region.
    The third row shows the signal region with no additional jets, and
    the bottom row shows the signal region with additional jets.
  }
  \label{fig:vh-zee-med-2018}
\end{figure}

The signal strength extracted from the fit can be separated a number of ways.
The inclusive signal strength is shown in Figure~\ref{fig:vh-inclusive} to be
$0.568^{+0.154}_{-0.147} \mathrm{(stat)}^{+0.134}_{-0.133} \mathrm{(sys)}$.
The uncertainties put this measurement within 2.1 standard deviations of the Standard Model.
Figure~\ref{fig:vh-whzh} shows that this low value is caused primarily by a low cross section measurement of $Z\!H$,
which is 2.9 standard deviations below the Standard Model.
This single measured parameter of interest is the largest discrepancy measured from the Standard Model.
Additional investigation splits the production mechanism of the vector boson.
The results of this study are shown in Figure~\ref{fig:vh-ggzh},
where the $ggZ\!H$ process is shown to produce this deviation with the Standard Model.
However, the uncertainty for this process alone is so large that the measured result is
only 1.5 standard deviations below the Standard Model.
Figure~\ref{fig:vh-stxs} shows the full STXS measurement results.
Both $Z\!H$ and $W\!H$ show increasing trends in differential cross section as $p_T(V)$ increases.
However, substituting the inclusive signal strength in for the Standard Model,
the STXS scans have an agreement with a $p$-value of 9.3\%.
Standard Model kinematics cannot be entirely ruled out by this measure.

\begin{figure}
  \centering
  \includegraphics[width=0.7\linewidth]{figures/210309_incl_unblinded_Xbb_8f854f5a_a866aef8/scan_nominal_r.pdf}
  \caption[Inclusive likelihood scan of $V\!H$]{
    The likelihood scan of the inclusive signal strength of the
    $V\!H$ analysis.
  }
  \label{fig:vh-inclusive}
\end{figure}

\begin{figure}
  \centering
  \includegraphics[width=0.8\linewidth]{figures/210309_vh_unblinded_Xbb_8f854f5a_a866aef8/summary_stxs.pdf}
  \caption[Measured values of $W\!H$ and $Z\!H$]{
    The measured most likely values of $W\!H$ and $Z\!H$ in the
    $V\!H$ analysis.
  }
  \label{fig:vh-whzh}
\end{figure}

\begin{figure}
  \centering
  \includegraphics[width=0.8\linewidth]{figures/210413_perproc_unblinded_Xbb_c215af6f_48193845/summary_stxs.pdf}
  \caption[Measured values of $qqW\!H$, $qqZ\!H$, and $ggZ\!H$]{
    The measured most likely values of with the $V\!H$ process split into
    $qqW\!H$, $qqZ\!H$, and $ggZ\!H$.
  }
  \label{fig:vh-ggzh}
\end{figure}

\begin{figure}
  \centering
  \includegraphics[width=0.8\linewidth]{figures/210308_STXSfine_400split_unblinded_Xbb_8f854f5a_a866aef8/summary_stxs.pdf}
  \caption[Measured STXS values of $V\!H$]{
    The measured most likely values of all STXS bins in the
    $V\!H$ analysis.
  }
  \label{fig:vh-stxs}
\end{figure}

The total coupling between the Higgs boson and vector bosons, as well as the coupling between
the Higgs boson and bottom quarks is measured using the kappa framework \cite{de_Blas_2020}.
Results of this analysis alone are shown in Figure~\ref{fig:kappa}.
Standard Model values of $\kappa_V = \kappa_b = 1$ fall within the 68\% confidence level.
The results of this $\kappa$ scan can be additionally constrained by combining
this analysis with others that focus on using either vector bosons for Higgs production
or the $b\bar{b}$ decay path.
%
\begin{figure}
  \centering
  \includegraphics[width=0.7\linewidth]{figures/limit.pdf}
  \caption[Measured values of $\kappa_V$ and $\kappa_b$]{
    The possible values of $\kappa_V$ and $\kappa_b$ are shown above.
    Values of 1.0 reproduce Standard Model predictions.
  }
  \label{fig:kappa}
\end{figure}
%
% The absolute values of the measured cross section are compared to the Standard Model predictions for each
% STXS bin \cite{de2016handbook,Aad:2723187,Aad:2727500} in Table~\ref{tab:xs}.
% %
% \begin{table}
%   \centering
%   \caption[Measured cross sections for each STXS bin.]{
%     The measured cross section for each STXS bin is below.
%     Negative cross sections are unphysical, but despite these,
%     each bin is less than three standard deviations from their respective Standard Model values.
%   }
%   \renewcommand{\arraystretch}{1.5}
%   \begin{tabular}{|c|c|c|c|}
%     \hline
%     Process & $p_T(V)$ & Standard Model [fb] & Measured [fb] \\
%     \hline
%     $W(\ell\nu)H(b\bar{b})$ & $150 \text{---} \SI{250}{GeV}$ & $24.0 \pm 1.1$ & $5.8 \pm 14.4$ \\
%     $W(\ell\nu)H(b\bar{b})$ & $250 \text{---} \SI{400}{GeV}$ & $5.83 \pm 0.26$ & $7.3 \pm 3.2$ \\
%     $W(\ell\nu)H(b\bar{b})$ & $\SI{400}{GeV} \text{---} \infty$ & $1.25 \pm 0.06$ & $2.9 \pm 1.0$ \\
%     \hline
%     $Z(\ell\ell/\nu\nu)H(b\bar{b})$ & $75 \text{---} \SI{150}{GeV}$ & $50.6 \pm 4.1$ & $-12.6 \pm 30.9$ \\
%     $Z(\ell\ell/\nu\nu)H(b\bar{b})$ + no jets & $150 \text{---} \SI{250}{GeV}$ & $12.5 \pm 2.1$ & $-2.4 \pm 5.1$ \\
%     $Z(\ell\ell/\nu\nu)H(b\bar{b})$ + jets & $150 \text{---} \SI{250}{GeV}$ & $6.3 \pm 1.4$ & $-1.4 \pm 5.8$ \\
%     $Z(\ell\ell/\nu\nu)H(b\bar{b})$ & $250 \text{---} \SI{400}{GeV}$ & $4.12 \pm 0.45$ & $2.5 \pm 1.8$ \\
%     $Z(\ell\ell/\nu\nu)H(b\bar{b})$ & $\SI{400}{GeV} \text{---} \infty$ & $0.72 \pm 0.05$ & $0.78 \pm 0.48$ \\
%     \hline
%   \end{tabular}
%   \label{tab:xs}
% \end{table}

The last piece of the analysis shows where future work can improve measurement sensitivity.
Nuisance parameters are gathered into groups of nuisances, and the relative effect of each group on the overall
systematic uncertainty is evaluated.
The results for each of the STXS bins is shown in Table~\ref{tab:groups}.
The first two rows of each table, ``Signal'' and ``Background'', are the theoretical uncertainties applied
to each type of simulation.
These uncertainties can be reduced slightly by using a lepton collider instead of a hadron collider
since there would be no PDF uncertainties in that case.
%
\begin{sidewaystable}
  \centering
  \caption[Contributions of systematic uncertainties groups to result]{
    The contribution of different groups of systematic uncertainties are given below for each of the STXS bins.
    The ``Signal'' and ``Background'' groups are theoretical systematic uncertainties.
    The rest of the groups are results of the experimental apparatus or analysis technique.
    The ``Meas. XS'' uncertainties are related to the measured cross sections of
    the single-top \cite{2017752} and di-boson \cite{2017533}
    processes that are used to normalize backgrounds in this analysis.
    The total systematic uncertainty as well as the statistical uncertainty for each bin are also included
    for comparison.
  }
  \begin{tabular}{|l|c|c|c|}
    \hline
& WH med. & WH high & WH highest \\
\hline
Signal & {+0.060} {-0.000} & {+0.117} {-0.034} & {+0.223} {-0.117}\\
Background & {+0.052} {-0.047} & {+0.053} {-0.025} & {+0.084} {-0.060}\\
\hline
$b$-tagging & {+0.128} {-0.118} & {+0.084} {-0.069} & {+0.092} {-0.077}\\
Jet energy & {+0.225} {-0.203} & {+0.090} {-0.081} & {+0.116} {-0.076}\\
Lepton ID & {+0.022} {-0.000} & {+0.054} {-0.042} & {+0.023} {-0.047}\\
LO to NLO & {+0.047} {-0.038} & {+0.035} {-0.005} & {+0.050} {-0.050}\\
Luminosity & {+0.024} {-0.000} & {+0.045} {-0.000} & {+0.066} {-0.043}\\
Meas. XS & {+0.034} {-0.000} & {+0.039} {-0.000} & {+0.047} {-0.057}\\
Triggers & {+0.011} {-0.000} & {+0.022} {-0.035} & {+0.034} {-0.048}\\
$p_T(V)$ Mig. & {+0.165} {-0.156} & {+0.133} {-0.123} & {+0.305} {-0.293}\\
\hline
Total Sys. & {+0.460} {-0.450} & {+0.362} {-0.333} & {+0.518} {-0.458}\\
Stat & {+0.413} {-0.406} & {+0.447} {-0.430} & {+0.672} {-0.633}\\
\hline
  \end{tabular}
  \begin{tabular}{|l|c|c|c|c|c|}
    \hline
& ZH low & ZH no J & ZH with J & ZH high & ZH highest \\
\hline
Signal & {+0.049} {-0.164} & {+0.040} {-0.103} & {+0.113} {-0.280} & {+0.142} {-0.031} & {+0.130} {-0.025}\\
Background & {+0.038} {-0.050} & {+0.000} {-0.015} & {+0.069} {-0.065} & {+0.027} {-0.027} & {+0.038} {-0.025}\\
\hline
$b$-tagging & {+0.131} {-0.135} & {+0.038} {-0.036} & {+0.098} {-0.093} & {+0.052} {-0.046} & {+0.058} {-0.043}\\
Jet energy & {+0.050} {-0.048} & {+0.019} {-0.030} & {+0.101} {-0.093} & {+0.045} {-0.040} & {+0.070} {-0.069}\\
Lepton ID & {+0.032} {-0.032} & {+0.007} {-0.021} & {+0.000} {-0.067} & {+0.018} {-0.012} & {+0.028} {-0.022}\\
LO to NLO & {+0.154} {-0.162} & {+0.035} {-0.039} & {+0.113} {-0.083} & {+0.044} {-0.040} & {+0.045} {-0.025}\\
Luminosity & {+0.016} {-0.020} & {+0.000} {-0.012} & {+0.017} {-0.014} & {+0.023} {-0.023} & {+0.039} {-0.012}\\
Meas. XS & {+0.000} {-0.016} & {+0.000} {-0.000} & {+0.000} {-0.000} & {+0.026} {-0.016} & {+0.004} {-0.000}\\
Triggers & {+0.024} {-0.010} & {+0.000} {-0.000} & {+0.029} {-0.000} & {+0.000} {-0.010} & {+0.023} {-0.013}\\
$p_T(V)$ Mig. & {+0.362} {-0.387} & {+0.091} {-0.079} & {+0.232} {-0.210} & {+0.133} {-0.121} & {+0.252} {-0.235}\\
\hline
Total Sys. & {+0.529} {-0.553} & {+0.205} {-0.216} & {+0.464} {-0.506} & {+0.274} {-0.220} & {+0.355} {-0.306}\\
Stat & {+0.479} {-0.469} & {+0.360} {-0.346} & {+0.781} {-0.738} & {+0.369} {-0.352} & {+0.620} {-0.563}\\
\hline
  \end{tabular}
  \label{tab:groups}
\end{sidewaystable}
%
However, these theoretical uncertainties have a generally smaller effect than
some of the experimental systematic uncertainties.
In particular, the $p_T(V)$ background migration systematic introduced in Section~\ref{sec:pT-mig}
has the largest effect on the measurement sensitivity.
This systematic effect was assumed to be large since there was no thorough study on this correction.
Future work focusing on this correction can improve the measurement sensitivity in all STXS bins.
Beyond that, the jet energy has a large effect,
which is why there was a dedicated study as outlined in Section~\ref{sec:bjet-energy-corr}.
Finally, the $b$-tagging systematics were more significant than the jet energy for some bins.
Progress on simulation of jets, in particular simulation of heavy-flavor jets,
would be able to reduce both of these systematics as well.
