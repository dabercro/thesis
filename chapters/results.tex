\chapter{Analysis Results} \label{ch:results}

\section{Corrections and Uncertainties}

Despite efforts to simulate LHC collisions as accurately as possible,
a number of differences in the distributions predicted by MC and present in data arise.
This is a result of not being able to predict the beam conditions exactly
and not being able to predict the calibration accurately.
This is made more difficult since the detector degrades in its high radiation environment.

Corrections are made to the simulation by reweighting based on the pileup
and by scaling the predicted energies based on particle type.
Most of the corrections are derived by dedicated groups that provide
the entire CMS collaboration with proper calibrations.
This analysis does depend in particular on predictions of $b$-jet energies,
which are not corrected centrally.
Therefore this section includes a description of how that correction is derived.

\subsection{Muons}

\subsection{Electrons}

\subsection{Jets and MET}

\subsection{$b$-Jet Energy Correction}

Even when distributions of individual variables agree between MC and data,
correlations are often different.
These correlations are also important in the evaluation of a DNN.
The DNN used to estimate the energy of $b$-jets therefore has differing performance
in MC and data.
In particular, it is better at estimating the true energy of a b-jet in MC.
The energies evaulated in MC must be smeared in order to accurately simulate
the resolution of jets in data after they have been modified by the DNN regression.

One way to measure jet energy resolution is to consider an event
where a jet is recoiling off of a Z boson that decays into leptons.
In principle, the Z boson's transverse momentum is balanced with the
jet's transverse momentum.
Measurements of lepton energies in the CMS detector is relatively precise,
so the ratio of the reconstructed jet's
$p_T$ to the Z boson's $p_T$ allows measurement of the jet energy resolution.
Ideally, this measurement would be done with an collision resulting in one $Z$ boson decay,
and one jet.
However, this is an infrequent occurrence.
Instead, events with two jets are selected, with one jet having relatively low $p_T$.
A fit is performed to estimate resolution characteristics
where the second jet would have $p_T = \SI{0}{GeV}$.

These events are selected using the following requirements:

\begin{itemize}
\item Exactly two muons or two electrons must pass the selection criteria for the
  di-leptons channels described in Section~\ref{sec:resolved-2}.
\item The two selected leptons must be oppositely charged.
\item The di-lepton kinematics must satisfy \\ $p_{T,\ell\ell} > \SI{100}{GeV}$ and
  $\SI{71}{GeV} < m_{\ell\ell} < \SI{111}{GeV}$.
\item Exactly two jets must pass the preselection described in Section~\ref{sec:resolved-2}.
\item The leading jet must also satisfy $\Delta\phi(j, \ell\ell) > 2.8$
\item The ratio between the sub-leading jet $p_T$ and
  the di-lepton $p_T$ must be less than 0.3.
\end{itemize}

\section{Combination Fit Method}

\section{Results}
