\chapter{Simulation} \label{ch:simulation}

After the detector is well understood,
predictions on how it responds in the LHC enviroment can be made.
The number of ways the detector could possibly respond are nearly infinite.
Therefore, simulation is performed using Monte Carlo methods,
and the resulting analysis is statistical in nature.
The data format for simulation results is similar to the data format for
data collected from the detector.
Unobservable information about intermediate steps
in the simulation is also stored,
but otherwise the data is the predicted output of a collection of events.

The simulation itself consists of several steps
each outlined in a separate section of this chapter.
First the background processes that will appear in our analysis must be known.
Identifying all of these processes is necessary to quantify and characterize
the signal events that are also mixed in to our selection.
Then each of these processes must be simulated to determine the final state particles
that the detector will observe.
Each process will looks slightly different in our signal selection.
Events outside of the selection must also be simulated so that they can be studied
for accuracy in separate phase spaces that do not include the Higgs events.
After the final state particles are predicted,
the detector response to those particles passing through must be simulated.
This allows researchers to compare the physical readouts
they can observe to predicted detector signals.
Finally, using the phase spaces outside of the desired signal process,
minor corrections to the simulation can be made.
Simulated energies from the detector model might not be the exact same as
what the physical detector produces, for example,
and they must be made to match to make the signal process cleanly appear in the analysis.

\section{Backgrounds to the Analysis}

In order to effectively measure Higgs production,
we need to be able to accurately estimate other events
that end up in our selection.
For example, the final state for the two lepton decay in Figure~\ref{fig:two-lep-diagram}
can also be achieved by a Drell-Yan process radiating jets
or a $t\bar{t}$ event where both $W$ bosons from the top decays decay leptonically.
Feynman diagrams in Figure~\ref{fig:dy-2lep} and Figure~\ref{fig:tt-2lep}
show how the two respective processes can result in
the same final state as the signal process.
The Drell-Yan process can also radiate jets initiated by lighter flavor quarks
that are mistakenly identified as $b$-jets,
and those make up a significant portion of the backgrounds as well.
Less significant, but still important backgrounds include processes like
di-boson production, QCD jets, and signal top processes.

\begin{figure}
  \centering
  \begin{fmffile}{dy_2lep}
    \fmfframe(0,0)(0, 20){
    \begin{fmfgraph*}(250, 150)
      \fmfleft{i0,i1}
      \fmfright{o2,o3,o0,o1}
      \fmf{quark}{i1,v0,v1,i0}
      \fmflabel{$\bar{f}$}{i0}
      \fmflabel{$f$}{i1}
      \fmf{boson, label=$Z$}{v0,v2}
      \fmf{fermion}{o0,v2,o1}
      \fmflabel{$\ell^+$}{o0}
      \fmflabel{$\ell^-$}{o1}
      \fmf{gluon, label=$g$}{v1,v3}
      \fmf{fermion}{o2,v3,o3}
      \fmflabel{$\bar{b}$}{o2}
      \fmflabel{$b$}{o3}
    \end{fmfgraph*}
    }
  \end{fmffile}
  \caption[Feynman diagram for DY + jets background]{
    Above is the Feynman diagram matching the two lepton final state coming from
    Drell-Yan and jets.
  }
  \label{fig:dy-2lep}
\end{figure}

\begin{figure}
  \centering
  \begin{fmffile}{tt_2lep}
    \fmfframe(0,0)(0, 20){
    \begin{fmfgraph*}(250, 150)
      \fmfleft{i0,i1}
      \fmfright{o0,o1,o2,o3,o4,o5}
      \fmf{quark}{i1,v0,i0}
      \fmflabel{$\bar{f}$}{i0}
      \fmflabel{$f$}{i1}
      \fmf{gluon, label=$g$}{v0,v1}
      \fmf{fermion, label=$\bar{t}$}{v2,v1}
      \fmf{fermion, label=$t$}{v1,v3}
      \fmf{fermion}{v2,o0}
      \fmflabel{$\bar{b}$}{o0}
      \fmf{boson, label=$W^-$}{v2,v4}
      \fmf{fermion}{o1,v4,o2}
      \fmflabel{$\bar{\nu}$}{o1}
      \fmflabel{$\ell^-$}{o2}
      \fmf{fermion}{v3,o5}
      \fmflabel{$b$}{o5}
      \fmf{boson, label=$W^+$}{v3,v5}
      \fmf{fermion}{o3,v5,o4}
      \fmflabel{$\ell^+$}{o3}
      \fmflabel{$\nu$}{o4}
    \end{fmfgraph*}
    }
  \end{fmffile}
  \caption[Feynman diagram for $t\bar{t}$ background]{
    Above is the Feynman diagram matching the two lepton final state coming from
    fully leptonic $t\bar{t}$ decay.
    In events with little energy carried away by the neutrinos,
    this can appear to be the same as the two-lepton signal process.
  }
  \label{fig:tt-2lep}
\end{figure}

The backgrounds for the one- and zero-lepton signal decay channels are
caused by similar processes.
For the one-lepton decays of $WH$, the Drell-Yan background in Figure~\ref{fig:dy-2lep}
is replaced with a flavor changing current of $W$ + jets.
The $t\bar{t}$ background would instead be caused by either a hadronic decay of
one of the $W$ bosons in Figure~\ref{fig:tt-2lep}
or by one of the pictured leptons travelling out of the detector without being observed.
For the zero-lepton channel, the Drell Yan process is instead replaced with
$Z \rightarrow \nu\bar{\nu}$.
The $t\bar{t}$ process still needs high MET in order to appear to
contain a hard neutrino presence,
so it is most often caused when a single $W$ decays leptonically with the lepton
falling outside of the detector acceptance.
For both of these channels, di-boson, QCD, and single top backgrounds can also contribute.
This simplifies the methods needed to generate Monte Carlo samples,
since the decay mode of each intermediate particle can be randomly selected for each trial.

\section{Event Particle Generation}

The physical processes that occur at the LHC all contain QCD-driven phenomena.
As a result, the part of the simulation that predicts the particles present in the detector
has two distinct parts.
QCD is perturbative at small distances,
and other forces are perturbative at all distances.
The collisions themselves are in this regime,
so the initial- and final-state particles over a distance of femtometers
can be simulated using typical calculations using
perturbative rules described by Feynman diagrams.
The decay of unstable particles can also be simulated this way.
Once particles interacting through QCD exceed this distance,
well before reaching the detector,
hadronization, or parton showers, must be simulated differently.
There are two different sub-sections describing these techniques.
The exact generators and configurations used to simulate each process for this analysis
are detailed in Appendix~\ref{app:generator}.

\subsection{Short-Scale Simulation}

Events are generated by selecting results and assigning weights in a way proportional
to the phase space and the matrix element squared of the event.
The phase space integral has the following form \cite{Peskin:257493}.
\begin{gather}
  \int d\Pi_n = \left( \prod_f \int \frac{d^3p_f}{(2\pi)^3} \frac{1}{2E_f} \right)
                (2\pi)^4 \delta^{(4)}(P - \sum p_f)
\end{gather}
$P$ is the total initial 4-momentum and $f$ runs over all final state particles.
This phase space integral is Lorentz invariant.
Once an event has been selected, the available phase space can then be used to assign
directions to the final particles.

The proportional matrix elements are described in Chapter~\ref{ch:theory},
but most of the diagrams described were Leading Order (LO).
Generators used in this analysis can also simulate Next to Leading Order (NLO) processes
thanks to the FKS method of subtracting particles to avoid double counting them during the
showering calcuation.
However, the option to use NLO simulation is not always used.
The results of NLO calculations more accurately predict physical processes,
but they also take more computational resources,
resulting in larger measurement uncertainties due to statistical limitations.

The generators used for short-scale simulation in this analysis
are POWHEG \cite{Oleari_2010} and MadGraph5 \cite{hirschi2015automated}.

\subsection{Parton Showers}

The final state particles from the short-scale simulation
include a number of free quarks and gluons.
As mentioned in Section~\ref{sec:produce-vector},
it is energetically favorable for color-charged particles to create additional
quark/anti-quark pairs to screen the color charge.
This is known as hadronization or parton showering and
is simulated separately from the calculation of tree-level processes.
Hadronization happens well before particles reach the CMS detector,
so the results are needed to predict the detector response.
Accurate simulation of this process is important for all collisions at the LHC,
which produces much QCD background.
It is also important to accurately simulate the constituents of individual jets
since this analysis includes detailed inspection of each jet
in order to identify $b$-jets and to estimate the amount of energy carried away by neutrinos.

To be able to analyze the simulation in the same way data is processed,
Monte Carlo simulation is used to predict precise final states of the jets.
CMS uses the Lund model \cite{ANDERSSON198331} as implemented in
\texttt{PYTHIA8} \cite{SJOSTRAND2015159}.

\section{Detector Simulation}

After determining all of the final state particles that will reach the detector,
the interaction between these particles and the detector components must be simulated.
Multiple simulations of $pp$ collisions are combined to simulate pileup,
and then the particle propagation through various materials is done with
\texttt{GEANT4} \cite{AGOSTINELLI2003250}.
The full CMS detector geometry is maintained within CMSSW \cite{Hildreth_2015}
using a framework written in the Unified Modelling Language \cite{Lefebure:687188}.
To be able to process the simulated data in the exact same fashion as the measured data,
the readout of the electronics is also simulated.

\section{Corrections to Simulation}

\subsection{Muons}

\subsection{Electrons}

\subsection{Jets and MET}
